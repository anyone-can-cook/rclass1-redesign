% Options for packages loaded elsewhere
\PassOptionsToPackage{unicode}{hyperref}
\PassOptionsToPackage{hyphens}{url}
\PassOptionsToPackage{dvipsnames,svgnames,x11names}{xcolor}
%
\documentclass[
  8pt,
  ignorenonframetext,
  dvipsnames]{beamer}
\usepackage{pgfpages}
\setbeamertemplate{caption}[numbered]
\setbeamertemplate{caption label separator}{: }
\setbeamercolor{caption name}{fg=normal text.fg}
\beamertemplatenavigationsymbolsempty
% Prevent slide breaks in the middle of a paragraph
\widowpenalties 1 10000
\raggedbottom
\setbeamertemplate{part page}{
  \centering
  \begin{beamercolorbox}[sep=16pt,center]{part title}
    \usebeamerfont{part title}\insertpart\par
  \end{beamercolorbox}
}
\setbeamertemplate{section page}{
  \centering
  \begin{beamercolorbox}[sep=12pt,center]{part title}
    \usebeamerfont{section title}\insertsection\par
  \end{beamercolorbox}
}
\setbeamertemplate{subsection page}{
  \centering
  \begin{beamercolorbox}[sep=8pt,center]{part title}
    \usebeamerfont{subsection title}\insertsubsection\par
  \end{beamercolorbox}
}
\AtBeginPart{
  \frame{\partpage}
}
\AtBeginSection{
  \ifbibliography
  \else
    \frame{\sectionpage}
  \fi
}
\AtBeginSubsection{
  \frame{\subsectionpage}
}
\usepackage{amsmath,amssymb}
\usepackage{iftex}
\ifPDFTeX
  \usepackage[T1]{fontenc}
  \usepackage[utf8]{inputenc}
  \usepackage{textcomp} % provide euro and other symbols
\else % if luatex or xetex
  \usepackage{unicode-math} % this also loads fontspec
  \defaultfontfeatures{Scale=MatchLowercase}
  \defaultfontfeatures[\rmfamily]{Ligatures=TeX,Scale=1}
\fi
\usepackage{lmodern}
\ifPDFTeX\else
  % xetex/luatex font selection
\fi
% Use upquote if available, for straight quotes in verbatim environments
\IfFileExists{upquote.sty}{\usepackage{upquote}}{}
\IfFileExists{microtype.sty}{% use microtype if available
  \usepackage[]{microtype}
  \UseMicrotypeSet[protrusion]{basicmath} % disable protrusion for tt fonts
}{}
\makeatletter
\@ifundefined{KOMAClassName}{% if non-KOMA class
  \IfFileExists{parskip.sty}{%
    \usepackage{parskip}
  }{% else
    \setlength{\parindent}{0pt}
    \setlength{\parskip}{6pt plus 2pt minus 1pt}}
}{% if KOMA class
  \KOMAoptions{parskip=half}}
\makeatother
\usepackage{xcolor}
\newif\ifbibliography
\usepackage{color}
\usepackage{fancyvrb}
\newcommand{\VerbBar}{|}
\newcommand{\VERB}{\Verb[commandchars=\\\{\}]}
\DefineVerbatimEnvironment{Highlighting}{Verbatim}{commandchars=\\\{\}}
% Add ',fontsize=\small' for more characters per line
\usepackage{framed}
\definecolor{shadecolor}{RGB}{248,248,248}
\newenvironment{Shaded}{\begin{snugshade}}{\end{snugshade}}
\newcommand{\AlertTok}[1]{\textcolor[rgb]{0.94,0.16,0.16}{#1}}
\newcommand{\AnnotationTok}[1]{\textcolor[rgb]{0.56,0.35,0.01}{\textbf{\textit{#1}}}}
\newcommand{\AttributeTok}[1]{\textcolor[rgb]{0.13,0.29,0.53}{#1}}
\newcommand{\BaseNTok}[1]{\textcolor[rgb]{0.00,0.00,0.81}{#1}}
\newcommand{\BuiltInTok}[1]{#1}
\newcommand{\CharTok}[1]{\textcolor[rgb]{0.31,0.60,0.02}{#1}}
\newcommand{\CommentTok}[1]{\textcolor[rgb]{0.56,0.35,0.01}{\textit{#1}}}
\newcommand{\CommentVarTok}[1]{\textcolor[rgb]{0.56,0.35,0.01}{\textbf{\textit{#1}}}}
\newcommand{\ConstantTok}[1]{\textcolor[rgb]{0.56,0.35,0.01}{#1}}
\newcommand{\ControlFlowTok}[1]{\textcolor[rgb]{0.13,0.29,0.53}{\textbf{#1}}}
\newcommand{\DataTypeTok}[1]{\textcolor[rgb]{0.13,0.29,0.53}{#1}}
\newcommand{\DecValTok}[1]{\textcolor[rgb]{0.00,0.00,0.81}{#1}}
\newcommand{\DocumentationTok}[1]{\textcolor[rgb]{0.56,0.35,0.01}{\textbf{\textit{#1}}}}
\newcommand{\ErrorTok}[1]{\textcolor[rgb]{0.64,0.00,0.00}{\textbf{#1}}}
\newcommand{\ExtensionTok}[1]{#1}
\newcommand{\FloatTok}[1]{\textcolor[rgb]{0.00,0.00,0.81}{#1}}
\newcommand{\FunctionTok}[1]{\textcolor[rgb]{0.13,0.29,0.53}{\textbf{#1}}}
\newcommand{\ImportTok}[1]{#1}
\newcommand{\InformationTok}[1]{\textcolor[rgb]{0.56,0.35,0.01}{\textbf{\textit{#1}}}}
\newcommand{\KeywordTok}[1]{\textcolor[rgb]{0.13,0.29,0.53}{\textbf{#1}}}
\newcommand{\NormalTok}[1]{#1}
\newcommand{\OperatorTok}[1]{\textcolor[rgb]{0.81,0.36,0.00}{\textbf{#1}}}
\newcommand{\OtherTok}[1]{\textcolor[rgb]{0.56,0.35,0.01}{#1}}
\newcommand{\PreprocessorTok}[1]{\textcolor[rgb]{0.56,0.35,0.01}{\textit{#1}}}
\newcommand{\RegionMarkerTok}[1]{#1}
\newcommand{\SpecialCharTok}[1]{\textcolor[rgb]{0.81,0.36,0.00}{\textbf{#1}}}
\newcommand{\SpecialStringTok}[1]{\textcolor[rgb]{0.31,0.60,0.02}{#1}}
\newcommand{\StringTok}[1]{\textcolor[rgb]{0.31,0.60,0.02}{#1}}
\newcommand{\VariableTok}[1]{\textcolor[rgb]{0.00,0.00,0.00}{#1}}
\newcommand{\VerbatimStringTok}[1]{\textcolor[rgb]{0.31,0.60,0.02}{#1}}
\newcommand{\WarningTok}[1]{\textcolor[rgb]{0.56,0.35,0.01}{\textbf{\textit{#1}}}}
\usepackage{longtable,booktabs,array}
\usepackage{calc} % for calculating minipage widths
\usepackage{caption}
% Make caption package work with longtable
\makeatletter
\def\fnum@table{\tablename~\thetable}
\makeatother
\setlength{\emergencystretch}{3em} % prevent overfull lines
\providecommand{\tightlist}{%
  \setlength{\itemsep}{0pt}\setlength{\parskip}{0pt}}
\setcounter{secnumdepth}{-\maxdimen} % remove section numbering

%packages
\usepackage{graphicx}
\usepackage{rotating}
\usepackage{hyperref}

\usepackage{tikz} % used for text highlighting, amongst others
\usepackage{comment}

%title slide stuff
%\institute{Department of Education}
%\title{Managing and Manipulating Data Using R}

%
\setbeamertemplate{navigation symbols}{} % get rid of navigation icons:
\setbeamertemplate{footline}[page number]

%\setbeamertemplate{frametitle}{\thesection \hspace{0.2cm} \insertframetitle}
\setbeamertemplate{section in toc}[sections numbered]
%\setbeamertemplate{subsection in toc}[subsections numbered]
\setbeamertemplate{subsection in toc}{%
  \leavevmode\leftskip=3.2em\color{gray}\rlap{\hskip-2em\inserttocsectionnumber.\inserttocsubsectionnumber}\inserttocsubsection\par
}

%define colors
%\definecolor{uva_orange}{RGB}{216,141,42} % UVa orange (Rotunda orange)
\definecolor{mygray}{rgb}{0.95, 0.95, 0.95} % for highlighted text
	% grey is equal parts red, green, blue. higher values >> lighter grey
	%\definecolor{lightgraybo}{rgb}{0.83, 0.83, 0.83}

% new commands

%highlight text with very light grey
\newcommand*{\hlg}[1]{%
	\tikz[baseline=(X.base)] \node[rectangle, fill=mygray] (X) {#1};%
}
%, inner sep=0.3mm
%highlight text with very light grey and use font associated with code
\newcommand*{\hlgc}[1]{\texttt{\hlg{#1}}}

%modifying back ticks to add grey background
\let\OldTexttt\texttt
\renewcommand{\texttt}[1]{\OldTexttt{\hlg{#1}}}


\begin{comment}

% Font
\usepackage[defaultfam,light,tabular,lining]{montserrat}
\usepackage[T1]{fontenc}
\renewcommand*\oldstylenums[1]{{\fontfamily{Montserrat-TOsF}\selectfont #1}}

% Change color of boldface text to darkgray
\renewcommand{\textbf}[1]{{\color{darkgray}\bfseries\fontfamily{Montserrat-TOsF}#1}}

% Bullet points
\setbeamertemplate{itemize item}{\color{BlueViolet}$\circ$}
\setbeamertemplate{itemize subitem}{\color{BrickRed}$\triangleright$}
\setbeamertemplate{itemize subsubitem}{$-$}

% Reduce space before lists
%\addtobeamertemplate{itemize/enumerate body begin}{}{\vspace*{-8pt}}

\let\olditem\item
\renewcommand{\item}{%
  \olditem\vspace{4pt}
}

% decreasing space before and after level-2 bullet block
%\addtobeamertemplate{itemize/enumerate subbody begin}{}{\vspace*{-3pt}}
%\addtobeamertemplate{itemize/enumerate subbody end}{}{\vspace*{-3pt}}

% decreasing space before and after level-3 bullet block
%\addtobeamertemplate{itemize/enumerate subsubbody begin}{}{\vspace*{-2pt}}
%\addtobeamertemplate{itemize/enumerate subsubbody end}{}{\vspace*{-2pt}}

\end{comment}

%Section numbering
\setbeamertemplate{section page}{%
    \begingroup
        \begin{beamercolorbox}[sep=10pt,center,rounded=true,shadow=true]{section title}
        \usebeamerfont{section title}\thesection~\insertsection\par
        \end{beamercolorbox}
    \endgroup
}

\setbeamertemplate{subsection page}{%
    \begingroup
        \begin{beamercolorbox}[sep=6pt,center,rounded=true,shadow=true]{subsection title}
        \usebeamerfont{subsection title}\thesection.\thesubsection~\insertsubsection\par
        \end{beamercolorbox}
    \endgroup
}
\ifLuaTeX
  \usepackage{selnolig}  % disable illegal ligatures
\fi
\IfFileExists{bookmark.sty}{\usepackage{bookmark}}{\usepackage{hyperref}}
\IfFileExists{xurl.sty}{\usepackage{xurl}}{} % add URL line breaks if available
\urlstyle{same}
\hypersetup{
  pdftitle={Enter the tidyverse: Processing across rows},
  colorlinks=true,
  linkcolor={Maroon},
  filecolor={Maroon},
  citecolor={Blue},
  urlcolor={blue},
  pdfcreator={LaTeX via pandoc}}

\title{Enter the tidyverse: Processing across rows}
\subtitle{Managing and Manipulating Data Using R}
\author{}
\date{\vspace{-2.5em}}

\begin{document}
\frame{\titlepage}

\hypertarget{introduction}{%
\section{Introduction}\label{introduction}}

\begin{frame}{Lecture outline}
\protect\hypertarget{lecture-outline}{}
\tableofcontents
\end{frame}

\begin{frame}[fragile]{Libraries we will use today}
\protect\hypertarget{libraries-we-will-use-today}{}
``Load'' the package we will use today (output omitted)

\begin{itemize}
\tightlist
\item
  \textbf{you must run this code chunk}
\end{itemize}

\begin{Shaded}
\begin{Highlighting}[]
\FunctionTok{library}\NormalTok{(tidyverse)}
\end{Highlighting}
\end{Shaded}

If package not yet installed, then must install before you load. Install
in ``console'' rather than .Rmd file

\begin{itemize}
\tightlist
\item
  Generic syntax: \texttt{install.packages("package\_name")}
\item
  Install ``tidyverse'': \texttt{install.packages("tidyverse")}
\end{itemize}

Note: when we load package, name of package is not in quotes; but when
we install package, name of package is in quotes:

\begin{itemize}
\tightlist
\item
  \texttt{install.packages("tidyverse")}
\item
  \texttt{library(tidyverse)}
\end{itemize}
\end{frame}

\begin{frame}[fragile]{Data we will use today}
\protect\hypertarget{data-we-will-use-today}{}
Data on off-campus recruiting events by public universities

\begin{itemize}
\tightlist
\item
  Object \texttt{df\_event}

  \begin{itemize}
  \tightlist
  \item
    One observation per university, recruiting event
  \end{itemize}
\end{itemize}

\begin{Shaded}
\begin{Highlighting}[]
\FunctionTok{rm}\NormalTok{(}\AttributeTok{list =} \FunctionTok{ls}\NormalTok{()) }\CommentTok{\# remove all objects}

\CommentTok{\#load dataset with one obs per recruiting event}
\FunctionTok{load}\NormalTok{(}\FunctionTok{url}\NormalTok{(}\StringTok{"https://github.com/ozanj/rclass/raw/master/data/recruiting/recruit\_event\_somevars.RData"}\NormalTok{))}
\CommentTok{\#load("../../data/recruiting/recruit\_event\_allvars.Rdata")}
\end{Highlighting}
\end{Shaded}
\end{frame}

\begin{frame}[fragile]{Processing across variables vs.~processing across
observations}
\protect\hypertarget{processing-across-variables-vs.-processing-across-observations}{}
Visits by UC Berkeley to public high schools

\begin{verbatim}
#> # A tibble: 5 x 6
#>   school_id    state tot_stu_pub fr_lunch pct_fr_lunch med_inc
#>   <chr>        <chr>       <dbl>    <dbl>        <dbl>   <dbl>
#> 1 340882002126 NJ           1846       29       0.0157 178732 
#> 2 340147000250 NJ           1044       50       0.0479  62288 
#> 3 340561003796 NJ           1505      298       0.198  100684.
#> 4 340165005124 NJ           1900       43       0.0226 160476.
#> 5 341341003182 NJ           1519      130       0.0856 144346
\end{verbatim}

\medskip

So far, we have focused on ``processing across variables'\,'

\begin{itemize}
\tightlist
\item
  Performing calculations across columns (i.e., vars), typically within
  a row (i.e., observation)
\item
  Example: percent free-reduced lunch (above)
\end{itemize}

\medskip

Processing across obs (focus of today's lecture)

\begin{itemize}
\tightlist
\item
  Performing calculations across rows (i.e., obs), often within a column
  (i.e., variable)
\item
  Example: Average household income of visited high schools, by state
\end{itemize}
\end{frame}

\begin{frame}[fragile]{Why processing across observations}
\protect\hypertarget{why-processing-across-observations}{}
\textbf{Note}

\begin{itemize}
\tightlist
\item
  In today's lecture, I'll use the terms ``observations'' and ``rows''
  interchangeably
\end{itemize}

\textbf{Creation of analysis datasets often requires calculations across
obs}

Examples:

\begin{itemize}
\tightlist
\item
  You have a dataset with one observation per student-term and want to
  create a variable of credits attempted per term
\item
  You have a dataset with one observation per student-term and want to
  create a variable of GPA for the semester or cumulative GPA for all
  semesters
\item
  Number of off-campus recruiting events university makes to each state
\item
  Average household income at visited versus non-visited high schools
\end{itemize}

\textbf{Creating graphs and tables of descripive stats usually require
calculations across obs}

Example: Want to create a graph that shows number of recruiting events
by event ``type'' (e.g., public HS, private HS) for each university

\begin{itemize}
\tightlist
\item
  Start with \texttt{df\_event} dataset that has one obervation per
  university, recruiting event
\item
  Create new data frame object that has one observation per university
  and event type and has variable for number of events

  \begin{itemize}
  \tightlist
  \item
    this variable calculated by counting number of rows in each
    combination of university and event type
  \end{itemize}
\item
  This new data frame object is the input for creating desired graph
\end{itemize}
\end{frame}

\hypertarget{introduce-group_by-and-summarize}{%
\section{Introduce group\_by() and
summarize()}\label{introduce-group_by-and-summarize}}

\begin{frame}[fragile]{Strategy for teaching processing across obs}
\protect\hypertarget{strategy-for-teaching-processing-across-obs}{}
In \texttt{tidyverse} the \texttt{group\_by()} and \texttt{summarize()}
functions are the primary means of performing calculations across
observations

\begin{itemize}
\tightlist
\item
  Usually, processing across observations requires using
  \texttt{group\_by()} and \texttt{summarize()} together
\item
  \texttt{group\_by()} typically not very useful by itself
\item
  \texttt{summarize()} {[}with or without \texttt{group\_by()}{]} can be
  helpful for creating summary statistics that are the inputs for tables
  or graphs you create
\end{itemize}

How we'll teach:

\begin{itemize}
\tightlist
\item
  introduce \texttt{group\_by()} and \texttt{summarize()} separately

  \begin{itemize}
  \tightlist
  \item
    goal: you understand what each function does
  \end{itemize}
\item
  then we'll combine them
\end{itemize}
\end{frame}

\hypertarget{group_by}{%
\subsection{group\_by}\label{group_by}}

\begin{frame}[fragile]{group\_by()}
\protect\hypertarget{group_by-1}{}
\textbf{Description}: ``\texttt{group\_by()} takes an existing data
frame and converts it into a grouped data frame where operations are
performed''by group''. \texttt{ungroup()} removes grouping.''

\begin{itemize}
\tightlist
\item
  part of \textbf{dplyr} package within \textbf{tidyverse}; not part of
  \textbf{Base R}
\item
  works best with pipes \texttt{\%\textgreater{}\%} and
  \texttt{summarize()} function {[}described below{]}
\end{itemize}

\textbf{Basic syntax}: \texttt{group\_by(.data,\ ...)}

\begin{itemize}
\tightlist
\item
  \texttt{.data} argument refers to name of data frame
\item
  \texttt{...} argument refers to names of ``group\_by'' variables,
  separated by commas

  \begin{itemize}
  \tightlist
  \item
    Can ``group by'' one or many variables
  \item
    Typically, ``group\_by'' variables are character, factor, or integer
    variables
  \end{itemize}
\end{itemize}

\medskip

Possible ``group by'' variables in \texttt{df\_event} data:

\begin{itemize}
\tightlist
\item
  university name/id; event type (e.g., public HS, private HS); state
\end{itemize}

\textbf{Example}: in \texttt{df\_event}, create frequency count of
\texttt{event\_type} {[}output omitted{]}

\begin{Shaded}
\begin{Highlighting}[]
\FunctionTok{names}\NormalTok{(df\_event)}
\CommentTok{\#without group\_by()}
\NormalTok{df\_event }\SpecialCharTok{\%\textgreater{}\%} \FunctionTok{count}\NormalTok{(event\_type)}
\NormalTok{df\_event }\SpecialCharTok{\%\textgreater{}\%} \FunctionTok{count}\NormalTok{(instnm)}
\CommentTok{\#group\_by() university}
\NormalTok{df\_event }\SpecialCharTok{\%\textgreater{}\%} \FunctionTok{group\_by}\NormalTok{(instnm) }\SpecialCharTok{\%\textgreater{}\%} \FunctionTok{count}\NormalTok{(event\_type)}
\end{Highlighting}
\end{Shaded}
\end{frame}

\begin{frame}[fragile]{\texttt{group\_by()}}
\protect\hypertarget{group_by-2}{}
By itself \texttt{group\_by()} doesn't do much; it just prints data

\begin{itemize}
\tightlist
\item
  Below, group \texttt{df\_event} data by university, event type, and
  event state
\end{itemize}

\begin{Shaded}
\begin{Highlighting}[]
\CommentTok{\#group\_by (without pipes)}
\FunctionTok{group\_by}\NormalTok{(df\_event, univ\_id, event\_type, event\_state)}
\CommentTok{\#group\_by (with pipes)}
\NormalTok{df\_event }\SpecialCharTok{\%\textgreater{}\%} \FunctionTok{group\_by}\NormalTok{(univ\_id, event\_type, event\_state) }
\NormalTok{df\_event }\SpecialCharTok{\%\textgreater{}\%} \FunctionTok{group\_by}\NormalTok{(univ\_id, event\_type, event\_state) }\SpecialCharTok{\%\textgreater{}\%} \FunctionTok{glimpse}\NormalTok{()}
\end{Highlighting}
\end{Shaded}

But once an object is grouped, all subsequent functions are run
separately ``by group''

\begin{itemize}
\tightlist
\item
  recall that \texttt{count()} counts number of observations by group
\end{itemize}

\begin{Shaded}
\begin{Highlighting}[]
\CommentTok{\# count number of observations in group, ungrouped data}
\NormalTok{df\_event  }\SpecialCharTok{\%\textgreater{}\%} \FunctionTok{count}\NormalTok{()}
\CommentTok{\#group by and then count obs}
\NormalTok{df\_event }\SpecialCharTok{\%\textgreater{}\%} \FunctionTok{group\_by}\NormalTok{(univ\_id) }\SpecialCharTok{\%\textgreater{}\%} \FunctionTok{count}\NormalTok{()}
\NormalTok{df\_event }\SpecialCharTok{\%\textgreater{}\%} \FunctionTok{group\_by}\NormalTok{(univ\_id) }\SpecialCharTok{\%\textgreater{}\%} \FunctionTok{count}\NormalTok{() }\SpecialCharTok{\%\textgreater{}\%} \FunctionTok{glimpse}\NormalTok{()}
  \CommentTok{\#df\_event \%\textgreater{}\% group\_by(univ\_id) \%\textgreater{}\% count() \%\textgreater{}\% View()}

\NormalTok{df\_event }\SpecialCharTok{\%\textgreater{}\%} \FunctionTok{group\_by}\NormalTok{(univ\_id, event\_type) }\SpecialCharTok{\%\textgreater{}\%} \FunctionTok{count}\NormalTok{()}
\NormalTok{df\_event }\SpecialCharTok{\%\textgreater{}\%} \FunctionTok{group\_by}\NormalTok{(univ\_id, event\_type) }\SpecialCharTok{\%\textgreater{}\%} \FunctionTok{count}\NormalTok{() }\SpecialCharTok{\%\textgreater{}\%} \FunctionTok{glimpse}\NormalTok{()}
  \CommentTok{\#df\_event \%\textgreater{}\% group\_by(univ\_id, event\_type) \%\textgreater{}\% count() \%\textgreater{}\% View()}
\end{Highlighting}
\end{Shaded}
\end{frame}

\begin{frame}[fragile]{Grouping not retained unless you \textbf{assign}
it}
\protect\hypertarget{grouping-not-retained-unless-you-assign-it}{}
Below, we'll use \texttt{class()} function to show whether data frame is
grouped

\begin{itemize}
\tightlist
\item
  will talk more about \texttt{class()} next week, but for now, just
  think of it as a function that provides information about an object
\item
  similar to \texttt{typeof()}, but \texttt{class()} provides different
  info about object
\end{itemize}

\medskip

Grouping is not retained unless you \textbf{assign} it

\begin{Shaded}
\begin{Highlighting}[]
\FunctionTok{class}\NormalTok{(df\_event)}
\CommentTok{\#\textgreater{} [1] "tbl\_df"     "tbl"        "data.frame"}
\end{Highlighting}
\end{Shaded}

\begin{Shaded}
\begin{Highlighting}[]
\NormalTok{df\_event }\SpecialCharTok{\%\textgreater{}\%} \FunctionTok{group\_by}\NormalTok{(univ\_id, event\_type, event\_state)}
\NormalTok{df\_event\_grp }\OtherTok{\textless{}{-}}\NormalTok{ df\_event }\SpecialCharTok{\%\textgreater{}\%} \FunctionTok{group\_by}\NormalTok{(univ\_id, event\_type, event\_state) }\CommentTok{\# using pipes}
\end{Highlighting}
\end{Shaded}

\begin{Shaded}
\begin{Highlighting}[]
\FunctionTok{class}\NormalTok{(df\_event\_grp)}
\CommentTok{\#\textgreater{} [1] "grouped\_df" "tbl\_df"     "tbl"        "data.frame"}
\end{Highlighting}
\end{Shaded}
\end{frame}

\begin{frame}[fragile]{Un-grouping an object}
\protect\hypertarget{un-grouping-an-object}{}
Use \texttt{ungroup(object)} to un-group grouped data

\begin{Shaded}
\begin{Highlighting}[]
\FunctionTok{class}\NormalTok{(df\_event\_grp)}
\CommentTok{\#\textgreater{} [1] "grouped\_df" "tbl\_df"     "tbl"        "data.frame"}
\NormalTok{df\_event\_grp }\OtherTok{\textless{}{-}} \FunctionTok{ungroup}\NormalTok{(df\_event\_grp)}
\FunctionTok{class}\NormalTok{(df\_event\_grp)}
\CommentTok{\#\textgreater{} [1] "tbl\_df"     "tbl"        "data.frame"}
\FunctionTok{rm}\NormalTok{(df\_event\_grp)}
\end{Highlighting}
\end{Shaded}
\end{frame}

\begin{frame}{\texttt{group\_by()} student exercise}
\protect\hypertarget{group_by-student-exercise}{}
\begin{enumerate}
\tightlist
\item
  Group by ``instnm'' and get a frequency count.

  \begin{itemize}
  \tightlist
  \item
    How many rows and columns do you have? What do the number of rows
    mean?\\
  \end{itemize}
\item
  Now group by ``instnm'' \textbf{and} ``event\_type'' and get a
  frequency count.

  \begin{itemize}
  \tightlist
  \item
    How many rows and columns do you have? What do the number of rows
    mean?
  \end{itemize}
\item
  \textbf{Bonus:} In the same code chunk, group by ``instnm'' and
  ``event\_type'', but this time filter for observations where
  ``med\_inc'' is greater than 75000 and get a frequency count.
\end{enumerate}
\end{frame}

\begin{frame}[fragile]{\texttt{group\_by()} student exercise solutions}
\protect\hypertarget{group_by-student-exercise-solutions}{}
\begin{enumerate}
\tightlist
\item
  Group by ``instnm'' and get a frequency count.

  \begin{itemize}
  \tightlist
  \item
    How many rows and columns do you have? What do the number of rows
    mean?
  \end{itemize}
\end{enumerate}

\begin{Shaded}
\begin{Highlighting}[]
\NormalTok{df\_event }\SpecialCharTok{\%\textgreater{}\%}
  \FunctionTok{group\_by}\NormalTok{(instnm) }\SpecialCharTok{\%\textgreater{}\%}
  \FunctionTok{count}\NormalTok{() }
\CommentTok{\#\textgreater{} \# A tibble: 16 x 2}
\CommentTok{\#\textgreater{} \# Groups:   instnm [16]}
\CommentTok{\#\textgreater{}    instnm          n}
\CommentTok{\#\textgreater{}    \textless{}chr\textgreater{}       \textless{}int\textgreater{}}
\CommentTok{\#\textgreater{}  1 Arkansas      994}
\CommentTok{\#\textgreater{}  2 Bama         4258}
\CommentTok{\#\textgreater{}  3 CU Boulder   1439}
\CommentTok{\#\textgreater{}  4 Cinci         679}
\CommentTok{\#\textgreater{}  5 Kansas       1014}
\CommentTok{\#\textgreater{}  6 NC State      640}
\CommentTok{\#\textgreater{}  7 Pitt         1225}
\CommentTok{\#\textgreater{}  8 Rutgers      1135}
\CommentTok{\#\textgreater{}  9 S Illinois    549}
\CommentTok{\#\textgreater{} 10 Stony Brook   730}
\CommentTok{\#\textgreater{} 11 UC Berkeley   879}
\CommentTok{\#\textgreater{} 12 UC Irvine     539}
\CommentTok{\#\textgreater{} 13 UGA           827}
\CommentTok{\#\textgreater{} 14 UM Amherst    908}
\CommentTok{\#\textgreater{} 15 UNL          1397}
\CommentTok{\#\textgreater{} 16 USCC         1467}
\end{Highlighting}
\end{Shaded}
\end{frame}

\begin{frame}[fragile]{\texttt{group\_by()} student exercise solutions}
\protect\hypertarget{group_by-student-exercise-solutions-1}{}
\begin{enumerate}
\setcounter{enumi}{1}
\tightlist
\item
  Now group by ``instnm'' \textbf{and} ``event\_type'' and get a
  frequency count.

  \begin{itemize}
  \tightlist
  \item
    How many rows and columns do you have? What do the number of rows
    mean?
  \end{itemize}
\end{enumerate}

\begin{Shaded}
\begin{Highlighting}[]
\NormalTok{df\_event }\SpecialCharTok{\%\textgreater{}\%}
  \FunctionTok{group\_by}\NormalTok{(instnm, event\_type) }\SpecialCharTok{\%\textgreater{}\%}
  \FunctionTok{count}\NormalTok{() }
\CommentTok{\#\textgreater{} \# A tibble: 80 x 3}
\CommentTok{\#\textgreater{} \# Groups:   instnm, event\_type [80]}
\CommentTok{\#\textgreater{}    instnm   event\_type      n}
\CommentTok{\#\textgreater{}    \textless{}chr\textgreater{}    \textless{}chr\textgreater{}       \textless{}int\textgreater{}}
\CommentTok{\#\textgreater{}  1 Arkansas 2yr college    32}
\CommentTok{\#\textgreater{}  2 Arkansas 4yr college    14}
\CommentTok{\#\textgreater{}  3 Arkansas other         112}
\CommentTok{\#\textgreater{}  4 Arkansas private hs    222}
\CommentTok{\#\textgreater{}  5 Arkansas public hs     614}
\CommentTok{\#\textgreater{}  6 Bama     2yr college   127}
\CommentTok{\#\textgreater{}  7 Bama     4yr college   158}
\CommentTok{\#\textgreater{}  8 Bama     other         608}
\CommentTok{\#\textgreater{}  9 Bama     private hs    963}
\CommentTok{\#\textgreater{} 10 Bama     public hs    2402}
\CommentTok{\#\textgreater{} \# i 70 more rows}
\end{Highlighting}
\end{Shaded}
\end{frame}

\begin{frame}[fragile]{\texttt{group\_by()} student exercise solutions}
\protect\hypertarget{group_by-student-exercise-solutions-2}{}
\begin{enumerate}
\setcounter{enumi}{2}
\tightlist
\item
  \textbf{Bonus:} Group by ``instnm'' and ``event\_type'', but this time
  filter for observations where ``med\_inc'' is greater than 75000 and
  get a frequency count.
\end{enumerate}

\begin{Shaded}
\begin{Highlighting}[]
\NormalTok{df\_event }\SpecialCharTok{\%\textgreater{}\%}
  \FunctionTok{group\_by}\NormalTok{(instnm, event\_type) }\SpecialCharTok{\%\textgreater{}\%}
  \FunctionTok{filter}\NormalTok{(med\_inc }\SpecialCharTok{\textgreater{}} \DecValTok{75000}\NormalTok{) }\SpecialCharTok{\%\textgreater{}\%}
  \FunctionTok{count}\NormalTok{()}
\CommentTok{\#\textgreater{} \# A tibble: 80 x 3}
\CommentTok{\#\textgreater{} \# Groups:   instnm, event\_type [80]}
\CommentTok{\#\textgreater{}    instnm   event\_type      n}
\CommentTok{\#\textgreater{}    \textless{}chr\textgreater{}    \textless{}chr\textgreater{}       \textless{}int\textgreater{}}
\CommentTok{\#\textgreater{}  1 Arkansas 2yr college     7}
\CommentTok{\#\textgreater{}  2 Arkansas 4yr college     3}
\CommentTok{\#\textgreater{}  3 Arkansas other          30}
\CommentTok{\#\textgreater{}  4 Arkansas private hs     99}
\CommentTok{\#\textgreater{}  5 Arkansas public hs     303}
\CommentTok{\#\textgreater{}  6 Bama     2yr college    21}
\CommentTok{\#\textgreater{}  7 Bama     4yr college    42}
\CommentTok{\#\textgreater{}  8 Bama     other         249}
\CommentTok{\#\textgreater{}  9 Bama     private hs    477}
\CommentTok{\#\textgreater{} 10 Bama     public hs    1478}
\CommentTok{\#\textgreater{} \# i 70 more rows}
\end{Highlighting}
\end{Shaded}
\end{frame}

\hypertarget{summarize}{%
\subsection{summarize()}\label{summarize}}

\begin{frame}[fragile]{\texttt{summarize()} function}
\protect\hypertarget{summarize-function}{}
\textbf{Description}: \texttt{summarize()} calculates across rows; then
collapses into single row

\begin{itemize}
\tightlist
\item
  \texttt{summarize()} create scalar vars summarizing variables of
  existing data frame
\item
  if you first group data frame using \texttt{group\_by()},
  \texttt{summarize()} creates summary vars separately for each group,
  returning object with one row per group
\item
  if data frame not grouped, \texttt{summarize()} will result in one
  row.
\end{itemize}

\textbf{Syntax}: \texttt{summarize(.data,\ ...)}

\begin{itemize}
\tightlist
\item
  \texttt{.data}: a data frame; omit if using \texttt{summarize()} after
  pipe \texttt{\%\textgreater{}\%}
\item
  \texttt{...}: Name-value pairs of summary functions separated by
  commas

  \begin{itemize}
  \tightlist
  \item
    ``name'' will be the name of new variable you will create
  \item
    ``value'' should be expression that returns a single value like
    \texttt{min(x)}, \texttt{n()}
  \item
    variable names do not need to be placed within quotes
  \end{itemize}
\end{itemize}

\textbf{Value} (what \texttt{summarize()} returns/creates)

\begin{itemize}
\tightlist
\item
  Object of same class as \texttt{.data.}; object will have one obs per
  ``by group''
\end{itemize}

\textbf{Useful functions (i.e., ``helper functions'')}

\begin{itemize}
\tightlist
\item
  Standalone functions called \emph{within} \texttt{summarize()}, e.g.,
  \texttt{mean()}, \texttt{n()}
\item
  e.g., count function \texttt{n()} takes no arguments; returns number
  of rows in group
\end{itemize}
\end{frame}

\begin{frame}[fragile]{Investigate objects created by
\texttt{summarize()}}
\protect\hypertarget{investigate-objects-created-by-summarize}{}
\textbf{Example}: Count total number of events

\begin{itemize}
\tightlist
\item
  function \texttt{n()} from \texttt{dplyr} package (\texttt{dplyr::n})
  returns the size of the current group
\end{itemize}

\begin{Shaded}
\begin{Highlighting}[]
\CommentTok{\#?n \# dplyr function n() gives current group size}
\FunctionTok{summarize}\NormalTok{(df\_event, }\AttributeTok{num\_events=}\FunctionTok{n}\NormalTok{()) }\CommentTok{\# without pipes}

\NormalTok{df\_event }\SpecialCharTok{\%\textgreater{}\%} \FunctionTok{summarize}\NormalTok{(}\AttributeTok{num\_events=}\FunctionTok{n}\NormalTok{()) }\CommentTok{\# with pipes}
\NormalTok{df\_event }\SpecialCharTok{\%\textgreater{}\%} \FunctionTok{summarize}\NormalTok{(}\AttributeTok{num\_events=}\FunctionTok{n}\NormalTok{()) }\SpecialCharTok{\%\textgreater{}\%} \FunctionTok{str}\NormalTok{() }\CommentTok{\# use str to see what pipe returned}
  \CommentTok{\#df\_event \%\textgreater{}\% summarize(num\_events=n()) \%\textgreater{}\% View()}
\end{Highlighting}
\end{Shaded}

\textbf{Example}: What is max value of \texttt{med\_inc} across all
events

\begin{itemize}
\tightlist
\item
  function \texttt{max()} from \texttt{base} package
  (\texttt{base::max}) returns max value
\end{itemize}

\begin{Shaded}
\begin{Highlighting}[]
\CommentTok{\#?max \# base R function max() returns max value}
\NormalTok{df\_event }\SpecialCharTok{\%\textgreater{}\%} \FunctionTok{summarize}\NormalTok{(}\AttributeTok{max\_inc=}\FunctionTok{max}\NormalTok{(med\_inc, }\AttributeTok{na.rm =} \ConstantTok{TRUE}\NormalTok{))}
\NormalTok{df\_event }\SpecialCharTok{\%\textgreater{}\%} \FunctionTok{summarize}\NormalTok{(}\AttributeTok{max\_inc=}\FunctionTok{max}\NormalTok{(med\_inc, }\AttributeTok{na.rm =} \ConstantTok{TRUE}\NormalTok{)) }\SpecialCharTok{\%\textgreater{}\%} \FunctionTok{str}\NormalTok{()}
\end{Highlighting}
\end{Shaded}
\end{frame}

\begin{frame}[fragile]{Investigate objects created by
\texttt{summarize()}}
\protect\hypertarget{investigate-objects-created-by-summarize-1}{}
\textbf{Example}: Count total number of events AND max value of median
income

\begin{Shaded}
\begin{Highlighting}[]
\NormalTok{df\_event }\SpecialCharTok{\%\textgreater{}\%} \FunctionTok{summarize}\NormalTok{(}
    \AttributeTok{num\_events=}\FunctionTok{n}\NormalTok{(), }
    \AttributeTok{max\_inc=}\FunctionTok{max}\NormalTok{(med\_inc, }\AttributeTok{na.rm =} \ConstantTok{TRUE}\NormalTok{)}
\NormalTok{  )}
\CommentTok{\#\textgreater{} \# A tibble: 1 x 2}
\CommentTok{\#\textgreater{}   num\_events max\_inc}
\CommentTok{\#\textgreater{}        \textless{}int\textgreater{}   \textless{}dbl\textgreater{}}
\CommentTok{\#\textgreater{} 1      18680  250001}

\CommentTok{\# show object returned by pipe}
\NormalTok{df\_event }\SpecialCharTok{\%\textgreater{}\%} \FunctionTok{summarize}\NormalTok{(}
    \AttributeTok{num\_events=}\FunctionTok{n}\NormalTok{(),}
    \AttributeTok{max\_inc=}\FunctionTok{max}\NormalTok{(med\_inc, }\AttributeTok{na.rm =} \ConstantTok{TRUE}\NormalTok{)}
\NormalTok{  ) }\SpecialCharTok{\%\textgreater{}\%} \FunctionTok{str}\NormalTok{()}
\CommentTok{\#\textgreater{} tibble [1 x 2] (S3: tbl\_df/tbl/data.frame)}
\CommentTok{\#\textgreater{}  $ num\_events: int 18680}
\CommentTok{\#\textgreater{}  $ max\_inc   : num 250001}

\CommentTok{\#df\_event \%\textgreater{}\% summarize(num\_events=n(), max\_inc=max(med\_inc, na.rm = TRUE)) \%\textgreater{}\% View()}
\end{Highlighting}
\end{Shaded}
\end{frame}

\begin{frame}[fragile]{Investigate objects created by
\texttt{summarize()}}
\protect\hypertarget{investigate-objects-created-by-summarize-2}{}
\textbf{Example}: Count total number of events AND max value of median
income

We can use assignment to keep the object created by \texttt{summarize()}

\begin{Shaded}
\begin{Highlighting}[]
\NormalTok{df\_event\_temp }\OtherTok{\textless{}{-}}\NormalTok{ df\_event }\SpecialCharTok{\%\textgreater{}\%} 
  \FunctionTok{summarize}\NormalTok{(}\AttributeTok{num\_events=}\FunctionTok{n}\NormalTok{(), }\AttributeTok{max\_inc=}\FunctionTok{max}\NormalTok{(med\_inc, }\AttributeTok{na.rm =} \ConstantTok{TRUE}\NormalTok{))}

\NormalTok{df\_event\_temp}
\FunctionTok{rm}\NormalTok{(df\_event\_temp)}
\end{Highlighting}
\end{Shaded}

What if we forgot \texttt{na.rm\ =\ TRUE}?

\begin{itemize}
\tightlist
\item
  then \texttt{max\_inc} equals \texttt{NA} cuz can't perform
  calculation on \texttt{NA} values
\end{itemize}

\begin{Shaded}
\begin{Highlighting}[]
\NormalTok{df\_event }\SpecialCharTok{\%\textgreater{}\%} 
  \FunctionTok{summarize}\NormalTok{(}\AttributeTok{num\_events=}\FunctionTok{n}\NormalTok{(),}\AttributeTok{max\_inc=}\FunctionTok{max}\NormalTok{(med\_inc, }\AttributeTok{na.rm =} \ConstantTok{FALSE}\NormalTok{))}
\end{Highlighting}
\end{Shaded}
\end{frame}

\begin{frame}[fragile]{\textbf{Takeaways}, \texttt{summarize()}}
\protect\hypertarget{takeaways-summarize}{}
\begin{itemize}
\tightlist
\item
  By default, objects created by \texttt{summarize()} are data frames
  that contain variables created within \texttt{summarize()} and one
  observation {[}per ``by group''{]}
\item
  most ``helper'' functions (e.g., \texttt{max()}, \texttt{mean()} have
  option \texttt{na.rm} to keep/remove missing obs before performing
  calculations)

  \begin{itemize}
  \tightlist
  \item
    \texttt{na.rm\ =\ FALSE} (default); don't remove \texttt{NAs} prior
    to calculation

    \begin{itemize}
    \tightlist
    \item
      if any obs missing, then result of calculation is \texttt{NA}
    \end{itemize}
  \item
    \texttt{na.rm\ =\ TRUE} (default); remove \texttt{NAs} prior to
    calculation
  \end{itemize}
\item
  Object created by \texttt{summarize()} not retained unless you
  \textbf{assign} it (output omitted)
\end{itemize}

\begin{Shaded}
\begin{Highlighting}[]
\NormalTok{event\_temp }\OtherTok{\textless{}{-}}\NormalTok{ df\_event }\SpecialCharTok{\%\textgreater{}\%} \FunctionTok{summarize}\NormalTok{(}\AttributeTok{num\_events=}\FunctionTok{n}\NormalTok{(), }
  \AttributeTok{mean\_inc=}\FunctionTok{mean}\NormalTok{(med\_inc, }\AttributeTok{na.rm =} \ConstantTok{TRUE}\NormalTok{))}

\NormalTok{event\_temp}
\FunctionTok{rm}\NormalTok{(event\_temp)}
\end{Highlighting}
\end{Shaded}
\end{frame}

\begin{frame}[fragile]{Using {[}{]} operator to filter observations
within summarize}
\protect\hypertarget{using-operator-to-filter-observations-within-summarize}{}
Imagine we want to calculate avg. income, separately for in-state
vs.~out-of-state visits

\begin{itemize}
\tightlist
\item
  first, make sure we can identify in vs.~out-state (output omitted)
\end{itemize}

\begin{Shaded}
\begin{Highlighting}[]
\NormalTok{df\_event }\SpecialCharTok{\%\textgreater{}\%} \FunctionTok{filter}\NormalTok{(event\_state }\SpecialCharTok{==}\NormalTok{ instst) }\SpecialCharTok{\%\textgreater{}\%} \FunctionTok{count}\NormalTok{() }\CommentTok{\#in state}
\CommentTok{\#\textgreater{} \# A tibble: 1 x 1}
\CommentTok{\#\textgreater{}       n}
\CommentTok{\#\textgreater{}   \textless{}int\textgreater{}}
\CommentTok{\#\textgreater{} 1  5425}

\NormalTok{df\_event }\SpecialCharTok{\%\textgreater{}\%} \FunctionTok{filter}\NormalTok{(event\_state }\SpecialCharTok{!=}\NormalTok{ instst) }\SpecialCharTok{\%\textgreater{}\%} \FunctionTok{count}\NormalTok{() }\CommentTok{\#out state}
\CommentTok{\#\textgreater{} \# A tibble: 1 x 1}
\CommentTok{\#\textgreater{}       n}
\CommentTok{\#\textgreater{}   \textless{}int\textgreater{}}
\CommentTok{\#\textgreater{} 1 13255}
\end{Highlighting}
\end{Shaded}
\end{frame}

\begin{frame}[fragile]{Using {[}{]} operator to filter observations
within summarize}
\protect\hypertarget{using-operator-to-filter-observations-within-summarize-1}{}
Task: calculate mean income for: all events; in-state events;
out-of-state events

\begin{Shaded}
\begin{Highlighting}[]
\NormalTok{df\_event }\SpecialCharTok{\%\textgreater{}\%}
  \FunctionTok{summarize}\NormalTok{(}
    \AttributeTok{avg\_inc =} \FunctionTok{mean}\NormalTok{(med\_inc, }\AttributeTok{na.rm =} \ConstantTok{TRUE}\NormalTok{), }\CommentTok{\# all events}
    \AttributeTok{avg\_inc\_inst =} \FunctionTok{mean}\NormalTok{(med\_inc[event\_state }\SpecialCharTok{==}\NormalTok{ instst], }\AttributeTok{na.rm =} \ConstantTok{TRUE}\NormalTok{), }\CommentTok{\# in{-}state}
    \AttributeTok{avg\_inc\_outst =} \FunctionTok{mean}\NormalTok{(med\_inc[event\_state }\SpecialCharTok{!=}\NormalTok{ instst], }\AttributeTok{na.rm =} \ConstantTok{TRUE}\NormalTok{) }\CommentTok{\# out{-}state}
\NormalTok{  )}
\CommentTok{\#\textgreater{} \# A tibble: 1 x 3}
\CommentTok{\#\textgreater{}   avg\_inc avg\_inc\_inst avg\_inc\_outst}
\CommentTok{\#\textgreater{}     \textless{}dbl\textgreater{}        \textless{}dbl\textgreater{}         \textless{}dbl\textgreater{}}
\CommentTok{\#\textgreater{} 1  89089.       71589.        96162.}
\end{Highlighting}
\end{Shaded}

Understanding code:
\texttt{mean(med\_inc{[}event\_state\ ==\ instst{]},\ na.rm\ =\ TRUE)}

\begin{itemize}
\tightlist
\item
  \texttt{mean()} function takes atomic vector as first argument
\item
  the variable \texttt{med\_inc} is an numeric atomic vector
\item
  \texttt{event\_state\ ==\ instst} is condition that yields a logical
  atomic vector
\item
  From Base R lecture, \texttt{med\_inc{[}event\_state\ ==\ instst{]}}
  is:

  \begin{itemize}
  \tightlist
  \item
    ``subset atomic vectors using \texttt{{[}{]}}'' approach 3: ``Using
    logicals to return elements where corresponding logical is
    \texttt{TRUE}''
  \item
    isolates observations of \texttt{med\_inc} where
    \texttt{event\_state} is same as state the university is located in
    (\texttt{instst})
  \end{itemize}
\end{itemize}
\end{frame}

\begin{frame}[fragile]{Using \texttt{summarize()} to create descriptive
statistics table}
\protect\hypertarget{using-summarize-to-create-descriptive-statistics-table}{}
Often helpful to use \texttt{summarize()} to calculate summary
statistics that are the basis for a table of descriptive statistics

\textbf{Task}: create a table of descriptive statistics about variable
\texttt{med\_inc}

\begin{itemize}
\tightlist
\item
  want these measures: number of non-missing obs; mean; standard
  deviation
\end{itemize}

\begin{Shaded}
\begin{Highlighting}[]
\NormalTok{df\_event }\SpecialCharTok{\%\textgreater{}\%} \FunctionTok{mutate}\NormalTok{(}\AttributeTok{non\_miss\_inc =} \FunctionTok{is.na}\NormalTok{(med\_inc)}\SpecialCharTok{==}\DecValTok{0}\NormalTok{) }\SpecialCharTok{\%\textgreater{}\%}
  \FunctionTok{summarize}\NormalTok{(}
    \AttributeTok{n =} \FunctionTok{sum}\NormalTok{(non\_miss\_inc, }\AttributeTok{na.rm =} \ConstantTok{TRUE}\NormalTok{), }\CommentTok{\#SAMPLE SIZE all}
    \AttributeTok{avg\_inc =} \FunctionTok{mean}\NormalTok{(med\_inc, }\AttributeTok{na.rm =} \ConstantTok{TRUE}\NormalTok{), }\CommentTok{\# MEAN}
    \AttributeTok{std\_inc =} \FunctionTok{sd}\NormalTok{(med\_inc, }\AttributeTok{na.rm =} \ConstantTok{TRUE}\NormalTok{) }\CommentTok{\# STANDARD DEVIATION all events}
\NormalTok{  )}
\end{Highlighting}
\end{Shaded}

\textbf{Task}: same as above but separate measures for: all events;
in-state; out-of-state

\begin{Shaded}
\begin{Highlighting}[]
\NormalTok{df\_event }\SpecialCharTok{\%\textgreater{}\%} \FunctionTok{mutate}\NormalTok{(}\AttributeTok{non\_miss\_inc =} \FunctionTok{is.na}\NormalTok{(med\_inc)}\SpecialCharTok{==}\DecValTok{0}\NormalTok{) }\SpecialCharTok{\%\textgreater{}\%}
  \FunctionTok{summarize}\NormalTok{(}
    \AttributeTok{n =} \FunctionTok{sum}\NormalTok{(non\_miss\_inc, }\AttributeTok{na.rm =} \ConstantTok{TRUE}\NormalTok{), }\CommentTok{\#SAMPLE SIZE }
    \AttributeTok{n\_inst =} \FunctionTok{sum}\NormalTok{(non\_miss\_inc[event\_state }\SpecialCharTok{==}\NormalTok{ instst], }\AttributeTok{na.rm =} \ConstantTok{TRUE}\NormalTok{), }
    \AttributeTok{n\_outst =} \FunctionTok{sum}\NormalTok{(non\_miss\_inc[event\_state }\SpecialCharTok{!=}\NormalTok{ instst], }\AttributeTok{na.rm =} \ConstantTok{TRUE}\NormalTok{), }
    \AttributeTok{avg\_inc =} \FunctionTok{mean}\NormalTok{(med\_inc, }\AttributeTok{na.rm =} \ConstantTok{TRUE}\NormalTok{), }\CommentTok{\# MEAN}
    \AttributeTok{avg\_inc\_inst =} \FunctionTok{mean}\NormalTok{(med\_inc[event\_state }\SpecialCharTok{==}\NormalTok{ instst], }\AttributeTok{na.rm =} \ConstantTok{TRUE}\NormalTok{), }
    \AttributeTok{avg\_inc\_outst =} \FunctionTok{mean}\NormalTok{(med\_inc[event\_state }\SpecialCharTok{!=}\NormalTok{ instst], }\AttributeTok{na.rm =} \ConstantTok{TRUE}\NormalTok{),}
    \AttributeTok{std\_inc =} \FunctionTok{sd}\NormalTok{(med\_inc, }\AttributeTok{na.rm =} \ConstantTok{TRUE}\NormalTok{), }\CommentTok{\# STANDARD DEVIATION }
    \AttributeTok{std\_inc\_inst =} \FunctionTok{sd}\NormalTok{(med\_inc[event\_state }\SpecialCharTok{==}\NormalTok{ instst], }\AttributeTok{na.rm =} \ConstantTok{TRUE}\NormalTok{), }
    \AttributeTok{std\_inc\_outst =} \FunctionTok{sd}\NormalTok{(med\_inc[event\_state }\SpecialCharTok{!=}\NormalTok{ instst], }\AttributeTok{na.rm =} \ConstantTok{TRUE}\NormalTok{) }
\NormalTok{  )}
\end{Highlighting}
\end{Shaded}
\end{frame}

\begin{frame}[fragile]{\texttt{summarize()} student exercise}
\protect\hypertarget{summarize-student-exercise}{}
\begin{enumerate}
\tightlist
\item
  What is the min value of \texttt{med\_inc} across all events?

  \begin{itemize}
  \tightlist
  \item
    Hint: Use min()
  \end{itemize}
\item
  What is the mean value of \texttt{fr\_lunch} across all events?

  \begin{itemize}
  \tightlist
  \item
    Hint: Use mean()
  \end{itemize}
\end{enumerate}
\end{frame}

\begin{frame}[fragile]{\texttt{summarize()} student exercise}
\protect\hypertarget{summarize-student-exercise-1}{}
\begin{enumerate}
\tightlist
\item
  What is min value of \texttt{med\_inc} across all events?
\end{enumerate}

\begin{Shaded}
\begin{Highlighting}[]
\NormalTok{df\_event }\SpecialCharTok{\%\textgreater{}\%}
  \FunctionTok{summarize}\NormalTok{(}\AttributeTok{min\_med\_income =} \FunctionTok{min}\NormalTok{(med\_inc, }\AttributeTok{na.rm =} \ConstantTok{TRUE}\NormalTok{))}
\CommentTok{\#\textgreater{} \# A tibble: 1 x 1}
\CommentTok{\#\textgreater{}   min\_med\_income}
\CommentTok{\#\textgreater{}            \textless{}dbl\textgreater{}}
\CommentTok{\#\textgreater{} 1         12894.}
\end{Highlighting}
\end{Shaded}
\end{frame}

\begin{frame}[fragile]{\texttt{summarize()} student exercise}
\protect\hypertarget{summarize-student-exercise-2}{}
\begin{enumerate}
\setcounter{enumi}{1}
\tightlist
\item
  What is the mean value of \texttt{fr\_lunch} across all events?

  \begin{itemize}
  \tightlist
  \item
    Hint: Use mean()
  \end{itemize}
\end{enumerate}

\begin{Shaded}
\begin{Highlighting}[]
\NormalTok{df\_event }\SpecialCharTok{\%\textgreater{}\%}
  \FunctionTok{summarize}\NormalTok{(}\AttributeTok{mean\_fr\_lunch =} \FunctionTok{mean}\NormalTok{(fr\_lunch, }\AttributeTok{na.rm =} \ConstantTok{TRUE}\NormalTok{))}
\CommentTok{\#\textgreater{} \# A tibble: 1 x 1}
\CommentTok{\#\textgreater{}   mean\_fr\_lunch}
\CommentTok{\#\textgreater{}           \textless{}dbl\textgreater{}}
\CommentTok{\#\textgreater{} 1          475.}
\end{Highlighting}
\end{Shaded}
\end{frame}

\hypertarget{combining-group_by-and-summarize}{%
\section{Combining group\_by() and
summarize()}\label{combining-group_by-and-summarize}}

\begin{frame}[fragile]{Combining \texttt{summarize()} and
\texttt{group\_by}}
\protect\hypertarget{combining-summarize-and-group_by}{}
\texttt{summarize()} on ungrouped vs.~grouped data:

\begin{itemize}
\tightlist
\item
  By itself, \texttt{summarize()} performs calculations across all rows
  of data frame then collapses the data frame to a single row
\item
  When data frame is grouped, \texttt{summarize()} performs calculations
  across rows within a group and then collapses to a single row for each
  group
\end{itemize}

\textbf{Example}: Count the number of events for each university

\begin{itemize}
\tightlist
\item
  remember: \texttt{df\_event} has one observation per university,
  recruiting event
\end{itemize}

\begin{Shaded}
\begin{Highlighting}[]
\NormalTok{df\_event }\SpecialCharTok{\%\textgreater{}\%} \FunctionTok{summarize}\NormalTok{(}\AttributeTok{num\_events=}\FunctionTok{n}\NormalTok{())}
\NormalTok{df\_event }\SpecialCharTok{\%\textgreater{}\%} \FunctionTok{group\_by}\NormalTok{(instnm) }\SpecialCharTok{\%\textgreater{}\%} \FunctionTok{summarize}\NormalTok{(}\AttributeTok{num\_events=}\FunctionTok{n}\NormalTok{())}
\end{Highlighting}
\end{Shaded}

\begin{itemize}
\tightlist
\item
  Investigate the object created above
\end{itemize}

\begin{Shaded}
\begin{Highlighting}[]
\NormalTok{df\_event }\SpecialCharTok{\%\textgreater{}\%} \FunctionTok{group\_by}\NormalTok{(instnm) }\SpecialCharTok{\%\textgreater{}\%} \FunctionTok{summarize}\NormalTok{(}\AttributeTok{num\_events=}\FunctionTok{n}\NormalTok{()) }\SpecialCharTok{\%\textgreater{}\%} \FunctionTok{str}\NormalTok{()}
  \CommentTok{\#df\_event \%\textgreater{}\% group\_by(instnm) \%\textgreater{}\% summarize(num\_events=n()) \%\textgreater{}\% View()}
\end{Highlighting}
\end{Shaded}

\begin{itemize}
\tightlist
\item
  Or we could retain object for later use
\end{itemize}

\begin{Shaded}
\begin{Highlighting}[]
\NormalTok{event\_by\_univ }\OtherTok{\textless{}{-}}\NormalTok{ df\_event }\SpecialCharTok{\%\textgreater{}\%} \FunctionTok{group\_by}\NormalTok{(instnm) }\SpecialCharTok{\%\textgreater{}\%} \FunctionTok{summarize}\NormalTok{(}\AttributeTok{num\_events=}\FunctionTok{n}\NormalTok{())}
\FunctionTok{str}\NormalTok{(event\_by\_univ)}
\NormalTok{event\_by\_univ }\CommentTok{\# print}
\FunctionTok{rm}\NormalTok{(event\_by\_univ)}
\end{Highlighting}
\end{Shaded}
\end{frame}

\begin{frame}[fragile]{Combining \texttt{summarize()} and
\texttt{group\_by}}
\protect\hypertarget{combining-summarize-and-group_by-1}{}
\textbf{Task}

\begin{itemize}
\tightlist
\item
  Count number of recruiting events by institution and event\_type

  \begin{itemize}
  \tightlist
  \item
    within \texttt{summarize()}, we will take advantage of the helper
    function \texttt{n()}
  \item
    \texttt{n()} ``gives the current group size''; i.e., how many
    observations in current group
  \end{itemize}
\end{itemize}

\begin{Shaded}
\begin{Highlighting}[]
\CommentTok{\#?n}
\NormalTok{df\_event }\SpecialCharTok{\%\textgreater{}\%} \FunctionTok{group\_by}\NormalTok{(instnm, event\_type) }\SpecialCharTok{\%\textgreater{}\%} \FunctionTok{summarize}\NormalTok{(}\AttributeTok{num\_events=}\FunctionTok{n}\NormalTok{())}

\CommentTok{\#investigate object created}
\NormalTok{df\_event }\SpecialCharTok{\%\textgreater{}\%} \FunctionTok{group\_by}\NormalTok{(instnm, event\_type) }\SpecialCharTok{\%\textgreater{}\%} \FunctionTok{summarize}\NormalTok{(}\AttributeTok{num\_events=}\FunctionTok{n}\NormalTok{()) }\SpecialCharTok{\%\textgreater{}\%} \FunctionTok{glimpse}\NormalTok{()}
  \CommentTok{\#df\_event \%\textgreater{}\% group\_by(instnm, event\_type) \%\textgreater{}\% summarize(num\_events=n()) \%\textgreater{}\% View()}
\end{Highlighting}
\end{Shaded}
\end{frame}

\begin{frame}[fragile]{Combining \texttt{summarize()} and
\texttt{group\_by}}
\protect\hypertarget{combining-summarize-and-group_by-2}{}
\textbf{Task}

\begin{itemize}
\tightlist
\item
  Count number of recruiting events by institution and event\_type
\end{itemize}

Note that data frame object created by \texttt{group\_by()} and
\texttt{summarize()} can be input to graph

\begin{Shaded}
\begin{Highlighting}[]
\CommentTok{\#bar chart of number of events, all universities combined}
\NormalTok{df\_event }\SpecialCharTok{\%\textgreater{}\%} \FunctionTok{group\_by}\NormalTok{(instnm, event\_type) }\SpecialCharTok{\%\textgreater{}\%} 
  \FunctionTok{summarize}\NormalTok{(}\AttributeTok{num\_events=}\FunctionTok{n}\NormalTok{()) }\SpecialCharTok{\%\textgreater{}\%}
    \FunctionTok{ggplot}\NormalTok{(}\FunctionTok{aes}\NormalTok{(}\AttributeTok{x=}\NormalTok{event\_type, }\AttributeTok{y=}\NormalTok{num\_events)) }\SpecialCharTok{+}  \CommentTok{\# plot}
    \FunctionTok{ylab}\NormalTok{(}\StringTok{"Number of events"}\NormalTok{) }\SpecialCharTok{+} \FunctionTok{xlab}\NormalTok{(}\StringTok{"Event type"}\NormalTok{) }\SpecialCharTok{+}\FunctionTok{geom\_col}\NormalTok{()}

\CommentTok{\#bar chart of number of events, separete chart for each university}
\NormalTok{df\_event }\SpecialCharTok{\%\textgreater{}\%} \FunctionTok{group\_by}\NormalTok{(instnm, event\_type) }\SpecialCharTok{\%\textgreater{}\%} 
  \FunctionTok{summarize}\NormalTok{(}\AttributeTok{num\_events=}\FunctionTok{n}\NormalTok{()) }\SpecialCharTok{\%\textgreater{}\%}
    \FunctionTok{ggplot}\NormalTok{(}\FunctionTok{aes}\NormalTok{(}\AttributeTok{x=}\NormalTok{event\_type, }\AttributeTok{y=}\NormalTok{num\_events)) }\SpecialCharTok{+}  \CommentTok{\# plot}
    \FunctionTok{ylab}\NormalTok{(}\StringTok{"Number of events"}\NormalTok{) }\SpecialCharTok{+} \FunctionTok{xlab}\NormalTok{(}\StringTok{"Event type"}\NormalTok{) }\SpecialCharTok{+} \FunctionTok{geom\_col}\NormalTok{() }\SpecialCharTok{+}
    \FunctionTok{coord\_flip}\NormalTok{() }\SpecialCharTok{+} \FunctionTok{facet\_wrap}\NormalTok{(}\SpecialCharTok{\textasciitilde{}}\NormalTok{ instnm) }
\end{Highlighting}
\end{Shaded}
\end{frame}

\begin{frame}[fragile]{Combining \texttt{summarize()} and
\texttt{group\_by}}
\protect\hypertarget{combining-summarize-and-group_by-3}{}
\textbf{Task}. Count number of recruiting events by institution,
event\_type, and whether event is in- or out-of-state
(var=\texttt{event\_inst})

\begin{itemize}
\tightlist
\item
  Note: in \texttt{group\_by()}, the optional \texttt{drop} argument
  controls whether empty groups dropped. default is
  \texttt{drop\ =\ TRUE}
\end{itemize}

\begin{Shaded}
\begin{Highlighting}[]

\NormalTok{df\_event }\SpecialCharTok{\%\textgreater{}\%} \FunctionTok{group\_by}\NormalTok{(instnm, event\_type, event\_inst) }\SpecialCharTok{\%\textgreater{}\%} 
  \FunctionTok{summarize}\NormalTok{(}\AttributeTok{num\_events=}\FunctionTok{n}\NormalTok{())}

\NormalTok{df\_event }\SpecialCharTok{\%\textgreater{}\%} \FunctionTok{group\_by}\NormalTok{(instnm, event\_type, event\_inst, }\AttributeTok{.drop =} \ConstantTok{TRUE}\NormalTok{) }\SpecialCharTok{\%\textgreater{}\%} 
  \FunctionTok{summarize}\NormalTok{(}\AttributeTok{num\_events=}\FunctionTok{n}\NormalTok{()) }

\NormalTok{df\_event }\SpecialCharTok{\%\textgreater{}\%} 
  \FunctionTok{group\_by}\NormalTok{(}\FunctionTok{as.factor}\NormalTok{(instnm), }\FunctionTok{as.factor}\NormalTok{(event\_type), }\FunctionTok{as.factor}\NormalTok{(event\_inst),}
           \AttributeTok{.drop =} \ConstantTok{FALSE}\NormalTok{) }\SpecialCharTok{\%\textgreater{}\%} \FunctionTok{summarize}\NormalTok{(}\AttributeTok{num\_events=}\FunctionTok{n}\NormalTok{()) }\SpecialCharTok{\%\textgreater{}\%} \FunctionTok{arrange}\NormalTok{(num\_events)}
\CommentTok{\# .drop=FALSE affects only grouping columns that are coded as factors}
\CommentTok{\# combinations that include non{-}factor grouping variables are still }
\CommentTok{\# silently dropped even with .drop=FALSE.}
\end{Highlighting}
\end{Shaded}
\end{frame}

\begin{frame}[fragile]{Combining \texttt{summarize()} and
\texttt{group\_by}}
\protect\hypertarget{combining-summarize-and-group_by-4}{}
Make a graph, showing in/out state as fill color of bar

\begin{Shaded}
\begin{Highlighting}[]
\NormalTok{df\_event }\SpecialCharTok{\%\textgreater{}\%} \FunctionTok{group\_by}\NormalTok{(instnm, event\_type, event\_inst) }\SpecialCharTok{\%\textgreater{}\%} 
  \FunctionTok{summarize}\NormalTok{(}\AttributeTok{num\_events=}\FunctionTok{n}\NormalTok{()) }\SpecialCharTok{\%\textgreater{}\%}
    \FunctionTok{ggplot}\NormalTok{(}\FunctionTok{aes}\NormalTok{(}\AttributeTok{x=}\NormalTok{event\_type, }\AttributeTok{y=}\NormalTok{num\_events, }\AttributeTok{fill =}\NormalTok{ event\_inst)) }\SpecialCharTok{+}  \CommentTok{\# plot}
    \FunctionTok{ylab}\NormalTok{(}\StringTok{"Number of events"}\NormalTok{) }\SpecialCharTok{+} \FunctionTok{xlab}\NormalTok{(}\StringTok{"Event type"}\NormalTok{) }\SpecialCharTok{+} \FunctionTok{geom\_col}\NormalTok{() }\SpecialCharTok{+}
    \FunctionTok{coord\_flip}\NormalTok{() }\SpecialCharTok{+} \FunctionTok{facet\_wrap}\NormalTok{(}\SpecialCharTok{\textasciitilde{}}\NormalTok{ instnm) }
\end{Highlighting}
\end{Shaded}
\end{frame}

\begin{frame}[fragile]{Combining \texttt{summarize()} and
\texttt{group\_by}}
\protect\hypertarget{combining-summarize-and-group_by-5}{}
\textbf{Task}

\begin{itemize}
\tightlist
\item
  By university, event type, event\_inst count the number of events and
  calculate the avg. pct white in the zip-code
\end{itemize}

\begin{Shaded}
\begin{Highlighting}[]
\NormalTok{df\_event }\SpecialCharTok{\%\textgreater{}\%} \FunctionTok{group\_by}\NormalTok{(instnm, event\_type, event\_inst) }\SpecialCharTok{\%\textgreater{}\%} 
  \FunctionTok{summarize}\NormalTok{(}\AttributeTok{num\_events=}\FunctionTok{n}\NormalTok{(),}
    \AttributeTok{mean\_pct\_white=}\FunctionTok{mean}\NormalTok{(pct\_white\_zip, }\AttributeTok{na.rm =} \ConstantTok{TRUE}\NormalTok{)}
\NormalTok{  )}

\CommentTok{\#investigate object you created}
\NormalTok{df\_event }\SpecialCharTok{\%\textgreater{}\%} \FunctionTok{group\_by}\NormalTok{(instnm, event\_type, event\_inst) }\SpecialCharTok{\%\textgreater{}\%} 
  \FunctionTok{summarize}\NormalTok{(}\AttributeTok{num\_events=}\FunctionTok{n}\NormalTok{(),}
    \AttributeTok{mean\_pct\_white=}\FunctionTok{mean}\NormalTok{(pct\_white\_zip, }\AttributeTok{na.rm =} \ConstantTok{FALSE}\NormalTok{)}
\NormalTok{  ) }\SpecialCharTok{\%\textgreater{}\%} \FunctionTok{glimpse}\NormalTok{()}
\end{Highlighting}
\end{Shaded}
\end{frame}

\begin{frame}[fragile]{Combining \texttt{summarize()} and
\texttt{group\_by}}
\protect\hypertarget{combining-summarize-and-group_by-6}{}
Recruiting events by UC Berkeley

\begin{Shaded}
\begin{Highlighting}[]
\NormalTok{df\_event }\SpecialCharTok{\%\textgreater{}\%} \FunctionTok{filter}\NormalTok{(univ\_id }\SpecialCharTok{==} \DecValTok{110635}\NormalTok{) }\SpecialCharTok{\%\textgreater{}\%} 
  \FunctionTok{group\_by}\NormalTok{(event\_type) }\SpecialCharTok{\%\textgreater{}\%} \FunctionTok{summarize}\NormalTok{(}\AttributeTok{num\_events=}\FunctionTok{n}\NormalTok{())}
\end{Highlighting}
\end{Shaded}

Let's create a dataset of recruiting events at UC Berkeley

\begin{Shaded}
\begin{Highlighting}[]
\NormalTok{event\_berk }\OtherTok{\textless{}{-}}\NormalTok{ df\_event }\SpecialCharTok{\%\textgreater{}\%} \FunctionTok{filter}\NormalTok{(univ\_id }\SpecialCharTok{==} \DecValTok{110635}\NormalTok{)}

\NormalTok{event\_berk }\SpecialCharTok{\%\textgreater{}\%} \FunctionTok{count}\NormalTok{(event\_type)}
\end{Highlighting}
\end{Shaded}
\end{frame}

\hypertarget{summarize-and-counts}{%
\subsection{summarize() and Counts}\label{summarize-and-counts}}

\begin{frame}[fragile]{\texttt{summarize()}: Counts}
\protect\hypertarget{summarize-counts}{}
The count function \texttt{n()} takes no arguments and returns the size
of the current group

\begin{Shaded}
\begin{Highlighting}[]
\NormalTok{event\_berk }\SpecialCharTok{\%\textgreater{}\%} \FunctionTok{group\_by}\NormalTok{(event\_type, event\_inst) }\SpecialCharTok{\%\textgreater{}\%} 
  \FunctionTok{summarize}\NormalTok{(}\AttributeTok{num\_events=}\FunctionTok{n}\NormalTok{())}
\end{Highlighting}
\end{Shaded}

Because counts are so important, \texttt{dplyr} package includes
separate \texttt{count()} function that can be called outside
\texttt{summarize()} function

\begin{Shaded}
\begin{Highlighting}[]
\NormalTok{event\_berk }\SpecialCharTok{\%\textgreater{}\%} \FunctionTok{group\_by}\NormalTok{(event\_type, event\_inst) }\SpecialCharTok{\%\textgreater{}\%} \FunctionTok{count}\NormalTok{()}
\end{Highlighting}
\end{Shaded}
\end{frame}

\begin{frame}[fragile]{\texttt{summarize()}: count with logical vectors
and \texttt{sum()}}
\protect\hypertarget{summarize-count-with-logical-vectors-and-sum}{}
Logical vectors have values \texttt{TRUE} and \texttt{FALSE}.

\begin{itemize}
\tightlist
\item
  When used with numeric functions, \texttt{TRUE} converted to 1 and
  \texttt{FALSE} to 0.
\end{itemize}

\texttt{sum()} is a numeric function that returns the sum of values

\begin{Shaded}
\begin{Highlighting}[]
\FunctionTok{sum}\NormalTok{(}\FunctionTok{c}\NormalTok{(}\DecValTok{5}\NormalTok{,}\DecValTok{10}\NormalTok{))}
\FunctionTok{sum}\NormalTok{(}\FunctionTok{c}\NormalTok{(}\ConstantTok{TRUE}\NormalTok{,}\ConstantTok{TRUE}\NormalTok{,}\ConstantTok{FALSE}\NormalTok{,}\ConstantTok{FALSE}\NormalTok{))}
\end{Highlighting}
\end{Shaded}

\texttt{is.na()} returns \texttt{TRUE} if value is \texttt{NA} and
otherwise returns \texttt{FALSE}

\begin{Shaded}
\begin{Highlighting}[]
\FunctionTok{is.na}\NormalTok{(}\FunctionTok{c}\NormalTok{(}\DecValTok{5}\NormalTok{,}\ConstantTok{NA}\NormalTok{,}\DecValTok{4}\NormalTok{,}\ConstantTok{NA}\NormalTok{))}\SpecialCharTok{{-}}
\FunctionTok{sum}\NormalTok{(}\FunctionTok{is.na}\NormalTok{(}\FunctionTok{c}\NormalTok{(}\DecValTok{5}\NormalTok{,}\ConstantTok{NA}\NormalTok{,}\DecValTok{4}\NormalTok{,}\ConstantTok{NA}\NormalTok{,}\DecValTok{5}\NormalTok{)))}
\CommentTok{\#\textgreater{} [1] {-}2 {-}1 {-}2 {-}1}
\FunctionTok{sum}\NormalTok{(}\SpecialCharTok{!}\FunctionTok{is.na}\NormalTok{(}\FunctionTok{c}\NormalTok{(}\DecValTok{5}\NormalTok{,}\ConstantTok{NA}\NormalTok{,}\DecValTok{4}\NormalTok{,}\ConstantTok{NA}\NormalTok{,}\DecValTok{5}\NormalTok{)))}
\CommentTok{\#\textgreater{} [1] 3}
\end{Highlighting}
\end{Shaded}

Application: How many missing/non-missing obs in variable
{[}\textbf{very important}{]}

\begin{Shaded}
\begin{Highlighting}[]
\NormalTok{event\_berk }\SpecialCharTok{\%\textgreater{}\%} \FunctionTok{group\_by}\NormalTok{(event\_type) }\SpecialCharTok{\%\textgreater{}\%} 
  \FunctionTok{summarize}\NormalTok{(}
    \AttributeTok{n\_events =} \FunctionTok{n}\NormalTok{(),}
    \AttributeTok{n\_miss\_inc =} \FunctionTok{sum}\NormalTok{(}\FunctionTok{is.na}\NormalTok{(med\_inc)),}
    \AttributeTok{n\_nonmiss\_inc =} \FunctionTok{sum}\NormalTok{(}\SpecialCharTok{!}\FunctionTok{is.na}\NormalTok{(med\_inc)),}
    \AttributeTok{n\_nonmiss\_fr\_lunch =} \FunctionTok{sum}\NormalTok{(}\SpecialCharTok{!}\FunctionTok{is.na}\NormalTok{(fr\_lunch))}
\NormalTok{  )}
\end{Highlighting}
\end{Shaded}
\end{frame}

\begin{frame}[fragile]{\texttt{summarize()} and count student exercise}
\protect\hypertarget{summarize-and-count-student-exercise}{}
Use one code chunk for this exercise. You could tackle this a step at a
time and run the entire code chunk when you have answered all parts of
this question. Create your own variable names.

\begin{enumerate}
\tightlist
\item
  Using the \texttt{event\_berk} object, filter observations where
  \texttt{event\_state} is VA and group by \texttt{event\_type}.

  \begin{enumerate}
  \tightlist
  \item
    Using the summarize function to create a variable that represents
    the count for each \texttt{event\_type}.\\
  \item
    Create a variable that represents the sum of missing obs for
    \texttt{med\_inc}.\\
  \item
    Create a variable that represents the sum of non-missing obs for
    \texttt{med\_inc}.\\
  \item
    \textbf{Bonus}: Arrange variable you created representing the count
    of each \texttt{event\_type} in descending order.
  \end{enumerate}
\end{enumerate}
\end{frame}

\begin{frame}[fragile]{\texttt{summarize()} and count student exercise
SOLUTION}
\protect\hypertarget{summarize-and-count-student-exercise-solution}{}
\begin{enumerate}
\tightlist
\item
  Using the \texttt{event\_berk} object filter observations where
  \texttt{event\_state} is VA and group by \texttt{event\_type}.

  \begin{enumerate}
  \tightlist
  \item
    Using the summarize function, create a variable that represents the
    count for each \texttt{event\_type}.\\
  \item
    Now get the sum of missing obs for \texttt{med\_inc}.\\
  \item
    Now get the sum of non-missing obs for \texttt{med\_inc}.
  \end{enumerate}
\end{enumerate}

\begin{Shaded}
\begin{Highlighting}[]
\NormalTok{event\_berk }\SpecialCharTok{\%\textgreater{}\%}
  \FunctionTok{filter}\NormalTok{(event\_state }\SpecialCharTok{==} \StringTok{"VA"}\NormalTok{) }\SpecialCharTok{\%\textgreater{}\%}
  \FunctionTok{group\_by}\NormalTok{(event\_type) }\SpecialCharTok{\%\textgreater{}\%}
  \FunctionTok{summarize}\NormalTok{(}
    \AttributeTok{n\_events =} \FunctionTok{n}\NormalTok{(),}
    \AttributeTok{n\_miss\_inc =} \FunctionTok{sum}\NormalTok{(}\FunctionTok{is.na}\NormalTok{(med\_inc)),}
    \AttributeTok{n\_nonmiss\_inc =} \FunctionTok{sum}\NormalTok{(}\SpecialCharTok{!}\FunctionTok{is.na}\NormalTok{(med\_inc))) }\SpecialCharTok{\%\textgreater{}\%}
  \FunctionTok{arrange}\NormalTok{(}\FunctionTok{desc}\NormalTok{(n\_events))}
\CommentTok{\#\textgreater{} \# A tibble: 3 x 4}
\CommentTok{\#\textgreater{}   event\_type n\_events n\_miss\_inc n\_nonmiss\_inc}
\CommentTok{\#\textgreater{}   \textless{}chr\textgreater{}         \textless{}int\textgreater{}      \textless{}int\textgreater{}         \textless{}int\textgreater{}}
\CommentTok{\#\textgreater{} 1 public hs        20          0            20}
\CommentTok{\#\textgreater{} 2 private hs       13          0            13}
\CommentTok{\#\textgreater{} 3 other             3          0             3}
\end{Highlighting}
\end{Shaded}
\end{frame}

\hypertarget{summarize-and-means}{%
\subsection{summarize() and means}\label{summarize-and-means}}

\begin{frame}[fragile]{\texttt{summarize()}: means}
\protect\hypertarget{summarize-means}{}
The \texttt{mean()} function within \texttt{summarize()} calculates
means, separately for each group

\begin{Shaded}
\begin{Highlighting}[]
\NormalTok{event\_berk }\SpecialCharTok{\%\textgreater{}\%} \FunctionTok{group\_by}\NormalTok{(event\_inst, event\_type) }\SpecialCharTok{\%\textgreater{}\%} \FunctionTok{summarize}\NormalTok{(}
  \AttributeTok{n\_events=}\FunctionTok{n}\NormalTok{(),}
  \AttributeTok{mean\_inc=}\FunctionTok{mean}\NormalTok{(med\_inc, }\AttributeTok{na.rm =} \ConstantTok{TRUE}\NormalTok{),}
  \AttributeTok{mean\_pct\_white=}\FunctionTok{mean}\NormalTok{(pct\_white\_zip, }\AttributeTok{na.rm =} \ConstantTok{TRUE}\NormalTok{))}
\CommentTok{\#\textgreater{} \# A tibble: 10 x 5}
\CommentTok{\#\textgreater{} \# Groups:   event\_inst [2]}
\CommentTok{\#\textgreater{}    event\_inst event\_type  n\_events mean\_inc mean\_pct\_white}
\CommentTok{\#\textgreater{}    \textless{}chr\textgreater{}      \textless{}chr\textgreater{}          \textless{}int\textgreater{}    \textless{}dbl\textgreater{}          \textless{}dbl\textgreater{}}
\CommentTok{\#\textgreater{}  1 In{-}State   2yr college      111   78486.           40.1}
\CommentTok{\#\textgreater{}  2 In{-}State   4yr college       14  131691.           58.0}
\CommentTok{\#\textgreater{}  3 In{-}State   other             49   75040.           37.6}
\CommentTok{\#\textgreater{}  4 In{-}State   private hs        35   95229.           48.4}
\CommentTok{\#\textgreater{}  5 In{-}State   public hs        259   87097.           39.6}
\CommentTok{\#\textgreater{}  6 Out{-}State  2yr college        1  153070.           89.7}
\CommentTok{\#\textgreater{}  7 Out{-}State  4yr college        4   76913.           65.8}
\CommentTok{\#\textgreater{}  8 Out{-}State  other             89   69004.           56.5}
\CommentTok{\#\textgreater{}  9 Out{-}State  private hs       134   87654.           64.3}
\CommentTok{\#\textgreater{} 10 Out{-}State  public hs        183  103603.           62.0}
\end{Highlighting}
\end{Shaded}
\end{frame}

\begin{frame}[fragile]{\texttt{summarize()}: means and \texttt{na.rm}
argument}
\protect\hypertarget{summarize-means-and-na.rm-argument}{}
Default behavior of ``aggregation functions'' (e.g.,
\texttt{summarize()})

\begin{itemize}
\tightlist
\item
  if \emph{input} has any missing values (\texttt{NA}), than output will
  be missing.
\end{itemize}

Many functions have argument \texttt{na.rm} (means ``remove
\texttt{NAs}'')

\begin{itemize}
\tightlist
\item
  \texttt{na.rm\ =\ FALSE} {[}the default for \texttt{mean()}{]}

  \begin{itemize}
  \tightlist
  \item
    Do not remove missing values from input before calculating
  \item
    Therefore, missing values in input will cause output to be missing
  \end{itemize}
\item
  \texttt{na.rm\ =\ TRUE}

  \begin{itemize}
  \tightlist
  \item
    Remove missing values from input before calculating
  \item
    Therefore, missing values in input will not cause output to be
    missing
  \end{itemize}
\end{itemize}

\begin{Shaded}
\begin{Highlighting}[]
\CommentTok{\#na.rm = FALSE; the default setting}
\NormalTok{event\_berk }\SpecialCharTok{\%\textgreater{}\%} \FunctionTok{group\_by}\NormalTok{(event\_inst, event\_type) }\SpecialCharTok{\%\textgreater{}\%} \FunctionTok{summarize}\NormalTok{(}
  \AttributeTok{n\_events=}\FunctionTok{n}\NormalTok{(),}
  \AttributeTok{n\_miss\_inc =} \FunctionTok{sum}\NormalTok{(}\FunctionTok{is.na}\NormalTok{(med\_inc)),}
  \AttributeTok{mean\_inc=}\FunctionTok{mean}\NormalTok{(med\_inc, }\AttributeTok{na.rm =} \ConstantTok{FALSE}\NormalTok{),}
  \AttributeTok{n\_miss\_frlunch =} \FunctionTok{sum}\NormalTok{(}\FunctionTok{is.na}\NormalTok{(fr\_lunch)),}
  \AttributeTok{mean\_fr\_lunch=}\FunctionTok{mean}\NormalTok{(fr\_lunch, }\AttributeTok{na.rm =} \ConstantTok{FALSE}\NormalTok{))}
\CommentTok{\# na.rm = TRUE}
\NormalTok{event\_berk }\SpecialCharTok{\%\textgreater{}\%} \FunctionTok{group\_by}\NormalTok{(event\_inst, event\_type) }\SpecialCharTok{\%\textgreater{}\%} \FunctionTok{summarize}\NormalTok{(}
  \AttributeTok{n\_events=}\FunctionTok{n}\NormalTok{(),}
  \AttributeTok{n\_miss\_inc =} \FunctionTok{sum}\NormalTok{(}\FunctionTok{is.na}\NormalTok{(med\_inc)),}
  \AttributeTok{mean\_inc=}\FunctionTok{mean}\NormalTok{(med\_inc, }\AttributeTok{na.rm =} \ConstantTok{TRUE}\NormalTok{),}
  \AttributeTok{n\_miss\_frlunch =} \FunctionTok{sum}\NormalTok{(}\FunctionTok{is.na}\NormalTok{(fr\_lunch)),}
  \AttributeTok{mean\_fr\_lunch=}\FunctionTok{mean}\NormalTok{(fr\_lunch, }\AttributeTok{na.rm =} \ConstantTok{TRUE}\NormalTok{))}
\end{Highlighting}
\end{Shaded}
\end{frame}

\begin{frame}[fragile]{Student exercise}
\protect\hypertarget{student-exercise}{}
\begin{enumerate}
\tightlist
\item
  Using the \texttt{event\_berk} object, group by \texttt{instnm},
  \texttt{event\_inst}, \& \texttt{event\_type}.

  \begin{enumerate}
  \tightlist
  \item
    Create vars for number non\_missing for these racial/ethnic groups
    (\texttt{pct\_white\_zip}, \texttt{pct\_black\_zip},
    \texttt{pct\_asian\_zip}, \texttt{pct\_hispanic\_zip},
    \texttt{pct\_amerindian\_zip}, \texttt{pct\_nativehawaii\_zip})\\
  \item
    Create vars for mean percent for each racial/ethnic group
  \end{enumerate}
\end{enumerate}
\end{frame}

\begin{frame}[fragile]{Student exercise solutions}
\protect\hypertarget{student-exercise-solutions}{}
\begin{Shaded}
\begin{Highlighting}[]
\NormalTok{event\_berk }\SpecialCharTok{\%\textgreater{}\%} \FunctionTok{group\_by}\NormalTok{(instnm, event\_inst, event\_type) }\SpecialCharTok{\%\textgreater{}\%}
  \FunctionTok{summarize}\NormalTok{(}
  \AttributeTok{n\_events=}\FunctionTok{n}\NormalTok{(),}
  \AttributeTok{n\_miss\_white =} \FunctionTok{sum}\NormalTok{(}\SpecialCharTok{!}\FunctionTok{is.na}\NormalTok{(pct\_white\_zip)),}
  \AttributeTok{mean\_white =} \FunctionTok{mean}\NormalTok{(pct\_white\_zip, }\AttributeTok{na.rm =} \ConstantTok{TRUE}\NormalTok{),}
  \AttributeTok{n\_miss\_black =} \FunctionTok{sum}\NormalTok{(}\SpecialCharTok{!}\FunctionTok{is.na}\NormalTok{(pct\_black\_zip)),}
  \AttributeTok{mean\_black =} \FunctionTok{mean}\NormalTok{(pct\_black\_zip, }\AttributeTok{na.rm =} \ConstantTok{TRUE}\NormalTok{),}
  \AttributeTok{n\_miss\_asian =} \FunctionTok{sum}\NormalTok{(}\SpecialCharTok{!}\FunctionTok{is.na}\NormalTok{(pct\_asian\_zip)),}
  \AttributeTok{mean\_asian =} \FunctionTok{mean}\NormalTok{(pct\_asian\_zip, }\AttributeTok{na.rm =} \ConstantTok{TRUE}\NormalTok{),}
  \AttributeTok{n\_miss\_lat =} \FunctionTok{sum}\NormalTok{(}\SpecialCharTok{!}\FunctionTok{is.na}\NormalTok{(pct\_hispanic\_zip)),}
  \AttributeTok{mean\_lat =} \FunctionTok{mean}\NormalTok{(pct\_hispanic\_zip, }\AttributeTok{na.rm =} \ConstantTok{TRUE}\NormalTok{),}
  \AttributeTok{n\_miss\_na =} \FunctionTok{sum}\NormalTok{(}\SpecialCharTok{!}\FunctionTok{is.na}\NormalTok{(pct\_amerindian\_zip)),}
  \AttributeTok{mean\_na =} \FunctionTok{mean}\NormalTok{(pct\_amerindian\_zip, }\AttributeTok{na.rm =} \ConstantTok{TRUE}\NormalTok{),}
  \AttributeTok{n\_miss\_nh =} \FunctionTok{sum}\NormalTok{(}\SpecialCharTok{!}\FunctionTok{is.na}\NormalTok{(pct\_nativehawaii\_zip)),}
  \AttributeTok{mean\_nh =} \FunctionTok{mean}\NormalTok{(pct\_nativehawaii\_zip, }\AttributeTok{na.rm =} \ConstantTok{TRUE}\NormalTok{)) }\SpecialCharTok{\%\textgreater{}\%}
  \FunctionTok{head}\NormalTok{(}\DecValTok{6}\NormalTok{)}
\CommentTok{\#\textgreater{} \# A tibble: 6 x 16}
\CommentTok{\#\textgreater{} \# Groups:   instnm, event\_inst [2]}
\CommentTok{\#\textgreater{}   instnm     event\_inst event\_type n\_events n\_miss\_white mean\_white n\_miss\_black}
\CommentTok{\#\textgreater{}   \textless{}chr\textgreater{}      \textless{}chr\textgreater{}      \textless{}chr\textgreater{}         \textless{}int\textgreater{}        \textless{}int\textgreater{}      \textless{}dbl\textgreater{}        \textless{}int\textgreater{}}
\CommentTok{\#\textgreater{} 1 UC Berkel\textasciitilde{} In{-}State   2yr colle\textasciitilde{}      111          106       40.1          106}
\CommentTok{\#\textgreater{} 2 UC Berkel\textasciitilde{} In{-}State   4yr colle\textasciitilde{}       14           12       58.0           12}
\CommentTok{\#\textgreater{} 3 UC Berkel\textasciitilde{} In{-}State   other            49           48       37.6           48}
\CommentTok{\#\textgreater{} 4 UC Berkel\textasciitilde{} In{-}State   private hs       35           35       48.4           35}
\CommentTok{\#\textgreater{} 5 UC Berkel\textasciitilde{} In{-}State   public hs       259          258       39.6          258}
\CommentTok{\#\textgreater{} 6 UC Berkel\textasciitilde{} Out{-}State  2yr colle\textasciitilde{}        1            1       89.7            1}
\CommentTok{\#\textgreater{} \# i 9 more variables: mean\_black \textless{}dbl\textgreater{}, n\_miss\_asian \textless{}int\textgreater{}, mean\_asian \textless{}dbl\textgreater{},}
\CommentTok{\#\textgreater{} \#   n\_miss\_lat \textless{}int\textgreater{}, mean\_lat \textless{}dbl\textgreater{}, n\_miss\_na \textless{}int\textgreater{}, mean\_na \textless{}dbl\textgreater{},}
\CommentTok{\#\textgreater{} \#   n\_miss\_nh \textless{}int\textgreater{}, mean\_nh \textless{}dbl\textgreater{}}
\end{Highlighting}
\end{Shaded}
\end{frame}

\hypertarget{summarize-and-logical-vectors-part-ii}{%
\subsection{summarize() and logical vectors, part
II}\label{summarize-and-logical-vectors-part-ii}}

\begin{frame}[fragile]{\texttt{summarize()}: counts with logical
vectors, part II}
\protect\hypertarget{summarize-counts-with-logical-vectors-part-ii}{}
Logical vectors (e.g., \texttt{is.na()}) useful for counting obs that
satisfy some condition

\begin{Shaded}
\begin{Highlighting}[]
\FunctionTok{is.na}\NormalTok{(}\FunctionTok{c}\NormalTok{(}\DecValTok{5}\NormalTok{,}\ConstantTok{NA}\NormalTok{,}\DecValTok{4}\NormalTok{,}\ConstantTok{NA}\NormalTok{))}
\CommentTok{\#\textgreater{} [1] FALSE  TRUE FALSE  TRUE}
\FunctionTok{typeof}\NormalTok{(}\FunctionTok{is.na}\NormalTok{(}\FunctionTok{c}\NormalTok{(}\DecValTok{5}\NormalTok{,}\ConstantTok{NA}\NormalTok{,}\DecValTok{4}\NormalTok{,}\ConstantTok{NA}\NormalTok{)))}
\CommentTok{\#\textgreater{} [1] "logical"}
\FunctionTok{sum}\NormalTok{(}\FunctionTok{is.na}\NormalTok{(}\FunctionTok{c}\NormalTok{(}\DecValTok{5}\NormalTok{,}\ConstantTok{NA}\NormalTok{,}\DecValTok{4}\NormalTok{,}\ConstantTok{NA}\NormalTok{)))}
\CommentTok{\#\textgreater{} [1] 2}
\end{Highlighting}
\end{Shaded}

\textbf{Task}: Using object \texttt{event\_berk}, calculate the
following measures for each combination of \texttt{event\_type} and
\texttt{event\_inst}:

\begin{itemize}
\tightlist
\item
  count of number of rows for each group
\item
  count of rows non-missing for both \texttt{pct\_black\_zip} and
  \texttt{pct\_hispanic\_zip}
\item
  count of number of visits to communities where the \texttt{sum} of
  Black and Latinx people comprise more than 50\% of the total
  population
\end{itemize}

\begin{Shaded}
\begin{Highlighting}[]
\NormalTok{event\_berk }\SpecialCharTok{\%\textgreater{}\%} \FunctionTok{group\_by}\NormalTok{ (event\_inst, event\_type) }\SpecialCharTok{\%\textgreater{}\%} \FunctionTok{summarize}\NormalTok{(}
    \AttributeTok{n\_events=}\FunctionTok{n}\NormalTok{(),}
    \AttributeTok{n\_nonmiss\_latbl =} \FunctionTok{sum}\NormalTok{(}\SpecialCharTok{!}\FunctionTok{is.na}\NormalTok{(pct\_black\_zip) }\SpecialCharTok{\&} \SpecialCharTok{!}\FunctionTok{is.na}\NormalTok{(pct\_hispanic\_zip)),}
    \AttributeTok{n\_majority\_latbl=} \FunctionTok{sum}\NormalTok{(pct\_black\_zip}\SpecialCharTok{+}\NormalTok{ pct\_hispanic\_zip}\SpecialCharTok{\textgreater{}}\DecValTok{50}\NormalTok{, }\AttributeTok{na.rm =} \ConstantTok{TRUE}\NormalTok{)}
\NormalTok{  )}
\end{Highlighting}
\end{Shaded}

\texttt{(!is.na(pct\_black\_zip)\ \&\ !is.na(pct\_hispanic\_zip))} is a
logical condition

\begin{itemize}
\tightlist
\item
  condition is \texttt{TRUE} (evaluates to \texttt{1}) or \texttt{FALSE}
  (evaluates to \texttt{0}) for each obs
\end{itemize}
\end{frame}

\begin{frame}[fragile]{\texttt{summarize()}: logical vectors to count
\emph{proportions}}
\protect\hypertarget{summarize-logical-vectors-to-count-proportions}{}
Synatx:
\texttt{group\_by(vars)\ \%\textgreater{}\%\ summarize(prop\ =\ mean(TRUE/FALSE\ conditon))}

\textbf{Task}: separately for in-state/out-of-state, what proportion of
visits to public high schools are to communities with median income
greater than \$100,000?

Steps:

\begin{enumerate}
\tightlist
\item
  Filter public HS visits
\item
  group by in-state vs.~out-of-state
\item
  Create measure
\end{enumerate}

\begin{Shaded}
\begin{Highlighting}[]
\NormalTok{event\_berk }\SpecialCharTok{\%\textgreater{}\%} \FunctionTok{filter}\NormalTok{(event\_type }\SpecialCharTok{==} \StringTok{"public hs"}\NormalTok{) }\SpecialCharTok{\%\textgreater{}\%} \CommentTok{\# filter public hs visits}
  \FunctionTok{group\_by}\NormalTok{ (event\_inst) }\SpecialCharTok{\%\textgreater{}\%} \CommentTok{\# group by in{-}state vs. out{-}of{-}state}
  \FunctionTok{summarize}\NormalTok{(}
    \AttributeTok{n\_events=}\FunctionTok{n}\NormalTok{(), }\CommentTok{\# number of events by group}
    \AttributeTok{n\_nonmiss\_inc =} \FunctionTok{sum}\NormalTok{(}\SpecialCharTok{!}\FunctionTok{is.na}\NormalTok{(med\_inc)), }\CommentTok{\# w/ nonmissings values median inc,}
    \AttributeTok{p\_incgt100k =} \FunctionTok{mean}\NormalTok{(med\_inc}\SpecialCharTok{\textgreater{}}\DecValTok{100000}\NormalTok{, }\AttributeTok{na.rm=}\ConstantTok{TRUE}\NormalTok{)) }\CommentTok{\# proportion visits to $100K+ commmunities }
\CommentTok{\#\textgreater{} \# A tibble: 2 x 4}
\CommentTok{\#\textgreater{}   event\_inst n\_events n\_nonmiss\_inc p\_incgt100k}
\CommentTok{\#\textgreater{}   \textless{}chr\textgreater{}         \textless{}int\textgreater{}         \textless{}int\textgreater{}       \textless{}dbl\textgreater{}}
\CommentTok{\#\textgreater{} 1 In{-}State        259           256       0.273}
\CommentTok{\#\textgreater{} 2 Out{-}State       183           183       0.519}
\end{Highlighting}
\end{Shaded}
\end{frame}

\begin{frame}[fragile]{\texttt{summarize()}: logical vectors to count
\emph{proportions}}
\protect\hypertarget{summarize-logical-vectors-to-count-proportions-1}{}
\textbf{What if we forgot to put \texttt{na.rm=TRUE} in the above task?}

\medskip \textbf{Task}: separately for in-state/out-of-state, what
proportion of visits to public high schools are to communities with
median income greater than \$100,000?

\begin{Shaded}
\begin{Highlighting}[]
\NormalTok{event\_berk }\SpecialCharTok{\%\textgreater{}\%} \FunctionTok{filter}\NormalTok{(event\_type }\SpecialCharTok{==} \StringTok{"public hs"}\NormalTok{) }\SpecialCharTok{\%\textgreater{}\%} \CommentTok{\# filter public hs visits}
  \FunctionTok{group\_by}\NormalTok{ (event\_inst) }\SpecialCharTok{\%\textgreater{}\%} \CommentTok{\# group by in{-}state vs. out{-}of{-}state}
  \FunctionTok{summarize}\NormalTok{(}
    \AttributeTok{n\_events=}\FunctionTok{n}\NormalTok{(), }\CommentTok{\# number of events by group}
    \AttributeTok{n\_nonmiss\_inc =} \FunctionTok{sum}\NormalTok{(}\SpecialCharTok{!}\FunctionTok{is.na}\NormalTok{(med\_inc)), }\CommentTok{\# w/ nonmissings values median inc,}
    \AttributeTok{p\_incgt100k =} \FunctionTok{mean}\NormalTok{(med\_inc}\SpecialCharTok{\textgreater{}}\DecValTok{100000}\NormalTok{, }\AttributeTok{na.rm=}\ConstantTok{FALSE}\NormalTok{)) }\CommentTok{\# proportion visits to $100K+ commmunities }
\CommentTok{\#\textgreater{} \# A tibble: 2 x 4}
\CommentTok{\#\textgreater{}   event\_inst n\_events n\_nonmiss\_inc p\_incgt100k}
\CommentTok{\#\textgreater{}   \textless{}chr\textgreater{}         \textless{}int\textgreater{}         \textless{}int\textgreater{}       \textless{}dbl\textgreater{}}
\CommentTok{\#\textgreater{} 1 In{-}State        259           256      NA    }
\CommentTok{\#\textgreater{} 2 Out{-}State       183           183       0.519}
\end{Highlighting}
\end{Shaded}
\end{frame}

\begin{frame}[fragile]{\texttt{summarize()}: Other ``helper'' functions}
\protect\hypertarget{summarize-other-helper-functions}{}
Lots of other functions we can use within \texttt{summarize()}

\medskip Common functions to use with \texttt{summarize()}:

\begin{longtable}[]{@{}ll@{}}
\toprule\noalign{}
Function & Description \\
\midrule\noalign{}
\endhead
\texttt{n} & count \\
\texttt{n\_distinct} & count unique values \\
\texttt{mean} & mean \\
\texttt{median} & median \\
\texttt{max} & largest value \\
\texttt{min} & smallest value \\
\texttt{sd} & standard deviation \\
\texttt{sum} & sum of values \\
\texttt{first} & first value \\
\texttt{last} & last value \\
\texttt{nth} & nth value \\
\texttt{any} & condition true for at least one value \\
\bottomrule\noalign{}
\end{longtable}

\emph{Note: These functions can also be used on their own or with
\texttt{mutate()}}
\end{frame}

\begin{frame}[fragile]{\texttt{summarize()}: Other functions}
\protect\hypertarget{summarize-other-functions}{}
Maximum value in a group

\begin{Shaded}
\begin{Highlighting}[]
\FunctionTok{max}\NormalTok{(}\FunctionTok{c}\NormalTok{(}\DecValTok{10}\NormalTok{,}\DecValTok{50}\NormalTok{,}\DecValTok{8}\NormalTok{))}
\CommentTok{\#\textgreater{} [1] 50}
\end{Highlighting}
\end{Shaded}

\textbf{Task}: For each combination of in-state/out-of-state and event
type, what is the maximum value of \texttt{med\_inc}?

\begin{Shaded}
\begin{Highlighting}[]
\NormalTok{event\_berk }\SpecialCharTok{\%\textgreater{}\%} \FunctionTok{group\_by}\NormalTok{(event\_type, event\_inst) }\SpecialCharTok{\%\textgreater{}\%} 
  \FunctionTok{summarize}\NormalTok{(}\AttributeTok{max\_inc =} \FunctionTok{max}\NormalTok{(med\_inc)) }\CommentTok{\# oops, we forgot to remove NAs!}

\NormalTok{event\_berk }\SpecialCharTok{\%\textgreater{}\%} \FunctionTok{group\_by}\NormalTok{(event\_type, event\_inst) }\SpecialCharTok{\%\textgreater{}\%} 
  \FunctionTok{summarize}\NormalTok{(}\AttributeTok{max\_inc =} \FunctionTok{max}\NormalTok{(med\_inc, }\AttributeTok{na.rm =} \ConstantTok{TRUE}\NormalTok{))}
\end{Highlighting}
\end{Shaded}
\end{frame}

\begin{frame}[fragile]{\texttt{summarize()}: Other functions}
\protect\hypertarget{summarize-other-functions-1}{}
Isolate first/last/nth observation in a group

\begin{Shaded}
\begin{Highlighting}[]
\NormalTok{x }\OtherTok{\textless{}{-}} \FunctionTok{c}\NormalTok{(}\DecValTok{10}\NormalTok{,}\DecValTok{15}\NormalTok{,}\DecValTok{20}\NormalTok{,}\DecValTok{25}\NormalTok{,}\DecValTok{30}\NormalTok{)}
\FunctionTok{first}\NormalTok{(x)}
\FunctionTok{last}\NormalTok{(x)}
\FunctionTok{nth}\NormalTok{(x,}\DecValTok{1}\NormalTok{)}
\FunctionTok{nth}\NormalTok{(x,}\DecValTok{3}\NormalTok{)}
\FunctionTok{nth}\NormalTok{(x,}\DecValTok{10}\NormalTok{)}
\end{Highlighting}
\end{Shaded}

\textbf{Task}: after sorting object \texttt{event\_berk} by
\texttt{event\_type} and \texttt{event\_datetime\_start}, what is the
value of \texttt{event\_date} for:

\begin{itemize}
\tightlist
\item
  first event for each event type?
\item
  the last event for each event type?
\item
  the 50th event for each event type?
\end{itemize}

\begin{Shaded}
\begin{Highlighting}[]
\NormalTok{event\_berk }\SpecialCharTok{\%\textgreater{}\%} \FunctionTok{arrange}\NormalTok{(event\_type, event\_datetime\_start) }\SpecialCharTok{\%\textgreater{}\%}
  \FunctionTok{group\_by}\NormalTok{(event\_type) }\SpecialCharTok{\%\textgreater{}\%}
  \FunctionTok{summarize}\NormalTok{(}
    \AttributeTok{n\_events =} \FunctionTok{n}\NormalTok{(),}
    \AttributeTok{date\_first=} \FunctionTok{first}\NormalTok{(event\_date),}
    \AttributeTok{date\_last=} \FunctionTok{last}\NormalTok{(event\_date),}
    \AttributeTok{date\_50th=} \FunctionTok{nth}\NormalTok{(event\_date, }\DecValTok{50}\NormalTok{)}
\NormalTok{  )}
\end{Highlighting}
\end{Shaded}
\end{frame}

\begin{frame}[fragile]{Student exercise}
\protect\hypertarget{student-exercise-1}{}
Identify value of \texttt{event\_date} for the \emph{nth} event in each
by group

\textbf{Specific task}:

\begin{itemize}
\tightlist
\item
  arrange (i.e., sort) by \texttt{event\_type} and
  \texttt{event\_datetme\_start}, then group by \texttt{event\_type},
  and then identify the value of \texttt{event\_date} for:

  \begin{itemize}
  \tightlist
  \item
    the first event in each by group (\texttt{event\_type})
  \item
    the second event in each by group
  \item
    the third event in each by group
  \item
    the fourth event in each by group
  \item
    the fifth event in each by group
  \end{itemize}
\end{itemize}
\end{frame}

\begin{frame}[fragile]{Student exercise solution}
\protect\hypertarget{student-exercise-solution}{}
\begin{Shaded}
\begin{Highlighting}[]
\NormalTok{event\_berk }\SpecialCharTok{\%\textgreater{}\%} \FunctionTok{arrange}\NormalTok{(event\_type, event\_datetime\_start) }\SpecialCharTok{\%\textgreater{}\%}
  \FunctionTok{group\_by}\NormalTok{(event\_type) }\SpecialCharTok{\%\textgreater{}\%}
  \FunctionTok{summarize}\NormalTok{(}
    \AttributeTok{n\_events =} \FunctionTok{n}\NormalTok{(),}
    \AttributeTok{date\_1st=} \FunctionTok{first}\NormalTok{(event\_date),}
    \AttributeTok{date\_2nd=} \FunctionTok{nth}\NormalTok{(event\_date,}\DecValTok{2}\NormalTok{),}
    \AttributeTok{date\_3rd=} \FunctionTok{nth}\NormalTok{(event\_date,}\DecValTok{3}\NormalTok{),}
    \AttributeTok{date\_4th=} \FunctionTok{nth}\NormalTok{(event\_date,}\DecValTok{4}\NormalTok{),}
    \AttributeTok{date\_5th=} \FunctionTok{nth}\NormalTok{(event\_date,}\DecValTok{5}\NormalTok{))}
\CommentTok{\#\textgreater{} \# A tibble: 5 x 7}
\CommentTok{\#\textgreater{}   event\_type  n\_events date\_1st   date\_2nd   date\_3rd   date\_4th   date\_5th  }
\CommentTok{\#\textgreater{}   \textless{}chr\textgreater{}          \textless{}int\textgreater{} \textless{}date\textgreater{}     \textless{}date\textgreater{}     \textless{}date\textgreater{}     \textless{}date\textgreater{}     \textless{}date\textgreater{}    }
\CommentTok{\#\textgreater{} 1 2yr college      112 2017{-}04{-}25 2017{-}09{-}05 2017{-}09{-}05 2017{-}09{-}06 2017{-}09{-}06}
\CommentTok{\#\textgreater{} 2 4yr college       18 2017{-}04{-}30 2017{-}05{-}01 2017{-}05{-}06 2017{-}09{-}13 2017{-}09{-}14}
\CommentTok{\#\textgreater{} 3 other            138 2017{-}04{-}11 2017{-}04{-}23 2017{-}04{-}25 2017{-}04{-}29 2017{-}05{-}14}
\CommentTok{\#\textgreater{} 4 private hs       169 2017{-}04{-}23 2017{-}04{-}24 2017{-}04{-}29 2017{-}04{-}30 2017{-}09{-}05}
\CommentTok{\#\textgreater{} 5 public hs        442 2017{-}04{-}14 2017{-}04{-}24 2017{-}04{-}26 2017{-}04{-}27 2017{-}04{-}27}
\end{Highlighting}
\end{Shaded}
\end{frame}

\hypertarget{summarize-across-multiple-columns}{%
\section{Summarize across multiple
columns}\label{summarize-across-multiple-columns}}

\begin{frame}[fragile]{What are column-wise operations?}
\protect\hypertarget{what-are-column-wise-operations}{}
\texttt{across()} allows you to perform the same operation on multiple
columns.

\medskip

\textbf{Description} - \texttt{across()} apply the same transformation
to multiple columns.

\textbf{Syntax} - \texttt{across(.cols,\ .fns,\ .names)}

\textbf{Arguments}

\begin{itemize}
\tightlist
\item
  \texttt{.cols} Columns to transform
\item
  \texttt{.fns} Function to apply to each of the selected columns. Some
  values include:

  \begin{itemize}
  \tightlist
  \item
    A function, e.g.~\texttt{mean()}
  \item
    A purr-style lambda, e.g.~\textasciitilde{} mean(.x, na.rm = TRUE)
  \item
    A named list of functions or lambdas,
    e.g.~\texttt{list(mean\ =\ mean,\ n\_miss\ =\ \textasciitilde{}sum(is.na(.x)))}
  \end{itemize}
\item
  \texttt{.names} A glue specification that describes how to name the
  output columns. Use \texttt{\{.col\}} to stand for the selected column
  name and \texttt{\{.fn\}} to stand for the name of the function being
  applied, e.g.~``\texttt{\{.col\}\_\{.fn\}}''
\end{itemize}
\end{frame}

\begin{frame}[fragile]{across() affects every variable}
\protect\hypertarget{across-affects-every-variable}{}
Syntax: \texttt{across(.cols,\ .fns,\ .names)}

\begin{itemize}
\tightlist
\item
  \texttt{.cols} Columns to transform
\item
  \texttt{.fns} Function to apply to each of the selected columns. Some
  values include:

  \begin{itemize}
  \tightlist
  \item
    A function, e.g.~\texttt{mean()}; A purr-style lambda,
    e.g.~\textasciitilde{} mean(.x, na.rm = TRUE); A named list of
    functions or lambdas,
    e.g.~\texttt{list(mean\ =\ mean,\ n\_miss\ =\ \textasciitilde{}sum(is.na(.x)))}
  \end{itemize}
\end{itemize}

\textbf{Task}:

\begin{itemize}
\tightlist
\item
  For U. Pittsburgh (\texttt{univ\_id\ =\ 215293}) events at public and
  private high schools, caclulate the \textbf{mean} value of
  \texttt{med\_inc} and \texttt{pct\_white\_zip} for each combination of
  \texttt{event\_type} and \texttt{event\_inst}
\end{itemize}

\begin{Shaded}
\begin{Highlighting}[]
\NormalTok{df\_event }\SpecialCharTok{\%\textgreater{}\%} 
  \FunctionTok{filter}\NormalTok{(univ\_id }\SpecialCharTok{==} \DecValTok{215293}\NormalTok{, event\_type }\SpecialCharTok{\%in\%} \FunctionTok{c}\NormalTok{(}\StringTok{"private hs"}\NormalTok{,}\StringTok{"public hs"}\NormalTok{)) }\SpecialCharTok{\%\textgreater{}\%}
  \FunctionTok{select}\NormalTok{(event\_type, event\_inst,med\_inc,pct\_white\_zip) }\SpecialCharTok{\%\textgreater{}\%}
  \FunctionTok{group\_by}\NormalTok{(event\_type,event\_inst) }\SpecialCharTok{\%\textgreater{}\%}
  \FunctionTok{summarize}\NormalTok{(}\FunctionTok{across}\NormalTok{(}\FunctionTok{c}\NormalTok{(med\_inc, pct\_white\_zip), mean))}
\end{Highlighting}
\end{Shaded}

\medskip

Try again, this time applying \texttt{na.rm\ =\ TRUE} - this is an
example of a purr style lambda
\texttt{\textasciitilde{}\ mean(.x,\ na.rm\ =\ TRUE)} argument ``for the
function calls in \texttt{.fns}.''

\begin{Shaded}
\begin{Highlighting}[]
\NormalTok{df\_event }\SpecialCharTok{\%\textgreater{}\%} 
  \FunctionTok{filter}\NormalTok{(univ\_id }\SpecialCharTok{==} \DecValTok{215293}\NormalTok{, event\_type }\SpecialCharTok{\%in\%} \FunctionTok{c}\NormalTok{(}\StringTok{"private hs"}\NormalTok{,}\StringTok{"public hs"}\NormalTok{)) }\SpecialCharTok{\%\textgreater{}\%}
  \FunctionTok{select}\NormalTok{(event\_type, event\_inst,med\_inc,pct\_white\_zip) }\SpecialCharTok{\%\textgreater{}\%}
  \FunctionTok{group\_by}\NormalTok{(event\_type,event\_inst) }\SpecialCharTok{\%\textgreater{}\%}
  \FunctionTok{summarize}\NormalTok{(}\FunctionTok{across}\NormalTok{(}\FunctionTok{c}\NormalTok{(med\_inc, pct\_white\_zip), }\SpecialCharTok{\textasciitilde{}}\FunctionTok{mean}\NormalTok{(.x, }\AttributeTok{na.rm =} \ConstantTok{TRUE}\NormalTok{)))}
\end{Highlighting}
\end{Shaded}
\end{frame}

\begin{frame}[fragile]{across() affects every variable}
\protect\hypertarget{across-affects-every-variable-1}{}
Syntax: \texttt{across(.cols,\ .fns,\ .names)}

\begin{itemize}
\tightlist
\item
  \texttt{.cols} Columns to transform
\item
  \texttt{.fns} Function to apply to each of the selected columns. Some
  values include:

  \begin{itemize}
  \tightlist
  \item
    A function, e.g.~\texttt{mean()}; A purr-style lambda,
    e.g.~\textasciitilde{} mean(.x, na.rm = TRUE); A named list of
    functions or lambdas,
    e.g.~\texttt{list(mean\ =\ mean,\ n\_miss\ =\ \textasciitilde{}sum(is.na(.x)))}
  \end{itemize}
\end{itemize}

\textbf{Task}:

\begin{itemize}
\item
  For U. Pittsburgh (\texttt{univ\_id\ =\ 215293}) events at public and
  private high schools, caclulate \textbf{mean} and \textbf{standard
  deviation} of \texttt{med\_inc} and \texttt{pct\_white\_zip} for each
  combination of \texttt{event\_type} and \texttt{event\_inst}
\item
  You can create a named list of functions to supply into the second
  argument \texttt{.fns}
\end{itemize}

\begin{Shaded}
\begin{Highlighting}[]
\NormalTok{mean\_sd }\OtherTok{\textless{}{-}} \FunctionTok{list}\NormalTok{(}
  \AttributeTok{mean =} \SpecialCharTok{\textasciitilde{}}\FunctionTok{mean}\NormalTok{(.x, }\AttributeTok{na.rm =} \ConstantTok{TRUE}\NormalTok{), }
  \AttributeTok{sd =} \SpecialCharTok{\textasciitilde{}}\FunctionTok{sd}\NormalTok{(.x, }\AttributeTok{na.rm =} \ConstantTok{TRUE}\NormalTok{)}
\NormalTok{)}

\NormalTok{df\_event }\SpecialCharTok{\%\textgreater{}\%} 
  \FunctionTok{filter}\NormalTok{(univ\_id }\SpecialCharTok{==} \DecValTok{215293}\NormalTok{, event\_type }\SpecialCharTok{\%in\%} \FunctionTok{c}\NormalTok{(}\StringTok{"private hs"}\NormalTok{,}\StringTok{"public hs"}\NormalTok{)) }\SpecialCharTok{\%\textgreater{}\%}
  \FunctionTok{select}\NormalTok{(event\_type, event\_inst,med\_inc,pct\_white\_zip) }\SpecialCharTok{\%\textgreater{}\%}
  \FunctionTok{group\_by}\NormalTok{(event\_type,event\_inst) }\SpecialCharTok{\%\textgreater{}\%}
  \FunctionTok{summarize}\NormalTok{(}\FunctionTok{across}\NormalTok{(}\FunctionTok{c}\NormalTok{(med\_inc, pct\_white\_zip), mean\_sd))}
\end{Highlighting}
\end{Shaded}
\end{frame}

\begin{frame}[fragile]{across() affects every variable}
\protect\hypertarget{across-affects-every-variable-2}{}
Syntax: \texttt{across(.cols,\ .fns,\ .names)}

\begin{itemize}
\tightlist
\item
  \texttt{.names} A glue specification that describes how to name the
  output columns. Use \texttt{\{.col\}} to stand for the selected column
  name and \texttt{\{.fn\}} to stand for the name of the function being
  applied, e.g.~``\texttt{\{.col\}\_\{.fn\}}''
\end{itemize}

Use this syntax to control variable name suffixes:

\begin{itemize}
\tightlist
\item
  \texttt{.names\ =\ "\{.col\}\_\{.fn\}"}
\end{itemize}

\begin{Shaded}
\begin{Highlighting}[]
\NormalTok{df\_event }\SpecialCharTok{\%\textgreater{}\%} 
  \FunctionTok{filter}\NormalTok{(univ\_id }\SpecialCharTok{==} \DecValTok{215293}\NormalTok{, event\_type }\SpecialCharTok{\%in\%} \FunctionTok{c}\NormalTok{(}\StringTok{"private hs"}\NormalTok{,}\StringTok{"public hs"}\NormalTok{)) }\SpecialCharTok{\%\textgreater{}\%}
  \FunctionTok{select}\NormalTok{(event\_type, event\_inst,med\_inc,pct\_white\_zip) }\SpecialCharTok{\%\textgreater{}\%}
  \FunctionTok{group\_by}\NormalTok{(event\_type,event\_inst) }\SpecialCharTok{\%\textgreater{}\%}
  \FunctionTok{summarize}\NormalTok{(}\FunctionTok{across}\NormalTok{(}\FunctionTok{c}\NormalTok{(med\_inc, pct\_white\_zip), mean\_sd, }\AttributeTok{.names =} \StringTok{"\{.col\}\_\{.fn\}"}\NormalTok{))}
\end{Highlighting}
\end{Shaded}

\begin{itemize}
\tightlist
\item
  \texttt{.names\ =\ "\{.fn\}\_\{.col\}"} Change the order of the name
  to include function first and then the column name
\end{itemize}

\begin{Shaded}
\begin{Highlighting}[]
\NormalTok{df\_event }\SpecialCharTok{\%\textgreater{}\%} 
  \FunctionTok{filter}\NormalTok{(univ\_id }\SpecialCharTok{==} \DecValTok{215293}\NormalTok{, event\_type }\SpecialCharTok{\%in\%} \FunctionTok{c}\NormalTok{(}\StringTok{"private hs"}\NormalTok{,}\StringTok{"public hs"}\NormalTok{)) }\SpecialCharTok{\%\textgreater{}\%}
  \FunctionTok{select}\NormalTok{(event\_type, event\_inst,med\_inc,pct\_white\_zip) }\SpecialCharTok{\%\textgreater{}\%}
  \FunctionTok{group\_by}\NormalTok{(event\_type,event\_inst) }\SpecialCharTok{\%\textgreater{}\%}
  \FunctionTok{summarize}\NormalTok{(}\FunctionTok{across}\NormalTok{(}\FunctionTok{c}\NormalTok{(med\_inc, pct\_white\_zip), mean\_sd, }\AttributeTok{.names =} \StringTok{"\{.fn\}\_\{.col\}"}\NormalTok{))}
\end{Highlighting}
\end{Shaded}
\end{frame}

\begin{frame}[fragile]{across() affects every variable}
\protect\hypertarget{across-affects-every-variable-3}{}
\textbf{Task}:

\begin{itemize}
\tightlist
\item
  Same task as before, but now calculate \textbf{mean}, \textbf{standard
  deviation}, \textbf{min}, and \textbf{max} of \texttt{med\_inc} and
  \texttt{pct\_white\_zip} for each combination of \texttt{event\_type}
  and \texttt{event\_inst}
\end{itemize}

\medskip

\begin{Shaded}
\begin{Highlighting}[]
\NormalTok{desc\_stat }\OtherTok{\textless{}{-}} \FunctionTok{list}\NormalTok{(}
  \AttributeTok{mean =} \SpecialCharTok{\textasciitilde{}}\FunctionTok{mean}\NormalTok{(.x, }\AttributeTok{na.rm =} \ConstantTok{TRUE}\NormalTok{), }
  \AttributeTok{sd =} \SpecialCharTok{\textasciitilde{}}\FunctionTok{sd}\NormalTok{(.x, }\AttributeTok{na.rm =} \ConstantTok{TRUE}\NormalTok{),}
  \AttributeTok{low =} \SpecialCharTok{\textasciitilde{}}\FunctionTok{min}\NormalTok{(.x, }\AttributeTok{na.rm =} \ConstantTok{TRUE}\NormalTok{),}
  \AttributeTok{high =} \SpecialCharTok{\textasciitilde{}}\FunctionTok{max}\NormalTok{(.x, }\AttributeTok{na.rm=}\ConstantTok{TRUE}\NormalTok{)}
\NormalTok{)}

\NormalTok{df\_event }\SpecialCharTok{\%\textgreater{}\%} 
  \FunctionTok{filter}\NormalTok{(univ\_id }\SpecialCharTok{==} \DecValTok{215293}\NormalTok{, event\_type }\SpecialCharTok{\%in\%} \FunctionTok{c}\NormalTok{(}\StringTok{"private hs"}\NormalTok{,}\StringTok{"public hs"}\NormalTok{)) }\SpecialCharTok{\%\textgreater{}\%}
  \FunctionTok{select}\NormalTok{(event\_type, event\_inst,med\_inc,pct\_white\_zip) }\SpecialCharTok{\%\textgreater{}\%}
  \FunctionTok{group\_by}\NormalTok{(event\_type,event\_inst) }\SpecialCharTok{\%\textgreater{}\%}
  \FunctionTok{summarize}\NormalTok{(}\FunctionTok{across}\NormalTok{(}\FunctionTok{c}\NormalTok{(med\_inc, pct\_white\_zip), desc\_stat, }\AttributeTok{.names =} \StringTok{"\{.col\}\_\{.fn\}"}\NormalTok{))}
\CommentTok{\#\textgreater{} \# A tibble: 4 x 10}
\CommentTok{\#\textgreater{} \# Groups:   event\_type [2]}
\CommentTok{\#\textgreater{}   event\_type event\_inst med\_inc\_mean med\_inc\_sd med\_inc\_low med\_inc\_high}
\CommentTok{\#\textgreater{}   \textless{}chr\textgreater{}      \textless{}chr\textgreater{}             \textless{}dbl\textgreater{}      \textless{}dbl\textgreater{}       \textless{}dbl\textgreater{}        \textless{}dbl\textgreater{}}
\CommentTok{\#\textgreater{} 1 private hs In{-}State         77115.     36559.      12894.      224163 }
\CommentTok{\#\textgreater{} 2 private hs Out{-}State       103915.     44220.      29630.      223556.}
\CommentTok{\#\textgreater{} 3 public hs  In{-}State         78408.     25841.      23168.      169036.}
\CommentTok{\#\textgreater{} 4 public hs  Out{-}State       114212.     39745.      21581       250001 }
\CommentTok{\#\textgreater{} \# i 4 more variables: pct\_white\_zip\_mean \textless{}dbl\textgreater{}, pct\_white\_zip\_sd \textless{}dbl\textgreater{},}
\CommentTok{\#\textgreater{} \#   pct\_white\_zip\_low \textless{}dbl\textgreater{}, pct\_white\_zip\_high \textless{}dbl\textgreater{}}
\end{Highlighting}
\end{Shaded}
\end{frame}

\begin{frame}[fragile]{across(), quosure style lambdas \textasciitilde{}
func\_name(.x)}
\protect\hypertarget{across-quosure-style-lambdas-func_name.x}{}
Syntax: \texttt{across(.cols,\ .fns,\ .names)}

\begin{itemize}
\tightlist
\item
  \texttt{.fns} Function to apply to each of the selected columns. Some
  values include:

  \begin{itemize}
  \tightlist
  \item
    A function, e.g.~\texttt{mean()}; A purr-style lambda,
    e.g.~\textasciitilde{} mean(.x, na.rm = TRUE); A named list of
    functions or lambdas,
    e.g.~\texttt{list(mean\ =\ mean,\ n\_miss\ =\ \textasciitilde{}sum(is.na(.x)))}
  \end{itemize}
\end{itemize}

\textbf{Task}: Calculate mean, number of obs, and number of non-missing
obs for variables

\begin{itemize}
\tightlist
\item
  Can exclude \texttt{n()} from the \texttt{.cols} argument in
  \texttt{across()} function because you only need to calculate once to
  get the number of observations for each group.
\end{itemize}

\medskip

\begin{Shaded}
\begin{Highlighting}[]
\NormalTok{mean\_obs\_nmiss }\OtherTok{\textless{}{-}} \FunctionTok{list}\NormalTok{(}
  \AttributeTok{mean =} \SpecialCharTok{\textasciitilde{}}\FunctionTok{mean}\NormalTok{(.x, }\AttributeTok{na.rm =} \ConstantTok{TRUE}\NormalTok{), }
  \AttributeTok{n\_non\_miss =} \SpecialCharTok{\textasciitilde{}}\FunctionTok{sum}\NormalTok{(}\SpecialCharTok{!}\FunctionTok{is.na}\NormalTok{(.x))}
\NormalTok{)}

\NormalTok{df\_event }\SpecialCharTok{\%\textgreater{}\%} 
  \FunctionTok{filter}\NormalTok{(univ\_id }\SpecialCharTok{==} \DecValTok{215293}\NormalTok{, event\_type }\SpecialCharTok{\%in\%} \FunctionTok{c}\NormalTok{(}\StringTok{"private hs"}\NormalTok{,}\StringTok{"public hs"}\NormalTok{)) }\SpecialCharTok{\%\textgreater{}\%}
  \FunctionTok{select}\NormalTok{(event\_type, event\_inst,med\_inc,pop\_total) }\SpecialCharTok{\%\textgreater{}\%}
  \FunctionTok{group\_by}\NormalTok{(event\_type,event\_inst) }\SpecialCharTok{\%\textgreater{}\%}
  \FunctionTok{summarize}\NormalTok{(}\AttributeTok{nrow =} \FunctionTok{n}\NormalTok{(), }\FunctionTok{across}\NormalTok{(}\FunctionTok{c}\NormalTok{(med\_inc, pop\_total), mean\_obs\_nmiss,}\AttributeTok{.names =} \StringTok{"\{.col\}\_\{.fn\}"}\NormalTok{))}
\end{Highlighting}
\end{Shaded}
\end{frame}

\begin{frame}[fragile]{across() and where() affects selected variables
if they meet a condition}
\protect\hypertarget{across-and-where-affects-selected-variables-if-they-meet-a-condition}{}
Syntax: \texttt{across(where(fn)),\ .fns,\ .names)}

\begin{itemize}
\tightlist
\item
  \texttt{fn} A function that returns TRUE or FALSE
\item
  \texttt{.fns} Function to apply to each of the selected columns. Some
  values include:

  \begin{itemize}
  \tightlist
  \item
    A function, e.g.~\texttt{mean()}; A purr-style lambda,
    e.g.~\textasciitilde{} mean(.x, na.rm = TRUE); A named list of
    functions or lambdas,
    e.g.~\texttt{list(mean\ =\ mean,\ n\_miss\ =\ \textasciitilde{}sum(is.na(.x)))}
  \end{itemize}
\end{itemize}

\medskip

\textbf{Task}: For U. Pittsburgh events at public and private high
schools, caclulate \textbf{mean}, \textbf{min}, and \textbf{max} of
variables \texttt{med\_inc} and \texttt{event\_date} for each
combination of \texttt{event\_type} and \texttt{event\_inst}

\begin{Shaded}
\begin{Highlighting}[]
\NormalTok{mean\_min\_max }\OtherTok{\textless{}{-}} \FunctionTok{list}\NormalTok{(}
  \AttributeTok{mean =} \SpecialCharTok{\textasciitilde{}}\FunctionTok{mean}\NormalTok{(.x, }\AttributeTok{na.rm =} \ConstantTok{TRUE}\NormalTok{),}
  \AttributeTok{low =} \SpecialCharTok{\textasciitilde{}}\FunctionTok{min}\NormalTok{(.x, }\AttributeTok{na.rm=}\ConstantTok{TRUE}\NormalTok{),}
  \AttributeTok{high =} \SpecialCharTok{\textasciitilde{}}\FunctionTok{max}\NormalTok{(.x, }\AttributeTok{na.rm=}\ConstantTok{TRUE}\NormalTok{)}
\NormalTok{)}

\NormalTok{df\_event }\SpecialCharTok{\%\textgreater{}\%} 
  \FunctionTok{filter}\NormalTok{(univ\_id }\SpecialCharTok{==} \DecValTok{215293}\NormalTok{, event\_type }\SpecialCharTok{\%in\%} \FunctionTok{c}\NormalTok{(}\StringTok{"private hs"}\NormalTok{,}\StringTok{"public hs"}\NormalTok{)) }\SpecialCharTok{\%\textgreater{}\%}
  \FunctionTok{select}\NormalTok{(event\_type, event\_inst, med\_inc, event\_date) }\SpecialCharTok{\%\textgreater{}\%}
  \FunctionTok{group\_by}\NormalTok{(event\_type,event\_inst) }\SpecialCharTok{\%\textgreater{}\%}
  \FunctionTok{summarize}\NormalTok{(}\FunctionTok{across}\NormalTok{(}\FunctionTok{where}\NormalTok{(is.double), mean\_min\_max))}
\end{Highlighting}
\end{Shaded}

\medskip

\begin{itemize}
\tightlist
\item
  Although \texttt{event\_date} is a Date class, it is type double and
  therefore we can apply \texttt{is.double} to the \texttt{where()}
  function to calculate the mean, min, and max.
\end{itemize}
\end{frame}

\begin{frame}[fragile]{across() and where() affects selected variables
if they meet a condition}
\protect\hypertarget{across-and-where-affects-selected-variables-if-they-meet-a-condition-1}{}
Useful if you want to apply functions to variables that are particular
\texttt{type} or \texttt{class}

Syntax: \texttt{across(where(fn)),\ .fns,\ .names)}

\begin{itemize}
\tightlist
\item
  \texttt{fn} A function that returns TRUE or FALSE
\item
  \texttt{.fns} Function to apply to each of the selected columns. Some
  values include:

  \begin{itemize}
  \tightlist
  \item
    A function, e.g.~\texttt{mean()}; A purr-style lambda,
    e.g.~\textasciitilde{} mean(.x, na.rm = TRUE); A named list of
    functions or lambdas,
    e.g.~\texttt{list(mean\ =\ mean,\ n\_miss\ =\ \textasciitilde{}sum(is.na(.x)))}
  \end{itemize}
\end{itemize}

\medskip

\textbf{Task}: For events by U. Pittsburgh at public and private high
schools, caclulate mean and standard deviation for \textbf{numeric
variables}

\begin{Shaded}
\begin{Highlighting}[]
\CommentTok{\#First, which vars are numeric}
\NormalTok{df\_event }\SpecialCharTok{\%\textgreater{}\%} 
  \FunctionTok{select}\NormalTok{(event\_type, event\_inst,instnm,school\_id,med\_inc,pct\_white\_zip) }\SpecialCharTok{\%\textgreater{}\%}
  \FunctionTok{glimpse}\NormalTok{()}

\NormalTok{df\_event }\SpecialCharTok{\%\textgreater{}\%} 
  \FunctionTok{filter}\NormalTok{(univ\_id }\SpecialCharTok{==} \DecValTok{215293}\NormalTok{, event\_type }\SpecialCharTok{\%in\%} \FunctionTok{c}\NormalTok{(}\StringTok{"private hs"}\NormalTok{,}\StringTok{"public hs"}\NormalTok{)) }\SpecialCharTok{\%\textgreater{}\%}
  \FunctionTok{select}\NormalTok{(event\_type, event\_inst,instnm,school\_id,med\_inc,pct\_white\_zip) }\SpecialCharTok{\%\textgreater{}\%}
  \FunctionTok{group\_by}\NormalTok{(event\_type,event\_inst) }\SpecialCharTok{\%\textgreater{}\%}
  \FunctionTok{summarize}\NormalTok{(}\FunctionTok{across}\NormalTok{(}\FunctionTok{where}\NormalTok{(is.numeric), mean\_sd))}
\end{Highlighting}
\end{Shaded}
\end{frame}

\hypertarget{attach-aggregate-measures-to-your-data-frame}{%
\section{Attach aggregate measures to your data
frame}\label{attach-aggregate-measures-to-your-data-frame}}

\begin{frame}[fragile]{Attach aggregate measures to your data frame}
\protect\hypertarget{attach-aggregate-measures-to-your-data-frame-1}{}
We can attach aggregate measures to a data frame by using group\_by
without summarize()

What do I mean by ``attaching aggregate measures to a data frame''?

\begin{itemize}
\tightlist
\item
  Calculate measures at the by\_group level, but attach them to original
  object rather than creating an object with one row for each by\_group
\end{itemize}

\textbf{Task}: Using \texttt{event\_berk} data frame, create (1) a
measure of average income across all events and (2) a measure of average
income for each event type

\begin{itemize}
\tightlist
\item
  resulting object should have same number of observations as
  \texttt{event\_berk}
\end{itemize}

Steps:

\begin{enumerate}
\tightlist
\item
  create measure of avg. income across all events without using
  \texttt{group\_by()} or \texttt{summarize()} and assign as (new)
  object
\item
  Using object from previous step, create measure of avg. income across
  by event type using \texttt{group\_by()} without \texttt{summarize()}
  and assign as new object
\end{enumerate}
\end{frame}

\begin{frame}[fragile]{Attach aggregate measures to your data frame}
\protect\hypertarget{attach-aggregate-measures-to-your-data-frame-2}{}
\textbf{Task}: Using \texttt{event\_berk} data frame, create (1) a
measure of average income across all events and (2) a measure of average
income for each event type

\begin{enumerate}
\tightlist
\item
  Create measure of average income across all events
\end{enumerate}

\begin{Shaded}
\begin{Highlighting}[]
\NormalTok{event\_berk\_temp }\OtherTok{\textless{}{-}}\NormalTok{ event\_berk }\SpecialCharTok{\%\textgreater{}\%} 
  \FunctionTok{arrange}\NormalTok{(event\_date) }\SpecialCharTok{\%\textgreater{}\%} \CommentTok{\# sort by event\_date (optional)}
  \FunctionTok{select}\NormalTok{(event\_date, event\_type,med\_inc) }\SpecialCharTok{\%\textgreater{}\%} \CommentTok{\# select vars to be retained (optioanl) }
  \FunctionTok{mutate}\NormalTok{(}\AttributeTok{avg\_inc =} \FunctionTok{mean}\NormalTok{(med\_inc, }\AttributeTok{na.rm=}\ConstantTok{TRUE}\NormalTok{)) }\CommentTok{\# create avg. inc measure}

\FunctionTok{dim}\NormalTok{(event\_berk\_temp)}
\NormalTok{event\_berk\_temp }\SpecialCharTok{\%\textgreater{}\%} \FunctionTok{head}\NormalTok{(}\DecValTok{5}\NormalTok{)}
\end{Highlighting}
\end{Shaded}

\begin{enumerate}
\setcounter{enumi}{1}
\tightlist
\item
  Create measure of average income by event type
\end{enumerate}

\begin{Shaded}
\begin{Highlighting}[]
\NormalTok{event\_berk\_temp }\OtherTok{\textless{}{-}}\NormalTok{ event\_berk\_temp }\SpecialCharTok{\%\textgreater{}\%} 
  \FunctionTok{group\_by}\NormalTok{(event\_type) }\SpecialCharTok{\%\textgreater{}\%} \CommentTok{\# grouping by event type}
  \FunctionTok{mutate}\NormalTok{(}\AttributeTok{avg\_inc\_type =} \FunctionTok{mean}\NormalTok{(med\_inc, }\AttributeTok{na.rm=}\ConstantTok{TRUE}\NormalTok{)) }\CommentTok{\# create avg. inc measure}
  
\FunctionTok{str}\NormalTok{(event\_berk\_temp)}
\NormalTok{event\_berk\_temp }\SpecialCharTok{\%\textgreater{}\%} \FunctionTok{head}\NormalTok{(}\DecValTok{5}\NormalTok{)}
\end{Highlighting}
\end{Shaded}
\end{frame}

\begin{frame}[fragile]{Attach aggregate measures to your data frame}
\protect\hypertarget{attach-aggregate-measures-to-your-data-frame-3}{}
\textbf{Task}: Using \texttt{event\_berk\_temp} from previous question,
create a measure that identifies whether \texttt{med\_inc} associated
with the event is higher/lower than average income for all events of
that type

Steps:

\begin{enumerate}
\tightlist
\item
  Create measure of average income for each event type {[}already
  done{]}
\item
  Create 0/1 indicator that identifies whether median income at event
  location is higher than average median income for events of that type
\end{enumerate}

\begin{Shaded}
\begin{Highlighting}[]
\CommentTok{\# average income at recruiting events across all universities}
\NormalTok{event\_berk\_tempv2 }\OtherTok{\textless{}{-}}\NormalTok{ event\_berk\_temp }\SpecialCharTok{\%\textgreater{}\%} 
  \FunctionTok{mutate}\NormalTok{(}\AttributeTok{gt\_avg\_inc\_type =}\NormalTok{ med\_inc }\SpecialCharTok{\textgreater{}}\NormalTok{ avg\_inc\_type) }\SpecialCharTok{\%\textgreater{}\%} 
  \FunctionTok{select}\NormalTok{(}\SpecialCharTok{{-}}\NormalTok{(avg\_inc)) }\CommentTok{\# drop avg\_inc (optional)}
\NormalTok{event\_berk\_tempv2 }\CommentTok{\# note how med\_ic = NA are treated}
\end{Highlighting}
\end{Shaded}

Same as above, but this time create integer indicator rather than
logical

\begin{Shaded}
\begin{Highlighting}[]
\NormalTok{event\_berk\_tempv2 }\OtherTok{\textless{}{-}}\NormalTok{ event\_berk\_tempv2 }\SpecialCharTok{\%\textgreater{}\%} 
  \FunctionTok{mutate}\NormalTok{(}\AttributeTok{gt\_avg\_inc\_type =} \FunctionTok{as.integer}\NormalTok{(med\_inc }\SpecialCharTok{\textgreater{}}\NormalTok{ avg\_inc\_type)) }
\NormalTok{event\_berk\_tempv2  }\SpecialCharTok{\%\textgreater{}\%} \FunctionTok{head}\NormalTok{(}\DecValTok{4}\NormalTok{)}
\end{Highlighting}
\end{Shaded}
\end{frame}

\begin{frame}[fragile]{Student exercise}
\protect\hypertarget{student-exercise-2}{}
Task: is \texttt{pct\_white\_zip} at a particular event higher or lower
than the average pct\_white\_zip for that \texttt{event\_type}?

\begin{itemize}
\tightlist
\item
  Note: all events attached to a particular zip\_code
\item
  \texttt{pct\_white\_zip}: pct of people in that zip\_code who identify
  as white
\end{itemize}

Steps in task:

\begin{itemize}
\tightlist
\item
  Create measure of average pct white for each event\_type
\item
  Compare whether pct\_white\_zip is higher or lower than this average
\end{itemize}
\end{frame}

\begin{frame}[fragile]{Student exercise solution}
\protect\hypertarget{student-exercise-solution-1}{}
Task: is \texttt{pct\_white\_zip} at a particular event higher or lower
than the average pct\_white\_zip for that \texttt{event\_type}?

\begin{Shaded}
\begin{Highlighting}[]
\NormalTok{event\_berk\_tempv3 }\OtherTok{\textless{}{-}}\NormalTok{ event\_berk }\SpecialCharTok{\%\textgreater{}\%} 
  \FunctionTok{arrange}\NormalTok{(event\_date) }\SpecialCharTok{\%\textgreater{}\%} \CommentTok{\# sort by event\_date (optional)}
  \FunctionTok{select}\NormalTok{(event\_date, event\_type, pct\_white\_zip) }\SpecialCharTok{\%\textgreater{}\%} \CommentTok{\#optional}
  \FunctionTok{group\_by}\NormalTok{(event\_type) }\SpecialCharTok{\%\textgreater{}\%} \CommentTok{\# grouping by event type}
  \FunctionTok{mutate}\NormalTok{(}\AttributeTok{avg\_pct\_white =} \FunctionTok{mean}\NormalTok{(pct\_white\_zip, }\AttributeTok{na.rm=}\ConstantTok{TRUE}\NormalTok{),}
         \AttributeTok{gt\_avg\_pctwhite\_type =} \FunctionTok{as.integer}\NormalTok{(pct\_white\_zip }\SpecialCharTok{\textgreater{}}\NormalTok{ avg\_pct\_white)) }
\NormalTok{event\_berk\_tempv3 }\SpecialCharTok{\%\textgreater{}\%} \FunctionTok{head}\NormalTok{(}\DecValTok{4}\NormalTok{)}
\CommentTok{\#\textgreater{} \# A tibble: 4 x 5}
\CommentTok{\#\textgreater{} \# Groups:   event\_type [3]}
\CommentTok{\#\textgreater{}   event\_date event\_type pct\_white\_zip avg\_pct\_white gt\_avg\_pctwhite\_type}
\CommentTok{\#\textgreater{}   \textless{}date\textgreater{}     \textless{}chr\textgreater{}              \textless{}dbl\textgreater{}         \textless{}dbl\textgreater{}                \textless{}int\textgreater{}}
\CommentTok{\#\textgreater{} 1 2017{-}04{-}11 other               37.2          49.7                    0}
\CommentTok{\#\textgreater{} 2 2017{-}04{-}14 public hs           78.3          48.9                    1}
\CommentTok{\#\textgreater{} 3 2017{-}04{-}23 private hs          84.7          61.0                    1}
\CommentTok{\#\textgreater{} 4 2017{-}04{-}23 other               20.9          49.7                    0}
\end{Highlighting}
\end{Shaded}
\end{frame}

\end{document}
