% Options for packages loaded elsewhere
\PassOptionsToPackage{unicode}{hyperref}
\PassOptionsToPackage{hyphens}{url}
\PassOptionsToPackage{dvipsnames,svgnames*,x11names*}{xcolor}
%
\documentclass[
  8pt,
  ignorenonframetext,
  dvipsnames]{beamer}
\usepackage{pgfpages}
\setbeamertemplate{caption}[numbered]
\setbeamertemplate{caption label separator}{: }
\setbeamercolor{caption name}{fg=normal text.fg}
\beamertemplatenavigationsymbolsempty
% Prevent slide breaks in the middle of a paragraph
\widowpenalties 1 10000
\raggedbottom
\setbeamertemplate{part page}{
  \centering
  \begin{beamercolorbox}[sep=16pt,center]{part title}
    \usebeamerfont{part title}\insertpart\par
  \end{beamercolorbox}
}
\setbeamertemplate{section page}{
  \centering
  \begin{beamercolorbox}[sep=12pt,center]{part title}
    \usebeamerfont{section title}\insertsection\par
  \end{beamercolorbox}
}
\setbeamertemplate{subsection page}{
  \centering
  \begin{beamercolorbox}[sep=8pt,center]{part title}
    \usebeamerfont{subsection title}\insertsubsection\par
  \end{beamercolorbox}
}
\AtBeginPart{
  \frame{\partpage}
}
\AtBeginSection{
  \ifbibliography
  \else
    \frame{\sectionpage}
  \fi
}
\AtBeginSubsection{
  \frame{\subsectionpage}
}
\usepackage{lmodern}
\usepackage{amssymb,amsmath}
\usepackage{ifxetex,ifluatex}
\ifnum 0\ifxetex 1\fi\ifluatex 1\fi=0 % if pdftex
  \usepackage[T1]{fontenc}
  \usepackage[utf8]{inputenc}
  \usepackage{textcomp} % provide euro and other symbols
\else % if luatex or xetex
  \usepackage{unicode-math}
  \defaultfontfeatures{Scale=MatchLowercase}
  \defaultfontfeatures[\rmfamily]{Ligatures=TeX,Scale=1}
\fi
% Use upquote if available, for straight quotes in verbatim environments
\IfFileExists{upquote.sty}{\usepackage{upquote}}{}
\IfFileExists{microtype.sty}{% use microtype if available
  \usepackage[]{microtype}
  \UseMicrotypeSet[protrusion]{basicmath} % disable protrusion for tt fonts
}{}
\makeatletter
\@ifundefined{KOMAClassName}{% if non-KOMA class
  \IfFileExists{parskip.sty}{%
    \usepackage{parskip}
  }{% else
    \setlength{\parindent}{0pt}
    \setlength{\parskip}{6pt plus 2pt minus 1pt}}
}{% if KOMA class
  \KOMAoptions{parskip=half}}
\makeatother
\usepackage{xcolor}
\IfFileExists{xurl.sty}{\usepackage{xurl}}{} % add URL line breaks if available
\IfFileExists{bookmark.sty}{\usepackage{bookmark}}{\usepackage{hyperref}}
\hypersetup{
  pdftitle={Guidelines for Investigating, cleaning, and creating variables},
  colorlinks=true,
  linkcolor=Maroon,
  filecolor=Maroon,
  citecolor=Blue,
  urlcolor=blue,
  pdfcreator={LaTeX via pandoc}}
\urlstyle{same} % disable monospaced font for URLs
\newif\ifbibliography
\usepackage{color}
\usepackage{fancyvrb}
\newcommand{\VerbBar}{|}
\newcommand{\VERB}{\Verb[commandchars=\\\{\}]}
\DefineVerbatimEnvironment{Highlighting}{Verbatim}{commandchars=\\\{\}}
% Add ',fontsize=\small' for more characters per line
\usepackage{framed}
\definecolor{shadecolor}{RGB}{248,248,248}
\newenvironment{Shaded}{\begin{snugshade}}{\end{snugshade}}
\newcommand{\AlertTok}[1]{\textcolor[rgb]{0.94,0.16,0.16}{#1}}
\newcommand{\AnnotationTok}[1]{\textcolor[rgb]{0.56,0.35,0.01}{\textbf{\textit{#1}}}}
\newcommand{\AttributeTok}[1]{\textcolor[rgb]{0.77,0.63,0.00}{#1}}
\newcommand{\BaseNTok}[1]{\textcolor[rgb]{0.00,0.00,0.81}{#1}}
\newcommand{\BuiltInTok}[1]{#1}
\newcommand{\CharTok}[1]{\textcolor[rgb]{0.31,0.60,0.02}{#1}}
\newcommand{\CommentTok}[1]{\textcolor[rgb]{0.56,0.35,0.01}{\textit{#1}}}
\newcommand{\CommentVarTok}[1]{\textcolor[rgb]{0.56,0.35,0.01}{\textbf{\textit{#1}}}}
\newcommand{\ConstantTok}[1]{\textcolor[rgb]{0.00,0.00,0.00}{#1}}
\newcommand{\ControlFlowTok}[1]{\textcolor[rgb]{0.13,0.29,0.53}{\textbf{#1}}}
\newcommand{\DataTypeTok}[1]{\textcolor[rgb]{0.13,0.29,0.53}{#1}}
\newcommand{\DecValTok}[1]{\textcolor[rgb]{0.00,0.00,0.81}{#1}}
\newcommand{\DocumentationTok}[1]{\textcolor[rgb]{0.56,0.35,0.01}{\textbf{\textit{#1}}}}
\newcommand{\ErrorTok}[1]{\textcolor[rgb]{0.64,0.00,0.00}{\textbf{#1}}}
\newcommand{\ExtensionTok}[1]{#1}
\newcommand{\FloatTok}[1]{\textcolor[rgb]{0.00,0.00,0.81}{#1}}
\newcommand{\FunctionTok}[1]{\textcolor[rgb]{0.00,0.00,0.00}{#1}}
\newcommand{\ImportTok}[1]{#1}
\newcommand{\InformationTok}[1]{\textcolor[rgb]{0.56,0.35,0.01}{\textbf{\textit{#1}}}}
\newcommand{\KeywordTok}[1]{\textcolor[rgb]{0.13,0.29,0.53}{\textbf{#1}}}
\newcommand{\NormalTok}[1]{#1}
\newcommand{\OperatorTok}[1]{\textcolor[rgb]{0.81,0.36,0.00}{\textbf{#1}}}
\newcommand{\OtherTok}[1]{\textcolor[rgb]{0.56,0.35,0.01}{#1}}
\newcommand{\PreprocessorTok}[1]{\textcolor[rgb]{0.56,0.35,0.01}{\textit{#1}}}
\newcommand{\RegionMarkerTok}[1]{#1}
\newcommand{\SpecialCharTok}[1]{\textcolor[rgb]{0.00,0.00,0.00}{#1}}
\newcommand{\SpecialStringTok}[1]{\textcolor[rgb]{0.31,0.60,0.02}{#1}}
\newcommand{\StringTok}[1]{\textcolor[rgb]{0.31,0.60,0.02}{#1}}
\newcommand{\VariableTok}[1]{\textcolor[rgb]{0.00,0.00,0.00}{#1}}
\newcommand{\VerbatimStringTok}[1]{\textcolor[rgb]{0.31,0.60,0.02}{#1}}
\newcommand{\WarningTok}[1]{\textcolor[rgb]{0.56,0.35,0.01}{\textbf{\textit{#1}}}}
\setlength{\emergencystretch}{3em} % prevent overfull lines
\providecommand{\tightlist}{%
  \setlength{\itemsep}{0pt}\setlength{\parskip}{0pt}}
\setcounter{secnumdepth}{-\maxdimen} % remove section numbering

%packages
\usepackage{graphicx}
\usepackage{rotating}
\usepackage{hyperref}

\usepackage{tikz} % used for text highlighting, amongst others
\usepackage{comment}

%title slide stuff
%\institute{Department of Education}
%\title{Managing and Manipulating Data Using R}

%
\setbeamertemplate{navigation symbols}{} % get rid of navigation icons:
\setbeamertemplate{footline}[page number]

%\setbeamertemplate{frametitle}{\thesection \hspace{0.2cm} \insertframetitle}
\setbeamertemplate{section in toc}[sections numbered]
%\setbeamertemplate{subsection in toc}[subsections numbered]
\setbeamertemplate{subsection in toc}{%
  \leavevmode\leftskip=3.2em\color{gray}\rlap{\hskip-2em\inserttocsectionnumber.\inserttocsubsectionnumber}\inserttocsubsection\par
}

%define colors
%\definecolor{uva_orange}{RGB}{216,141,42} % UVa orange (Rotunda orange)
\definecolor{mygray}{rgb}{0.95, 0.95, 0.95} % for highlighted text
	% grey is equal parts red, green, blue. higher values >> lighter grey
	%\definecolor{lightgraybo}{rgb}{0.83, 0.83, 0.83}

% new commands

%highlight text with very light grey
\newcommand*{\hlg}[1]{%
	\tikz[baseline=(X.base)] \node[rectangle, fill=mygray] (X) {#1};%
}
%, inner sep=0.3mm
%highlight text with very light grey and use font associated with code
\newcommand*{\hlgc}[1]{\texttt{\hlg{#1}}}

%modifying back ticks to add grey background
\let\OldTexttt\texttt
\renewcommand{\texttt}[1]{\OldTexttt{\hlg{#1}}}


\begin{comment}

% Font
\usepackage[defaultfam,light,tabular,lining]{montserrat}
\usepackage[T1]{fontenc}
\renewcommand*\oldstylenums[1]{{\fontfamily{Montserrat-TOsF}\selectfont #1}}

% Change color of boldface text to darkgray
\renewcommand{\textbf}[1]{{\color{darkgray}\bfseries\fontfamily{Montserrat-TOsF}#1}}

% Bullet points
\setbeamertemplate{itemize item}{\color{BlueViolet}$\circ$}
\setbeamertemplate{itemize subitem}{\color{BrickRed}$\triangleright$}
\setbeamertemplate{itemize subsubitem}{$-$}

% Reduce space before lists
%\addtobeamertemplate{itemize/enumerate body begin}{}{\vspace*{-8pt}}

\let\olditem\item
\renewcommand{\item}{%
  \olditem\vspace{4pt}
}

% decreasing space before and after level-2 bullet block
%\addtobeamertemplate{itemize/enumerate subbody begin}{}{\vspace*{-3pt}}
%\addtobeamertemplate{itemize/enumerate subbody end}{}{\vspace*{-3pt}}

% decreasing space before and after level-3 bullet block
%\addtobeamertemplate{itemize/enumerate subsubbody begin}{}{\vspace*{-2pt}}
%\addtobeamertemplate{itemize/enumerate subsubbody end}{}{\vspace*{-2pt}}

%Section numbering
\setbeamertemplate{section page}{%
    \begingroup
        \begin{beamercolorbox}[sep=10pt,center,rounded=true,shadow=true]{section title}
        \usebeamerfont{section title}\thesection~\insertsection\par
        \end{beamercolorbox}
    \endgroup
}

\setbeamertemplate{subsection page}{%
    \begingroup
        \begin{beamercolorbox}[sep=6pt,center,rounded=true,shadow=true]{subsection title}
        \usebeamerfont{subsection title}\thesection.\thesubsection~\insertsubsection\par
        \end{beamercolorbox}
    \endgroup
}

\end{comment}
\ifluatex
  \usepackage{selnolig}  % disable illegal ligatures
\fi

\title{Guidelines for Investigating, cleaning, and creating variables}
\subtitle{EDUC 263: Introduction to Programming and Data Management
Using R}
\author{}
\date{\vspace{-2.5em}}

\begin{document}
\frame{\titlepage}

\hypertarget{introduction}{%
\section{Introduction}\label{introduction}}

\begin{frame}{What we will do today}
\protect\hypertarget{what-we-will-do-today}{}
\tableofcontents
\end{frame}

\begin{frame}[fragile]{Libraries}
\protect\hypertarget{libraries}{}
``Load'' the package we will use today (output omitted)

\begin{itemize}
\tightlist
\item
  \textbf{you must run this code chunk after installing these packages}
\end{itemize}

\begin{Shaded}
\begin{Highlighting}[]
\KeywordTok{library}\NormalTok{(tidyverse)}
\KeywordTok{library}\NormalTok{(haven)}
\KeywordTok{library}\NormalTok{(labelled)}
\end{Highlighting}
\end{Shaded}

\textbf{If package not yet installed}, then must install before you
load. Install in ``console'' rather than .Rmd file

\begin{itemize}
\tightlist
\item
  Generic syntax: \texttt{install.packages("package\_name")}
\item
  Install ``tidyverse'': \texttt{install.packages("tidyverse")}
\end{itemize}

Note: when we load package, name of package is not in quotes; but when
we install package, name of package is in quotes:

\begin{itemize}
\tightlist
\item
  \texttt{install.packages("tidyverse")}
\item
  \texttt{library(tidyverse)}
\end{itemize}
\end{frame}

\begin{frame}[fragile]{Data}
\protect\hypertarget{data}{}
Use \texttt{read\_dta()} function from \texttt{haven} to import Stata
dataset into R

\begin{Shaded}
\begin{Highlighting}[]
\NormalTok{hsls \textless{}{-}}\StringTok{ }\KeywordTok{read\_dta}\NormalTok{(}\DataTypeTok{file=}\StringTok{"https://github.com/ozanj/rclass/raw/master/data/hsls/hsls\_stu\_small.dta"}\NormalTok{)}
\end{Highlighting}
\end{Shaded}

Let's examine the data {[}you \textbf{must} run this code chunk{]}

\begin{Shaded}
\begin{Highlighting}[]
\NormalTok{hsls }\OperatorTok{\%\textgreater{}\%}\StringTok{ }\KeywordTok{names}\NormalTok{()}
\NormalTok{hsls }\OperatorTok{\%\textgreater{}\%}\StringTok{ }\KeywordTok{names}\NormalTok{() }\OperatorTok{\%\textgreater{}\%}\StringTok{ }\KeywordTok{str}\NormalTok{()}
\NormalTok{hsls }\OperatorTok{\%\textgreater{}\%}\StringTok{ }\KeywordTok{names}\NormalTok{() }\OperatorTok{\%\textgreater{}\%}\StringTok{ }\KeywordTok{tolower}\NormalTok{() }\OperatorTok{\%\textgreater{}\%}\StringTok{ }\KeywordTok{str}\NormalTok{()}

\KeywordTok{names}\NormalTok{(hsls) \textless{}{-}}\StringTok{ }\KeywordTok{tolower}\NormalTok{(}\KeywordTok{names}\NormalTok{(hsls)) }\CommentTok{\# convert names to lowercase}
\KeywordTok{names}\NormalTok{(hsls)}

\KeywordTok{str}\NormalTok{(hsls) }\CommentTok{\# ugh}

\KeywordTok{str}\NormalTok{(hsls}\OperatorTok{$}\NormalTok{s3classes)}
\KeywordTok{attributes}\NormalTok{(hsls}\OperatorTok{$}\NormalTok{s3classes)}
\KeywordTok{typeof}\NormalTok{(hsls}\OperatorTok{$}\NormalTok{s3classes)}
\KeywordTok{class}\NormalTok{(hsls}\OperatorTok{$}\NormalTok{s3classes)}
\end{Highlighting}
\end{Shaded}

Download the HSLS Codebook:
\url{https://nces.ed.gov/pubs2014/2014361_AppendixI.pdf}
\end{frame}

\hypertarget{exploratory-data-analysis-eda}{%
\section{Exploratory data analysis
(EDA)}\label{exploratory-data-analysis-eda}}

\begin{frame}{What is exploratory data analysis (EDA)?}
\protect\hypertarget{what-is-exploratory-data-analysis-eda}{}
The
\href{https://towardsdatascience.com/exploratory-data-analysis-8fc1cb20fd15}{Towards
Data Science} website has a nice definition of EDA:

\begin{quote}
``Exploratory Data Analysis refers to the critical process of performing
initial investigations on data so as to discover patterns,to spot
anomalies,to test hypothesis and to check assumptions with the help of
summary statistics''
\end{quote}

\textbf{This course focuses on ``data management'':}

\begin{itemize}
\tightlist
\item
  investigating and cleaning data for the purpose of creating analysis
  variables
\item
  Basically, everything that happens \textbf{before} you conduct
  analyses
\end{itemize}

\textbf{I think about ``exploratory data analysis for data quality''}

\begin{itemize}
\tightlist
\item
  Investigating values and patterns of variables from ``input data''
\item
  Identifying and cleaning errors or values that need to be changed
\item
  Creating analysis variables
\item
  Checking values of analysis variables agains values of input variables
\end{itemize}
\end{frame}

\begin{frame}{How we will teach exploratory data analysis}
\protect\hypertarget{how-we-will-teach-exploratory-data-analysis}{}
Will teach exploratory data analysis (EDA) in two sub-sections:

\begin{enumerate}
\tightlist
\item
  Introduce ``Tools of EDA'':

  \begin{itemize}
  \tightlist
  \item
    Demonstrate code to investigate variables and relatioship between
    variables
  \item
    Most of these tools are just the application of programming skills
    you have already learned
  \end{itemize}
\item
  Provide ``Guidelines for EDA''

  \begin{itemize}
  \tightlist
  \item
    Less about coding, more about practices you should follow and
    mentality necessary to ensure high data quality
  \end{itemize}
\end{enumerate}
\end{frame}

\hypertarget{tools-for-eda}{%
\subsection{Tools for EDA}\label{tools-for-eda}}

\begin{frame}[fragile]{Tools of EDA}
\protect\hypertarget{tools-of-eda}{}
\textbf{To do EDA for data quality, must master the following tools:}

\begin{itemize}
\tightlist
\item
  \medskip \textbf{Select, sort, filter, and print} in order to see data
  patterns, anomolies

  \begin{itemize}
  \tightlist
  \item
    Select and sort particular values of particular variables
  \item
    Print particular values of particular variables
  \end{itemize}
\item
  \textbf{One-way descriptive analyses} (i.e,. focus on one variable)

  \begin{itemize}
  \tightlist
  \item
    Descriptive analyses for continuous variables
  \item
    Descriptive analyses for discreet/categorical variables
  \end{itemize}
\item
  \textbf{Two-way descriptive analyses} (relationship between two
  variables)

  \begin{itemize}
  \tightlist
  \item
    Categorical by categorical
  \item
    Categorical by continuous
  \item
    Continuous by continuous
  \end{itemize}
\end{itemize}

Whenever using any of these tools, \textbf{pay close attention to
missing values and how they are coded}

\begin{itemize}
\tightlist
\item
  Often, the ``input'' variables don't code missing values as
  \texttt{NA}
\item
  Especially when working with survey data, missing values coded as a
  negative number (e.g., \texttt{-9},\texttt{-8},\texttt{-4}) with
  different negative values representing different reasons for data
  being missing
\item
  sometimes missing values coded as very high positive numbers
\item
  Therefore, important to investigate input vars prior to creating
  analysis vars
\end{itemize}
\end{frame}

\begin{frame}[fragile]{Tools of EDA}
\protect\hypertarget{tools-of-eda-1}{}
First, Let's create a smaller version of the HSLS:09 dataset

\begin{Shaded}
\begin{Highlighting}[]
\CommentTok{\#hsls \%\textgreater{}\% var\_label()}
\NormalTok{hsls\_small \textless{}{-}}\StringTok{ }\NormalTok{hsls }\OperatorTok{\%\textgreater{}\%}
\StringTok{  }\KeywordTok{select}\NormalTok{(stu\_id,x3univ1,x3sqstat,x4univ1,x4sqstat,s3classes,}
\NormalTok{         s3work,s3focus,s3clgft,s3workft,s3clgid,s3clgcntrl,}
\NormalTok{         s3clglvl,s3clgsel,s3clgstate,s3proglevel,x4evrappclg,}
\NormalTok{         x4evratndclg,x4atndclg16fb,x4ps1sector,x4ps1level,}
\NormalTok{         x4ps1ctrl,x4ps1select,x4refsector,x4reflevel,x4refctrl,}
\NormalTok{         x4refselect, x2sex,x2race,x2paredu,x2txmtscor,x4x2ses,x4x2sesq5)}
\end{Highlighting}
\end{Shaded}

\begin{Shaded}
\begin{Highlighting}[]
\KeywordTok{names}\NormalTok{(hsls\_small)}
\NormalTok{hsls\_small }\OperatorTok{\%\textgreater{}\%}\StringTok{ }\KeywordTok{var\_label}\NormalTok{()}
\end{Highlighting}
\end{Shaded}
\end{frame}

\begin{frame}[fragile]{Tools of EDA: select, sort, filter, and print}
\protect\hypertarget{tools-of-eda-select-sort-filter-and-print}{}
We've already know \texttt{select()}, \texttt{arrange()},
\texttt{filter()}

\medskip Select, sort, and print specific vars

\begin{Shaded}
\begin{Highlighting}[]
\CommentTok{\#sort and print}
\NormalTok{hsls\_small }\OperatorTok{\%\textgreater{}\%}\StringTok{ }\KeywordTok{arrange}\NormalTok{(}\KeywordTok{desc}\NormalTok{(stu\_id)) }\OperatorTok{\%\textgreater{}\%}\StringTok{ }
\StringTok{  }\KeywordTok{select}\NormalTok{(stu\_id,x3univ1,x3sqstat,s3classes,s3clglvl)}

\CommentTok{\#investigate variable attributes}
\NormalTok{hsls\_small }\OperatorTok{\%\textgreater{}\%}\StringTok{ }\KeywordTok{arrange}\NormalTok{(}\KeywordTok{desc}\NormalTok{(stu\_id)) }\OperatorTok{\%\textgreater{}\%}\StringTok{ }
\StringTok{  }\KeywordTok{select}\NormalTok{(stu\_id,x3univ1,x3sqstat,s3classes,s3clglvl) }\OperatorTok{\%\textgreater{}\%}\StringTok{ }\KeywordTok{str}\NormalTok{()}

\CommentTok{\#print observations with value labels rather than variable values}
\NormalTok{hsls\_small }\OperatorTok{\%\textgreater{}\%}\StringTok{ }\KeywordTok{arrange}\NormalTok{(}\KeywordTok{desc}\NormalTok{(stu\_id)) }\OperatorTok{\%\textgreater{}\%}\StringTok{ }
\StringTok{  }\KeywordTok{select}\NormalTok{(stu\_id,x3univ1,x3sqstat,s3classes,s3clglvl) }\OperatorTok{\%\textgreater{}\%}\StringTok{ }\KeywordTok{as\_factor}\NormalTok{()}
\end{Highlighting}
\end{Shaded}

Sometimes helpful to increase the number of observations printed

\begin{Shaded}
\begin{Highlighting}[]
\KeywordTok{class}\NormalTok{(hsls\_small) }\CommentTok{\#it\textquotesingle{}s a tibble, which is the "tidyverse" version of a data frame}
\KeywordTok{options}\NormalTok{(}\DataTypeTok{tibble.print\_min=}\DecValTok{50}\NormalTok{) }
\CommentTok{\# execute this in console}
\NormalTok{hsls\_small }\OperatorTok{\%\textgreater{}\%}\StringTok{ }\KeywordTok{arrange}\NormalTok{(}\KeywordTok{desc}\NormalTok{(stu\_id)) }\OperatorTok{\%\textgreater{}\%}
\StringTok{  }\KeywordTok{select}\NormalTok{(stu\_id,x3univ1,x3sqstat,s3classes,s3clglvl)}
\KeywordTok{options}\NormalTok{(}\DataTypeTok{tibble.print\_min=}\DecValTok{10}\NormalTok{) }\CommentTok{\# set default printing back to 10 lines}
\end{Highlighting}
\end{Shaded}
\end{frame}

\begin{frame}[fragile]{One-way descriptive stats for continuous vars,
Base R approach {[}SKIP{]}}
\protect\hypertarget{one-way-descriptive-stats-for-continuous-vars-base-r-approach-skip}{}
\begin{Shaded}
\begin{Highlighting}[]
\KeywordTok{mean}\NormalTok{(hsls\_small}\OperatorTok{$}\NormalTok{x2txmtscor)}
\KeywordTok{sd}\NormalTok{(hsls\_small}\OperatorTok{$}\NormalTok{x2txmtscor)}

\CommentTok{\#Careful: summary stats include value of {-}8!}
\KeywordTok{min}\NormalTok{(hsls\_small}\OperatorTok{$}\NormalTok{x2txmtscor)}
\KeywordTok{max}\NormalTok{(hsls\_small}\OperatorTok{$}\NormalTok{x2txmtscor)}
\end{Highlighting}
\end{Shaded}

Be careful with \texttt{NA} values

\begin{Shaded}
\begin{Highlighting}[]
\CommentTok{\#Create variable replacing {-}8 with NA}
\NormalTok{hsls\_small\_temp \textless{}{-}}\StringTok{ }\NormalTok{hsls\_small }\OperatorTok{\%\textgreater{}\%}\StringTok{ }
\StringTok{  }\KeywordTok{mutate}\NormalTok{(}\DataTypeTok{x2txmtscorv2=}\KeywordTok{ifelse}\NormalTok{(x2txmtscor}\OperatorTok{=={-}}\DecValTok{8}\NormalTok{,}\OtherTok{NA}\NormalTok{,x2txmtscor))}
\NormalTok{hsls\_small\_temp }\OperatorTok{\%\textgreater{}\%}\StringTok{ }\KeywordTok{filter}\NormalTok{(}\KeywordTok{is.na}\NormalTok{(x2txmtscorv2)) }\OperatorTok{\%\textgreater{}\%}\StringTok{ }\KeywordTok{count}\NormalTok{(x2txmtscorv2)}

\KeywordTok{mean}\NormalTok{(hsls\_small\_temp}\OperatorTok{$}\NormalTok{x2txmtscorv2)}
\KeywordTok{mean}\NormalTok{(hsls\_small\_temp}\OperatorTok{$}\NormalTok{x2txmtscorv2, }\DataTypeTok{na.rm=}\OtherTok{TRUE}\NormalTok{)}
\KeywordTok{rm}\NormalTok{(hsls\_small\_temp)}
\end{Highlighting}
\end{Shaded}
\end{frame}

\begin{frame}[fragile]{One-way descriptive stats for continuous vars,
Tidyverse approach}
\protect\hypertarget{one-way-descriptive-stats-for-continuous-vars-tidyverse-approach}{}
Use \texttt{summarise\_at()}, a variation of \texttt{summarise()}, to
make descriptive stats

\begin{itemize}
\tightlist
\item
  \texttt{.args=list(na.rm=TRUE)}= a named list of additional arguments
  to be added to all function calls
\end{itemize}

\textbf{Task}:

\begin{itemize}
\tightlist
\item
  calculate descriptive stats for \texttt{x2txmtscor}, math test score
\end{itemize}

\begin{Shaded}
\begin{Highlighting}[]
\CommentTok{\#?summarise\_at}
\NormalTok{hsls\_small }\OperatorTok{\%\textgreater{}\%}\StringTok{ }\KeywordTok{select}\NormalTok{(x2txmtscor) }\OperatorTok{\%\textgreater{}\%}\StringTok{ }\KeywordTok{var\_label}\NormalTok{()}
\CommentTok{\#\textgreater{} $x2txmtscor}
\CommentTok{\#\textgreater{} [1] "X2 Mathematics standardized theta score"}
\NormalTok{hsls\_small }\OperatorTok{\%\textgreater{}\%}\StringTok{ }
\StringTok{  }\KeywordTok{summarise\_at}\NormalTok{(}
    \DataTypeTok{.vars =} \KeywordTok{vars}\NormalTok{(x2txmtscor),}
    \DataTypeTok{.funs =} \KeywordTok{funs}\NormalTok{(mean, sd, min, max, }\DataTypeTok{.args=}\KeywordTok{list}\NormalTok{(}\DataTypeTok{na.rm=}\OtherTok{TRUE}\NormalTok{))}
\NormalTok{  )}
\CommentTok{\#\textgreater{} \# A tibble: 1 x 4}
\CommentTok{\#\textgreater{}    mean    sd   min   max}
\CommentTok{\#\textgreater{}   \textless{}dbl\textgreater{} \textless{}dbl\textgreater{} \textless{}dbl\textgreater{} \textless{}dbl\textgreater{}}
\CommentTok{\#\textgreater{} 1  44.1  21.8    {-}8  84.9}
\end{Highlighting}
\end{Shaded}
\end{frame}

\begin{frame}[fragile]{One-way descriptive stats for continuous vars,
Tidyverse approach}
\protect\hypertarget{one-way-descriptive-stats-for-continuous-vars-tidyverse-approach-1}{}
Can calculate descriptive stats for more than one variable at a time

\textbf{Task}:

\begin{itemize}
\tightlist
\item
  calculate descriptive stats for \texttt{x2txmtscor}, math test score,
  and \texttt{x4x2ses}, socioeconomic index score
\end{itemize}

\begin{Shaded}
\begin{Highlighting}[]
\NormalTok{hsls\_small }\OperatorTok{\%\textgreater{}\%}\StringTok{ }\KeywordTok{select}\NormalTok{(x2txmtscor,x4x2ses) }\OperatorTok{\%\textgreater{}\%}\StringTok{ }\KeywordTok{var\_label}\NormalTok{()}
\CommentTok{\#\textgreater{} $x2txmtscor}
\CommentTok{\#\textgreater{} [1] "X2 Mathematics standardized theta score"}
\CommentTok{\#\textgreater{} }
\CommentTok{\#\textgreater{} $x4x2ses}
\CommentTok{\#\textgreater{} [1] "X4 Revised X2 Socio{-}economic status composite"}

\NormalTok{hsls\_small }\OperatorTok{\%\textgreater{}\%}\StringTok{ }
\StringTok{  }\KeywordTok{summarise\_at}\NormalTok{(}
    \DataTypeTok{.vars =} \KeywordTok{vars}\NormalTok{(x2txmtscor,x4x2ses),}
    \DataTypeTok{.funs =} \KeywordTok{funs}\NormalTok{(mean, sd, min, max, }\DataTypeTok{.args=}\KeywordTok{list}\NormalTok{(}\DataTypeTok{na.rm=}\OtherTok{TRUE}\NormalTok{))}
\NormalTok{  )}
\CommentTok{\#\textgreater{} \# A tibble: 1 x 8}
\CommentTok{\#\textgreater{}   x2txmtscor\_mean x4x2ses\_mean x2txmtscor\_sd x4x2ses\_sd x2txmtscor\_min}
\CommentTok{\#\textgreater{}             \textless{}dbl\textgreater{}        \textless{}dbl\textgreater{}         \textless{}dbl\textgreater{}      \textless{}dbl\textgreater{}          \textless{}dbl\textgreater{}}
\CommentTok{\#\textgreater{} 1            44.1       {-}0.802          21.8       2.63             {-}8}
\CommentTok{\#\textgreater{} \# ... with 3 more variables: x4x2ses\_min \textless{}dbl\textgreater{}, x2txmtscor\_max \textless{}dbl\textgreater{},}
\CommentTok{\#\textgreater{} \#   x4x2ses\_max \textless{}dbl\textgreater{}}
\end{Highlighting}
\end{Shaded}
\end{frame}

\begin{frame}[fragile]{One-way descriptive stats for continuous vars,
Tidyverse approach}
\protect\hypertarget{one-way-descriptive-stats-for-continuous-vars-tidyverse-approach-2}{}
``Input vars'' in survey data often have negative values for
missing/skips

\begin{Shaded}
\begin{Highlighting}[]
\NormalTok{hsls\_small }\OperatorTok{\%\textgreater{}\%}\StringTok{ }\KeywordTok{filter}\NormalTok{(x2txmtscor}\OperatorTok{\textless{}}\DecValTok{0}\NormalTok{) }\OperatorTok{\%\textgreater{}\%}\StringTok{ }\KeywordTok{count}\NormalTok{(x2txmtscor)}
\end{Highlighting}
\end{Shaded}

R includes those negative values when calculating stats; you don't want
this

\begin{itemize}
\tightlist
\item
  Solution: create version of variable that replaces negative values
  with \texttt{NA}
\end{itemize}

\begin{Shaded}
\begin{Highlighting}[]
\NormalTok{hsls\_small }\OperatorTok{\%\textgreater{}\%}\StringTok{ }\KeywordTok{mutate}\NormalTok{(}\DataTypeTok{x2txmtscor\_na=}\KeywordTok{ifelse}\NormalTok{(x2txmtscor}\OperatorTok{\textless{}}\DecValTok{0}\NormalTok{,}\OtherTok{NA}\NormalTok{,x2txmtscor)) }\OperatorTok{\%\textgreater{}\%}
\StringTok{  }\KeywordTok{summarise\_at}\NormalTok{(}
    \DataTypeTok{.vars =} \KeywordTok{vars}\NormalTok{(x2txmtscor\_na),}
    \DataTypeTok{.funs =} \KeywordTok{funs}\NormalTok{(mean, sd, min, max, }\DataTypeTok{.args=}\KeywordTok{list}\NormalTok{(}\DataTypeTok{na.rm=}\OtherTok{TRUE}\NormalTok{))}
\NormalTok{  )}
\CommentTok{\#\textgreater{} \# A tibble: 1 x 4}
\CommentTok{\#\textgreater{}    mean    sd   min   max}
\CommentTok{\#\textgreater{}   \textless{}dbl\textgreater{} \textless{}dbl\textgreater{} \textless{}dbl\textgreater{} \textless{}dbl\textgreater{}}
\CommentTok{\#\textgreater{} 1  51.5  10.2  22.2  84.9}
\end{Highlighting}
\end{Shaded}

What if you didn't include \texttt{.args=list(na.rm=TRUE)}?

\begin{Shaded}
\begin{Highlighting}[]
\NormalTok{hsls\_small }\OperatorTok{\%\textgreater{}\%}\StringTok{ }\KeywordTok{mutate}\NormalTok{(}\DataTypeTok{x2txmtscor\_na=}\KeywordTok{ifelse}\NormalTok{(x2txmtscor}\OperatorTok{\textless{}}\DecValTok{0}\NormalTok{,}\OtherTok{NA}\NormalTok{,x2txmtscor)) }\OperatorTok{\%\textgreater{}\%}
\StringTok{  }\KeywordTok{summarise\_at}\NormalTok{(}
    \DataTypeTok{.vars =} \KeywordTok{vars}\NormalTok{(x2txmtscor\_na),}
    \DataTypeTok{.funs =} \KeywordTok{funs}\NormalTok{(mean, sd, min, max))}
\CommentTok{\#\textgreater{} \# A tibble: 1 x 4}
\CommentTok{\#\textgreater{}    mean    sd   min   max}
\CommentTok{\#\textgreater{}   \textless{}dbl\textgreater{} \textless{}dbl\textgreater{} \textless{}dbl\textgreater{} \textless{}dbl\textgreater{}}
\CommentTok{\#\textgreater{} 1    NA    NA    NA    NA}
\end{Highlighting}
\end{Shaded}
\end{frame}

\begin{frame}[fragile]{One-way descriptive stats for continuous vars,
Tidyverse approach}
\protect\hypertarget{one-way-descriptive-stats-for-continuous-vars-tidyverse-approach-3}{}
How to identify these missing/skip values if you don't have a codebook?

\begin{itemize}
\tightlist
\item
  \texttt{count()} combined with \texttt{filter()} helpful for finding
  extreme values of continuous vars, which are often associated with
  missing or skip
\end{itemize}

\begin{Shaded}
\begin{Highlighting}[]
\CommentTok{\#variable x2txmtscor}
\NormalTok{hsls\_small }\OperatorTok{\%\textgreater{}\%}\StringTok{ }\KeywordTok{filter}\NormalTok{(x2txmtscor}\OperatorTok{\textless{}}\DecValTok{0}\NormalTok{) }\OperatorTok{\%\textgreater{}\%}\StringTok{ }
\StringTok{  }\KeywordTok{count}\NormalTok{(x2txmtscor)}
\CommentTok{\#\textgreater{} \# A tibble: 1 x 2}
\CommentTok{\#\textgreater{}   x2txmtscor     n}
\CommentTok{\#\textgreater{}        \textless{}dbl\textgreater{} \textless{}int\textgreater{}}
\CommentTok{\#\textgreater{} 1         {-}8  2909}

\CommentTok{\#variable s3clglvl}
\NormalTok{hsls\_small }\OperatorTok{\%\textgreater{}\%}\StringTok{ }\KeywordTok{select}\NormalTok{(s3clglvl) }\OperatorTok{\%\textgreater{}\%}\StringTok{ }\KeywordTok{var\_label}\NormalTok{()}
\CommentTok{\#\textgreater{} $s3clglvl}
\CommentTok{\#\textgreater{} [1] "S3 Enrolled college IPEDS level"}

\NormalTok{hsls\_small }\OperatorTok{\%\textgreater{}\%}\StringTok{ }\KeywordTok{filter}\NormalTok{(s3clglvl}\OperatorTok{\textless{}}\DecValTok{0}\NormalTok{) }\OperatorTok{\%\textgreater{}\%}
\StringTok{  }\KeywordTok{count}\NormalTok{(s3clglvl)}
\CommentTok{\#\textgreater{} \# A tibble: 3 x 2}
\CommentTok{\#\textgreater{}                       s3clglvl     n}
\CommentTok{\#\textgreater{}                      \textless{}dbl+lbl\textgreater{} \textless{}int\textgreater{}}
\CommentTok{\#\textgreater{} 1 {-}9 [Missing]                   487}
\CommentTok{\#\textgreater{} 2 {-}8 [Unit non{-}response]        4945}
\CommentTok{\#\textgreater{} 3 {-}7 [Item legitimate skip/NA]  5022}
\end{Highlighting}
\end{Shaded}
\end{frame}

\begin{frame}[fragile]{One-way descriptive stats student exercise}
\protect\hypertarget{one-way-descriptive-stats-student-exercise}{}
\begin{enumerate}
\tightlist
\item
  Using the object \texttt{hsls}, identify variable type, variable
  class, and check the variable values and value labels of
  \texttt{x4ps1start}

  \begin{itemize}
  \tightlist
  \item
    variable \texttt{x4ps1start} identifies month and year student first
    started postsecondary education
  \item
    \textbf{Note}: This variable is a bit counterintuitive.

    \begin{itemize}
    \tightlist
    \item
      e.g., the value \texttt{201105} refers to May 2011
    \end{itemize}
  \end{itemize}
\item
  Get a frequency count of the variable \texttt{x4ps1start}\\
\item
  Get a frequency count of the variable, but this time only observations
  that have negative values \textbf{hint}: use filter()\\
\item
  Create a new version of the variable \texttt{x4ps1start\_na} that
  replaces negative values with NAs and use \texttt{summarise\_at()} to
  get the min and max value.
\end{enumerate}
\end{frame}

\begin{frame}[fragile]{One-way descriptive stats student exercise
solutions}
\protect\hypertarget{one-way-descriptive-stats-student-exercise-solutions}{}
\medskip

\begin{enumerate}
\tightlist
\item
  Using the object \texttt{hsls}, identify variable type, variable
  class, and check the variable vakyes and value labels of
  \texttt{x4ps1start}
\end{enumerate}

\begin{Shaded}
\begin{Highlighting}[]
\KeywordTok{typeof}\NormalTok{(hsls}\OperatorTok{$}\NormalTok{x4ps1start)}
\CommentTok{\#\textgreater{} [1] "double"}
\KeywordTok{class}\NormalTok{(hsls}\OperatorTok{$}\NormalTok{x4ps1start)}
\CommentTok{\#\textgreater{} [1] "haven\_labelled" "vctrs\_vctr"     "double"}

\NormalTok{hsls }\OperatorTok{\%\textgreater{}\%}\StringTok{ }\KeywordTok{select}\NormalTok{(x4ps1start) }\OperatorTok{\%\textgreater{}\%}\StringTok{ }\KeywordTok{var\_label}\NormalTok{()}
\CommentTok{\#\textgreater{} $x4ps1start}
\CommentTok{\#\textgreater{} [1] "X4 Month and year of enrollment at first postsecondary institution"}

\NormalTok{hsls }\OperatorTok{\%\textgreater{}\%}\StringTok{ }\KeywordTok{select}\NormalTok{(x4ps1start) }\OperatorTok{\%\textgreater{}\%}\StringTok{ }\KeywordTok{val\_labels}\NormalTok{()}
\CommentTok{\#\textgreater{} $x4ps1start}
\CommentTok{\#\textgreater{}                                      Missing }
\CommentTok{\#\textgreater{}                                           {-}9 }
\CommentTok{\#\textgreater{}                            Unit non{-}response }
\CommentTok{\#\textgreater{}                                           {-}8 }
\CommentTok{\#\textgreater{}                      Item legitimate skip/NA }
\CommentTok{\#\textgreater{}                                           {-}7 }
\CommentTok{\#\textgreater{}                     Component not applicable }
\CommentTok{\#\textgreater{}                                           {-}6 }
\CommentTok{\#\textgreater{} Item not administered: abbreviated interview }
\CommentTok{\#\textgreater{}                                           {-}4 }
\CommentTok{\#\textgreater{}                        Carry through missing }
\CommentTok{\#\textgreater{}                                           {-}3 }
\CommentTok{\#\textgreater{}                                   Don\textquotesingle{}t know }
\CommentTok{\#\textgreater{}                                           {-}1}
\end{Highlighting}
\end{Shaded}
\end{frame}

\begin{frame}[fragile]{One-way descriptive stats student exercise
solutions}
\protect\hypertarget{one-way-descriptive-stats-student-exercise-solutions-1}{}
\begin{enumerate}
\setcounter{enumi}{1}
\tightlist
\item
  Get a frequency count of the variable \texttt{x4ps1start}
\end{enumerate}

\begin{Shaded}
\begin{Highlighting}[]
\NormalTok{hsls }\OperatorTok{\%\textgreater{}\%}
\StringTok{  }\KeywordTok{count}\NormalTok{(x4ps1start)}
\CommentTok{\#\textgreater{} \# A tibble: 9 x 2}
\CommentTok{\#\textgreater{}                         x4ps1start     n}
\CommentTok{\#\textgreater{}                          \textless{}dbl+lbl\textgreater{} \textless{}int\textgreater{}}
\CommentTok{\#\textgreater{} 1     {-}9 [Missing]                   107}
\CommentTok{\#\textgreater{} 2     {-}8 [Unit non{-}response]        6168}
\CommentTok{\#\textgreater{} 3     {-}7 [Item legitimate skip/NA]  4281}
\CommentTok{\#\textgreater{} 4 201100                              57}
\CommentTok{\#\textgreater{} 5 201200                             206}
\CommentTok{\#\textgreater{} 6 201300                           10800}
\CommentTok{\#\textgreater{} 7 201400                            1295}
\CommentTok{\#\textgreater{} 8 201500                             471}
\CommentTok{\#\textgreater{} 9 201600                             118}
\end{Highlighting}
\end{Shaded}
\end{frame}

\begin{frame}[fragile]{One-way descriptive stats student exercise
solutions}
\protect\hypertarget{one-way-descriptive-stats-student-exercise-solutions-2}{}
\begin{enumerate}
\setcounter{enumi}{2}
\tightlist
\item
  Get a frequency count of the variable, but this time only observations
  that have negative values \textbf{hint}: use filter()
\end{enumerate}

\begin{Shaded}
\begin{Highlighting}[]
\NormalTok{hsls }\OperatorTok{\%\textgreater{}\%}\StringTok{ }
\StringTok{  }\KeywordTok{filter}\NormalTok{(x4ps1start}\OperatorTok{\textless{}}\DecValTok{0}\NormalTok{) }\OperatorTok{\%\textgreater{}\%}\StringTok{ }
\StringTok{  }\KeywordTok{count}\NormalTok{(x4ps1start)}
\CommentTok{\#\textgreater{} \# A tibble: 3 x 2}
\CommentTok{\#\textgreater{}                     x4ps1start     n}
\CommentTok{\#\textgreater{}                      \textless{}dbl+lbl\textgreater{} \textless{}int\textgreater{}}
\CommentTok{\#\textgreater{} 1 {-}9 [Missing]                   107}
\CommentTok{\#\textgreater{} 2 {-}8 [Unit non{-}response]        6168}
\CommentTok{\#\textgreater{} 3 {-}7 [Item legitimate skip/NA]  4281}
\end{Highlighting}
\end{Shaded}
\end{frame}

\begin{frame}[fragile]{One-way descriptive stats student exercise
solutions}
\protect\hypertarget{one-way-descriptive-stats-student-exercise-solutions-3}{}
\begin{enumerate}
\setcounter{enumi}{3}
\tightlist
\item
  Create a new version \texttt{x4ps1start\_na} of the variable
  \texttt{x4ps1start} that replaces negative values with NAs and use
  \texttt{summarise\_at()} to get the min and max value.
\end{enumerate}

\begin{Shaded}
\begin{Highlighting}[]
\NormalTok{hsls }\OperatorTok{\%\textgreater{}\%}\StringTok{ }\KeywordTok{mutate}\NormalTok{(}\DataTypeTok{x4ps1start\_na=}\KeywordTok{ifelse}\NormalTok{(x4ps1start}\OperatorTok{\textless{}}\DecValTok{0}\NormalTok{,}\OtherTok{NA}\NormalTok{,x4ps1start)) }\OperatorTok{\%\textgreater{}\%}
\StringTok{  }\KeywordTok{summarise\_at}\NormalTok{(}
    \DataTypeTok{.vars =} \KeywordTok{vars}\NormalTok{(x4ps1start\_na),}
    \DataTypeTok{.funs =} \KeywordTok{funs}\NormalTok{(min, max, }\DataTypeTok{.args=}\KeywordTok{list}\NormalTok{(}\DataTypeTok{na.rm=}\OtherTok{TRUE}\NormalTok{))}
\NormalTok{  )}
\CommentTok{\#\textgreater{} \# A tibble: 1 x 2}
\CommentTok{\#\textgreater{}      min    max}
\CommentTok{\#\textgreater{}    \textless{}dbl\textgreater{}  \textless{}dbl\textgreater{}}
\CommentTok{\#\textgreater{} 1 201100 201600}
\end{Highlighting}
\end{Shaded}
\end{frame}

\begin{frame}[fragile]{One-way descriptive stats for
discrete/categorical vars, Tidyverse approach}
\protect\hypertarget{one-way-descriptive-stats-for-discretecategorical-vars-tidyverse-approach}{}
Use \texttt{count()} to investigate values of discrete or categorical
variables

For variables where \texttt{class==labelled}

\begin{Shaded}
\begin{Highlighting}[]
\KeywordTok{class}\NormalTok{(hsls\_small}\OperatorTok{$}\NormalTok{s3classes)}
\KeywordTok{attributes}\NormalTok{(hsls\_small}\OperatorTok{$}\NormalTok{s3classes)}
\CommentTok{\#show counts of variable values}
\NormalTok{hsls\_small }\OperatorTok{\%\textgreater{}\%}\StringTok{ }\KeywordTok{count}\NormalTok{(s3classes) }\CommentTok{\#print in console to show both}
\CommentTok{\#show counts of value labels}
\NormalTok{hsls\_small }\OperatorTok{\%\textgreater{}\%}\StringTok{ }\KeywordTok{count}\NormalTok{(s3classes) }\OperatorTok{\%\textgreater{}\%}\StringTok{ }\KeywordTok{as\_factor}\NormalTok{()}
\end{Highlighting}
\end{Shaded}

\begin{itemize}
\tightlist
\item
  I like \texttt{count()} because the default setting is to show
  \texttt{NA} values too!
\end{itemize}

\begin{Shaded}
\begin{Highlighting}[]
\NormalTok{hsls\_small }\OperatorTok{\%\textgreater{}\%}\StringTok{ }\KeywordTok{mutate}\NormalTok{(}\DataTypeTok{s3classes\_na=}\KeywordTok{ifelse}\NormalTok{(s3classes}\OperatorTok{\textless{}}\DecValTok{0}\NormalTok{,}\OtherTok{NA}\NormalTok{,s3classes)) }\OperatorTok{\%\textgreater{}\%}\StringTok{ }
\StringTok{  }\KeywordTok{count}\NormalTok{(s3classes\_na)}
\end{Highlighting}
\end{Shaded}

Simultaneously show both values and value labels on count tables for
\texttt{class==labelled} if entered into console

\begin{itemize}
\tightlist
\item
  This requires some concepts/functions we haven't introduced {[}SKIP{]}
\end{itemize}

\begin{Shaded}
\begin{Highlighting}[]
\NormalTok{ hsls\_small }\OperatorTok{\%\textgreater{}\%}\StringTok{ }\KeywordTok{count}\NormalTok{(s3classes)}
\NormalTok{y \textless{}{-}}\StringTok{ }\NormalTok{hsls\_small }\OperatorTok{\%\textgreater{}\%}\StringTok{ }\KeywordTok{count}\NormalTok{(s3classes) }\OperatorTok{\%\textgreater{}\%}\StringTok{ }\KeywordTok{as\_factor}\NormalTok{()}
 \KeywordTok{bind\_cols}\NormalTok{(x[,}\DecValTok{1}\NormalTok{], y) }\CommentTok{\#wont show in updated R}
\end{Highlighting}
\end{Shaded}
\end{frame}

\begin{frame}[fragile]{Relationship between variables, categorical by
categorical}
\protect\hypertarget{relationship-between-variables-categorical-by-categorical}{}
Two-way frequency table, called ``cross tabulation'', important for data
quality

\begin{itemize}
\tightlist
\item
  When you create categorical analysis var from single categorical
  ``input'' var

  \begin{itemize}
  \tightlist
  \item
    Two-way tables show us whether we did this correctly\\
  \end{itemize}
\item
  Two-way tables helpful for understanding skip patterns in surveys
\end{itemize}

\textbf{key to syntax}

\begin{itemize}
\tightlist
\item
  \texttt{df\_name\ \%\textgreater{}\%\ group\_by(var1)\ \%\textgreater{}\%\ count(var2)}
  \textbf{OR}
\item
  \texttt{df\_name\ \%\textgreater{}\%\ count(var1,var2)}
\item
  play around with which variable is \texttt{var1} and which variable is
  \texttt{var2}
\end{itemize}
\end{frame}

\begin{frame}[fragile]{Relationship between variables, categorical by
categorical}
\protect\hypertarget{relationship-between-variables-categorical-by-categorical-1}{}
\textbf{Task}: Create a two-way table between \texttt{s3classes} and
\texttt{s3clglvl}

\begin{itemize}
\tightlist
\item
  Investigate variables
\end{itemize}

\begin{Shaded}
\begin{Highlighting}[]
\NormalTok{hsls\_small }\OperatorTok{\%\textgreater{}\%}\StringTok{ }\KeywordTok{select}\NormalTok{(s3classes,s3clglvl) }\OperatorTok{\%\textgreater{}\%}\StringTok{ }\KeywordTok{var\_label}\NormalTok{()}
\NormalTok{hsls\_small }\OperatorTok{\%\textgreater{}\%}\StringTok{ }\KeywordTok{select}\NormalTok{(s3classes,s3clglvl) }\OperatorTok{\%\textgreater{}\%}\StringTok{ }\KeywordTok{val\_labels}\NormalTok{()}
\end{Highlighting}
\end{Shaded}

\begin{itemize}
\tightlist
\item
  Create two-way table
\end{itemize}

\begin{Shaded}
\begin{Highlighting}[]
\NormalTok{hsls\_small }\OperatorTok{\%\textgreater{}\%}\StringTok{ }\KeywordTok{group\_by}\NormalTok{(s3classes) }\OperatorTok{\%\textgreater{}\%}\StringTok{ }\KeywordTok{count}\NormalTok{(s3clglvl) }\CommentTok{\# show values}
\NormalTok{hsls\_small }\OperatorTok{\%\textgreater{}\%}\StringTok{ }\KeywordTok{count}\NormalTok{(s3classes,s3clglvl)}
\CommentTok{\#hsls\_small \%\textgreater{}\% group\_by(s3classes) \%\textgreater{}\% count(s3clglvl) \%\textgreater{}\% as\_factor() \# show value labels}
\end{Highlighting}
\end{Shaded}

\begin{itemize}
\tightlist
\item
  Are these objects the same?
\end{itemize}

\begin{Shaded}
\begin{Highlighting}[]
\NormalTok{hsls\_small }\OperatorTok{\%\textgreater{}\%}\StringTok{ }\KeywordTok{group\_by}\NormalTok{(s3classes) }\OperatorTok{\%\textgreater{}\%}\StringTok{ }\KeywordTok{count}\NormalTok{(s3clglvl) }\OperatorTok{\%\textgreater{}\%}\StringTok{ }\KeywordTok{glimpse}\NormalTok{()}
\CommentTok{\#\textgreater{} Rows: 8}
\CommentTok{\#\textgreater{} Columns: 3}
\CommentTok{\#\textgreater{} Groups: s3classes [5]}
\CommentTok{\#\textgreater{} $ s3classes \textless{}dbl+lbl\textgreater{} {-}9, {-}8,  1,  1,  1,  1,  2,  3}
\CommentTok{\#\textgreater{} $ s3clglvl  \textless{}dbl+lbl\textgreater{} {-}9, {-}8, {-}9,  1,  2,  3, {-}7, {-}7}
\CommentTok{\#\textgreater{} $ n         \textless{}int\textgreater{} 59, 4945, 428, 8894, 3929, 226, 3401, 1621}
\NormalTok{hsls\_small }\OperatorTok{\%\textgreater{}\%}\StringTok{ }\KeywordTok{count}\NormalTok{(s3classes,s3clglvl) }\OperatorTok{\%\textgreater{}\%}\StringTok{ }\KeywordTok{glimpse}\NormalTok{()}
\CommentTok{\#\textgreater{} Rows: 8}
\CommentTok{\#\textgreater{} Columns: 3}
\CommentTok{\#\textgreater{} $ s3classes \textless{}dbl+lbl\textgreater{} {-}9, {-}8,  1,  1,  1,  1,  2,  3}
\CommentTok{\#\textgreater{} $ s3clglvl  \textless{}dbl+lbl\textgreater{} {-}9, {-}8, {-}9,  1,  2,  3, {-}7, {-}7}
\CommentTok{\#\textgreater{} $ n         \textless{}int\textgreater{} 59, 4945, 428, 8894, 3929, 226, 3401, 1621}
\end{Highlighting}
\end{Shaded}
\end{frame}

\begin{frame}[fragile]{Relationship between variables, categorical by
categorical}
\protect\hypertarget{relationship-between-variables-categorical-by-categorical-2}{}
Two-way frequency table, also called ``cross tabulation''

\textbf{Task}:

\begin{itemize}
\tightlist
\item
  Create a version of \texttt{s3classes} called \texttt{s3classes\_na}
  that changes negative values to \texttt{NA}
\item
  Create a two-way table between \texttt{s3classes\_na} and
  \texttt{s3clglvl}
\end{itemize}

\begin{Shaded}
\begin{Highlighting}[]
\NormalTok{hsls\_small }\OperatorTok{\%\textgreater{}\%}\StringTok{ }
\StringTok{  }\KeywordTok{mutate}\NormalTok{(}\DataTypeTok{s3classes\_na=}\KeywordTok{ifelse}\NormalTok{(s3classes}\OperatorTok{\textless{}}\DecValTok{0}\NormalTok{,}\OtherTok{NA}\NormalTok{,s3classes)) }\OperatorTok{\%\textgreater{}\%}
\StringTok{  }\KeywordTok{group\_by}\NormalTok{(s3classes\_na) }\OperatorTok{\%\textgreater{}\%}\StringTok{ }\KeywordTok{count}\NormalTok{(s3clglvl)}

\NormalTok{hsls\_small }\OperatorTok{\%\textgreater{}\%}\StringTok{ }
\StringTok{  }\KeywordTok{mutate}\NormalTok{(}\DataTypeTok{s3classes\_na=}\KeywordTok{ifelse}\NormalTok{(s3classes}\OperatorTok{\textless{}}\DecValTok{0}\NormalTok{,}\OtherTok{NA}\NormalTok{,s3classes)) }\OperatorTok{\%\textgreater{}\%}
\StringTok{  }\KeywordTok{count}\NormalTok{(s3classes\_na, s3clglvl)}


\CommentTok{\#example where we create some NA obs in the second variable}
\NormalTok{hsls\_small }\OperatorTok{\%\textgreater{}\%}\StringTok{ }
\StringTok{  }\KeywordTok{mutate}\NormalTok{(}\DataTypeTok{s3classes\_na=}\KeywordTok{ifelse}\NormalTok{(s3classes}\OperatorTok{\textless{}}\DecValTok{0}\NormalTok{,}\OtherTok{NA}\NormalTok{,s3classes),}
         \DataTypeTok{s3clglvl\_na=}\KeywordTok{ifelse}\NormalTok{(s3clglvl}\OperatorTok{=={-}}\DecValTok{7}\NormalTok{,}\OtherTok{NA}\NormalTok{,s3clglvl)) }\OperatorTok{\%\textgreater{}\%}
\StringTok{  }\KeywordTok{group\_by}\NormalTok{(s3classes\_na) }\OperatorTok{\%\textgreater{}\%}\StringTok{ }\KeywordTok{count}\NormalTok{(s3clglvl\_na)}


\NormalTok{hsls\_small }\OperatorTok{\%\textgreater{}\%}\StringTok{ }
\StringTok{  }\KeywordTok{mutate}\NormalTok{(}\DataTypeTok{s3classes\_na=}\KeywordTok{ifelse}\NormalTok{(s3classes}\OperatorTok{\textless{}}\DecValTok{0}\NormalTok{,}\OtherTok{NA}\NormalTok{,s3classes),}
         \DataTypeTok{s3clglvl\_na=}\KeywordTok{ifelse}\NormalTok{(s3clglvl}\OperatorTok{=={-}}\DecValTok{7}\NormalTok{,}\OtherTok{NA}\NormalTok{,s3clglvl)) }\OperatorTok{\%\textgreater{}\%}
\StringTok{  }\KeywordTok{count}\NormalTok{(s3classes\_na, s3clglvl\_na)}
\end{Highlighting}
\end{Shaded}
\end{frame}

\begin{frame}[fragile]{Relationship between variables, categorical by
categorical {[}SKIP{]}}
\protect\hypertarget{relationship-between-variables-categorical-by-categorical-skip}{}
Tables above are pretty ugly

Use the \texttt{spread()} function from \texttt{tidyr} package to create
table with one variable as columns and the other variable as rows

\begin{itemize}
\tightlist
\item
  The variable you place in \texttt{spread()} will be columns
\item
  We learn \texttt{spread()} function next week
\end{itemize}

\begin{Shaded}
\begin{Highlighting}[]
\NormalTok{hsls\_small }\OperatorTok{\%\textgreater{}\%}\StringTok{ }\KeywordTok{group\_by}\NormalTok{(s3classes) }\OperatorTok{\%\textgreater{}\%}\StringTok{ }\KeywordTok{count}\NormalTok{(s3clglvl) }\OperatorTok{\%\textgreater{}\%}\StringTok{ }
\StringTok{  }\KeywordTok{spread}\NormalTok{(s3classes, n)}

\NormalTok{hsls\_small }\OperatorTok{\%\textgreater{}\%}\StringTok{ }\KeywordTok{group\_by}\NormalTok{(s3classes) }\OperatorTok{\%\textgreater{}\%}\StringTok{ }\KeywordTok{count}\NormalTok{(s3clglvl) }\OperatorTok{\%\textgreater{}\%}\StringTok{ }
\StringTok{  }\KeywordTok{as\_factor}\NormalTok{() }\OperatorTok{\%\textgreater{}\%}\StringTok{  }\KeywordTok{spread}\NormalTok{(s3classes, n)}
\NormalTok{hsls\_small }\OperatorTok{\%\textgreater{}\%}\StringTok{ }\KeywordTok{group\_by}\NormalTok{(s3classes) }\OperatorTok{\%\textgreater{}\%}\StringTok{ }\KeywordTok{count}\NormalTok{(s3clglvl) }\OperatorTok{\%\textgreater{}\%}\StringTok{ }
\StringTok{  }\KeywordTok{as\_factor}\NormalTok{() }\OperatorTok{\%\textgreater{}\%}\StringTok{  }\KeywordTok{spread}\NormalTok{(s3clglvl, n)}
\end{Highlighting}
\end{Shaded}
\end{frame}

\begin{frame}[fragile]{Relationship between variables, categorical by
continuous}
\protect\hypertarget{relationship-between-variables-categorical-by-continuous}{}
Investigating relationship between multiple variables is a little
tougher when at least one of the variables is continuous

\textbf{Conditional mean} (like regression with continuous Y and one
categorical X):

\begin{itemize}
\tightlist
\item
  Shows average values of continous variables within groups
\item
  Groups are defined by your categorical variable(s)
\end{itemize}

\textbf{key to syntax}

\begin{itemize}
\tightlist
\item
  \texttt{group\_by(categorical\_var)\ \%\textgreater{}\%\ summarise\_at(.vars\ =\ vars(continuous\_var)}
\end{itemize}
\end{frame}

\begin{frame}[fragile]{Relationship between variables, categorical by
continuous}
\protect\hypertarget{relationship-between-variables-categorical-by-continuous-1}{}
\textbf{Task}

\begin{itemize}
\tightlist
\item
  Calculate mean math score, \texttt{x2txmtscor}, for each value of
  parental education, \texttt{x2paredu}
\end{itemize}

\begin{Shaded}
\begin{Highlighting}[]
\CommentTok{\#first, investigate parental education [print in console]}
\NormalTok{hsls\_small }\OperatorTok{\%\textgreater{}\%}\StringTok{ }\KeywordTok{count}\NormalTok{(x2paredu) }
\end{Highlighting}
\end{Shaded}

\begin{Shaded}
\begin{Highlighting}[]
\CommentTok{\# using dplyr to get average math score by parental education level [print in console]}
\NormalTok{hsls\_small }\OperatorTok{\%\textgreater{}\%}\StringTok{ }\KeywordTok{group\_by}\NormalTok{(x2paredu) }\OperatorTok{\%\textgreater{}\%}
\StringTok{    }\KeywordTok{summarise\_at}\NormalTok{(}\DataTypeTok{.vars =} \KeywordTok{vars}\NormalTok{(x2txmtscor),}
                 \DataTypeTok{.funs =} \KeywordTok{funs}\NormalTok{(mean, }\DataTypeTok{.args =} \KeywordTok{list}\NormalTok{(}\DataTypeTok{na.rm =} \OtherTok{TRUE}\NormalTok{))) }
\CommentTok{\#\textgreater{} \# A tibble: 8 x 2}
\CommentTok{\#\textgreater{}                                                              x2paredu x2txmtscor}
\CommentTok{\#\textgreater{}                                                             \textless{}dbl+lbl\textgreater{}      \textless{}dbl\textgreater{}}
\CommentTok{\#\textgreater{} 1 {-}8 [Unit non{-}response]                                                    {-}8  }
\CommentTok{\#\textgreater{} 2  1 [Less than high school]                                                44.3}
\CommentTok{\#\textgreater{} 3  2 [High school diploma or GED or alterntive HS credential]               47.2}
\CommentTok{\#\textgreater{} 4  3 [Certificate/diploma from school providing occupational trainin\textasciitilde{}       46.4}
\CommentTok{\#\textgreater{} 5  4 [Associate\textquotesingle{}s degree]                                                   48.9}
\CommentTok{\#\textgreater{} 6  5 [Bachelor\textquotesingle{}s degree]                                                    53.3}
\CommentTok{\#\textgreater{} 7  6 [Master\textquotesingle{}s degree]                                                      55.6}
\CommentTok{\#\textgreater{} 8  7 [Ph.D/M.D/Law/other high lvl prof degree]                              58.9}
\end{Highlighting}
\end{Shaded}
\end{frame}

\begin{frame}[fragile]{Relationship between variables, categorical by
continuous}
\protect\hypertarget{relationship-between-variables-categorical-by-continuous-2}{}
\textbf{Task}

\begin{itemize}
\tightlist
\item
  Calculate mean math score, \texttt{x2txmtscor}, for each value of
  \texttt{x2paredu}
\end{itemize}

For checking data quality, helpful to calculate other stats besides mean

\begin{Shaded}
\begin{Highlighting}[]
\NormalTok{hsls\_small }\OperatorTok{\%\textgreater{}\%}\StringTok{ }\KeywordTok{group\_by}\NormalTok{(x2paredu) }\OperatorTok{\%\textgreater{}\%}\StringTok{ }\CommentTok{\#[print in console]}
\StringTok{    }\KeywordTok{summarise\_at}\NormalTok{(}\DataTypeTok{.vars =} \KeywordTok{vars}\NormalTok{(x2txmtscor),}
                 \DataTypeTok{.funs =} \KeywordTok{funs}\NormalTok{(mean, min, max, }\DataTypeTok{.args =} \KeywordTok{list}\NormalTok{(}\DataTypeTok{na.rm =} \OtherTok{TRUE}\NormalTok{)))}
\end{Highlighting}
\end{Shaded}

Always Investigate presence of missing/skip values

\begin{Shaded}
\begin{Highlighting}[]
\NormalTok{hsls\_small }\OperatorTok{\%\textgreater{}\%}\StringTok{ }\KeywordTok{filter}\NormalTok{(x2paredu}\OperatorTok{\textless{}}\DecValTok{0}\NormalTok{) }\OperatorTok{\%\textgreater{}\%}\StringTok{ }\KeywordTok{count}\NormalTok{(x2paredu)}
\NormalTok{hsls\_small }\OperatorTok{\%\textgreater{}\%}\StringTok{ }\KeywordTok{filter}\NormalTok{(x2txmtscor}\OperatorTok{\textless{}}\DecValTok{0}\NormalTok{) }\OperatorTok{\%\textgreater{}\%}\StringTok{ }\KeywordTok{count}\NormalTok{(x2txmtscor)}
\end{Highlighting}
\end{Shaded}

Replace \texttt{-8} with \texttt{NA} and re-calculate conditional stats

\begin{Shaded}
\begin{Highlighting}[]
\NormalTok{hsls\_small }\OperatorTok{\%\textgreater{}\%}\StringTok{ }
\StringTok{  }\KeywordTok{mutate}\NormalTok{(}\DataTypeTok{x2paredu\_na=}\KeywordTok{ifelse}\NormalTok{(x2paredu}\OperatorTok{\textless{}}\DecValTok{0}\NormalTok{,}\OtherTok{NA}\NormalTok{,x2paredu),}
         \DataTypeTok{x2txmtscor\_na=}\KeywordTok{ifelse}\NormalTok{(x2txmtscor}\OperatorTok{\textless{}}\DecValTok{0}\NormalTok{,}\OtherTok{NA}\NormalTok{,x2txmtscor)) }\OperatorTok{\%\textgreater{}\%}\StringTok{ }
\StringTok{  }\KeywordTok{group\_by}\NormalTok{(x2paredu\_na) }\OperatorTok{\%\textgreater{}\%}
\StringTok{  }\KeywordTok{summarise\_at}\NormalTok{(}\DataTypeTok{.vars =} \KeywordTok{vars}\NormalTok{(x2txmtscor\_na),}
               \DataTypeTok{.funs =} \KeywordTok{funs}\NormalTok{(mean, min, max, }\DataTypeTok{.args =} \KeywordTok{list}\NormalTok{(}\DataTypeTok{na.rm =} \OtherTok{TRUE}\NormalTok{))) }\OperatorTok{\%\textgreater{}\%}
\StringTok{  }\KeywordTok{as\_factor}\NormalTok{()}
\CommentTok{\#\textgreater{} Warning in min(x2txmtscor\_na, na.rm = TRUE): no non{-}missing arguments to min;}
\CommentTok{\#\textgreater{} returning Inf}
\CommentTok{\#\textgreater{} Warning in max(x2txmtscor\_na, na.rm = TRUE): no non{-}missing arguments to max;}
\CommentTok{\#\textgreater{} returning {-}Inf}

\NormalTok{hsls\_small }\OperatorTok{\%\textgreater{}\%}\StringTok{ }\KeywordTok{count}\NormalTok{(s3classes,s3clglvl) }\OperatorTok{\%\textgreater{}\%}\StringTok{ }\NormalTok{as\_factor}
\end{Highlighting}
\end{Shaded}
\end{frame}

\begin{frame}[fragile]{Student exercise}
\protect\hypertarget{student-exercise}{}
Can use same approach to calculate conditional mean by multiple
\texttt{group\_by()} variables

\begin{itemize}
\tightlist
\item
  Just add additional variables within \texttt{group\_by()}
\end{itemize}

\begin{enumerate}
\tightlist
\item
  Calculate mean math test score (\texttt{x2txmtscor}), for each
  combination of parental education (\texttt{x2paredu}) and sex
  (\texttt{x2sex}).
\end{enumerate}
\end{frame}

\begin{frame}[fragile]{Student exercise solution}
\protect\hypertarget{student-exercise-solution}{}
\begin{enumerate}
\tightlist
\item
  Calculate mean math test score (\texttt{x2txmtscor}), for each
  combination of parental education (\texttt{x2paredu}) and sex
  (\texttt{x2sex})
\end{enumerate}

\begin{Shaded}
\begin{Highlighting}[]
\CommentTok{\#hsls\_small \%\textgreater{}\% count(x2sex)}

\NormalTok{hsls\_small }\OperatorTok{\%\textgreater{}\%}
\StringTok{  }\KeywordTok{group\_by}\NormalTok{(x2paredu,x2sex) }\OperatorTok{\%\textgreater{}\%}
\StringTok{  }\KeywordTok{summarise\_at}\NormalTok{(}\DataTypeTok{.vars =} \KeywordTok{vars}\NormalTok{(x2txmtscor),}
               \DataTypeTok{.funs =} \KeywordTok{funs}\NormalTok{(mean, }\DataTypeTok{.args =} \KeywordTok{list}\NormalTok{(}\DataTypeTok{na.rm =} \OtherTok{TRUE}\NormalTok{))) }\OperatorTok{\%\textgreater{}\%}
\StringTok{  }\KeywordTok{as\_factor}\NormalTok{()}
\end{Highlighting}
\end{Shaded}
\end{frame}

\hypertarget{guidelines-for-eda}{%
\subsection{Guidelines for EDA}\label{guidelines-for-eda}}

\begin{frame}[fragile]{Guidelines for ``EDA for data quality''}
\protect\hypertarget{guidelines-for-eda-for-data-quality}{}
Assme that your goal in ``EDA for data quality'' is to investigate
``input'' data sources and create ``analysis variables''

\begin{itemize}
\tightlist
\item
  Usually, your analysis dataset will incorporate multiple sources of
  input data, including data you collect (primary data) and/or data
  collected by others (secondary data)
\end{itemize}

While this is not a linear process, these are the broad steps I follow

\begin{enumerate}
\tightlist
\item
  Understand how input data sources were created

  \begin{itemize}
  \tightlist
  \item
    e.g., when working with survey data, have survey questionnaire and
    codebooks on hand
  \end{itemize}
\item
  For each input data source, identify the ``unit of analysis'' and
  which combination of variables uniquely identify observations
\item
  Investigate patterns in input variables
\item
  Create analysis variable from input variable(s)
\item
  Verify that analysis variable is created correctly through descriptive
  statistics that compare values of input variable(s) against values of
  the analysis variable
\end{enumerate}

\textbf{Always be aware of missing values}

\begin{itemize}
\tightlist
\item
  They will not always be coded as \texttt{NA} in input variables
\end{itemize}
\end{frame}

\begin{frame}{``Unit of analysis'' and which variables uniquely identify
observations}
\protect\hypertarget{unit-of-analysis-and-which-variables-uniquely-identify-observations}{}
``Unit of analysis'' refers to ``what does each observation represent''
in an input data source

\begin{itemize}
\tightlist
\item
  If each obs represents a student, you have ``student level data''
\item
  If each obs represents a student-course, you have ``student-course
  level data''
\item
  If each obs represents a school, you have ``school-level data''
\item
  If each obs represents a school-year, you have ``school-year level
  data''
\end{itemize}

How to identify unit of analysis

\begin{itemize}
\tightlist
\item
  data documentation
\item
  investigating the data set
\end{itemize}

We will go over syntax for identifying unit of analysis in subsequent
weeks
\end{frame}

\begin{frame}{Rules for variable creation}
\protect\hypertarget{rules-for-variable-creation}{}
Rules I follow for variable creation

\begin{enumerate}
\tightlist
\item
  \medskip Never modify ``input variable''; instead create new variable
  based on input variable(s)

  \begin{itemize}
  \tightlist
  \item
    Always keep input variables used to create new variables
  \end{itemize}
\item
  Investigate input variable(s) and relationship between input variables
\item
  Developing a plan for creation of analysis variable

  \begin{itemize}
  \tightlist
  \item
    e.g., for each possible value of input variables, what should value
    of analysis variable be?
  \end{itemize}
\item
  Write code to create analysis variable
\item
  Run descriptive checks to verify new variables are constructed
  correctly

  \begin{itemize}
  \tightlist
  \item
    Can ``comment out'' these checks, but don't delete them
  \end{itemize}
\item
  Document new variables with notes and labels
\end{enumerate}
\end{frame}

\begin{frame}[fragile]{Rules for variable creation}
\protect\hypertarget{rules-for-variable-creation-1}{}
\textbf{Task}:

\begin{itemize}
\tightlist
\item
  Create analysis for variable ses qunitile called \texttt{sesq5} based
  on \texttt{x4x2sesq5} that converts negative values to \texttt{NAs}
\end{itemize}

\begin{Shaded}
\begin{Highlighting}[]
\CommentTok{\#investigate input variable}
\NormalTok{hsls\_small }\OperatorTok{\%\textgreater{}\%}\StringTok{ }\KeywordTok{select}\NormalTok{(x4x2sesq5) }\OperatorTok{\%\textgreater{}\%}\StringTok{ }\KeywordTok{var\_label}\NormalTok{()}
\NormalTok{hsls\_small }\OperatorTok{\%\textgreater{}\%}\StringTok{ }\KeywordTok{select}\NormalTok{(x4x2sesq5) }\OperatorTok{\%\textgreater{}\%}\StringTok{ }\KeywordTok{val\_labels}\NormalTok{()}
\NormalTok{hsls\_small }\OperatorTok{\%\textgreater{}\%}\StringTok{ }\KeywordTok{select}\NormalTok{(x4x2sesq5) }\OperatorTok{\%\textgreater{}\%}\StringTok{ }\KeywordTok{count}\NormalTok{(x4x2sesq5)}
\NormalTok{hsls\_small }\OperatorTok{\%\textgreater{}\%}\StringTok{ }\KeywordTok{select}\NormalTok{(x4x2sesq5) }\OperatorTok{\%\textgreater{}\%}\StringTok{ }\KeywordTok{count}\NormalTok{(x4x2sesq5) }\OperatorTok{\%\textgreater{}\%}\StringTok{ }\KeywordTok{as\_factor}\NormalTok{()}

\CommentTok{\#create analysis variable}
\NormalTok{hsls\_small\_temp \textless{}{-}}\StringTok{ }\NormalTok{hsls\_small }\OperatorTok{\%\textgreater{}\%}\StringTok{ }
\StringTok{  }\KeywordTok{mutate}\NormalTok{(}\DataTypeTok{sesq5=}\KeywordTok{ifelse}\NormalTok{(x4x2sesq5}\OperatorTok{=={-}}\DecValTok{8}\NormalTok{,}\OtherTok{NA}\NormalTok{,x4x2sesq5)) }\CommentTok{\# approach 1}
\NormalTok{hsls\_small\_temp \textless{}{-}}\StringTok{ }\NormalTok{hsls\_small }\OperatorTok{\%\textgreater{}\%}\StringTok{ }
\StringTok{  }\KeywordTok{mutate}\NormalTok{(}\DataTypeTok{sesq5=}\KeywordTok{ifelse}\NormalTok{(x4x2sesq5}\OperatorTok{\textless{}}\DecValTok{0}\NormalTok{,}\OtherTok{NA}\NormalTok{,x4x2sesq5)) }\CommentTok{\# approach 2}

\CommentTok{\#verify}
\NormalTok{hsls\_small\_temp }\OperatorTok{\%\textgreater{}\%}\StringTok{ }\KeywordTok{group\_by}\NormalTok{(x4x2sesq5) }\OperatorTok{\%\textgreater{}\%}\StringTok{ }\KeywordTok{count}\NormalTok{(sesq5)}
\end{Highlighting}
\end{Shaded}
\end{frame}

\hypertarget{skip-patterns-in-survey-data}{%
\subsection{Skip patterns in survey
data}\label{skip-patterns-in-survey-data}}

\begin{frame}[fragile]{What are skip patterns}
\protect\hypertarget{what-are-skip-patterns}{}
Pretty easy to create an analysis variable based on a single input
variable

Harder to create analysis variables based on multiple input variables

\begin{itemize}
\tightlist
\item
  When working with survey data, even seemingly simple analysis
  variables require multiple input variables due to ``skip patterns''
\end{itemize}

What are ``skip patterns''?

\begin{itemize}
\tightlist
\item
  Response on a particular survey item determines whether respondent
  answers some set of subsequent questions
\item
  What are some examples of this?
\end{itemize}

Key to working with skip patterns

\begin{itemize}
\tightlist
\item
  Have the survey questionnaire on hand
\item
  Sometimes it appears that analysis variable requires only one input
  variable, but really depends on several input variables because of
  skip patterns

  \begin{itemize}
  \tightlist
  \item
    Don't just blindly turn ``missing'' and ``skips'' from survey data
    to \texttt{NAs} in your analysis variable
  \item
    Rather, trace why these ``missing'' and ``skips'' appear and decide
    how they should be coded in your analysis variable
  \end{itemize}
\end{itemize}
\end{frame}

\hypertarget{problem-set-8}{%
\section{Problem Set 8}\label{problem-set-8}}

\begin{frame}{Overview of problem set due next week}
\protect\hypertarget{overview-of-problem-set-due-next-week}{}
\textbf{Assignment}:

\begin{itemize}
\tightlist
\item
  create GPA from postsecondary transcript student-course level data
\end{itemize}

\textbf{Data source}: \href{https://nces.ed.gov/surveys/nls72/}{National
Longitudinal Study of 1972 (NLS72)}

\begin{itemize}
\tightlist
\item
  Follows 12th graders from 1972

  \begin{itemize}
  \tightlist
  \item
    Base year: 1972
  \item
    Follow-up surveys in: 1973, 1974, 1976, 1979, 1986
  \item
    Postsecondary transcripts collected in 1984
  \end{itemize}
\end{itemize}

\textbf{Why use such an old survey for this assignment?}

\begin{itemize}
\tightlist
\item
  NLS72 predates data privacy agreements; transcript data publicly
  available
\end{itemize}

\textbf{What we do to make assignment more manageable}

\begin{itemize}
\tightlist
\item
  last week's problem set created the input var: numgrade\_v2
\item
  we give you some hints/guidelines
\item
  but you are responsible for developing plan to create GPA vars and for
  executing plan (rather than us giving you step-by-step quations)
\end{itemize}

\textbf{Why this assignment?}

\begin{enumerate}
\tightlist
\item
  Give you more practice investigating data, cleaning data, creating
  variables that require processing across rows
\item
  Real world example of ``simple'' task with complex data management
  needs
\end{enumerate}

--\textgreater{}
\end{frame}

\end{document}
